\chapter{Mean-Field Theory in Hubbard lattices and magnetic ordering}\label{appendix:mean-field-hubbard}

In this Appendix the Mean-Field solutions to the Hubbard hamiltonian,
\[
	\hat H = 
	-t \sum_{\langle ij \rangle} \sum_\sigma \hat c_{i\sigma}^\dagger \hat c_{j\sigma}
	+ U \sum_i \hat n_{i\uparrow} \hat n_{i\downarrow}
	\qquad
	t, U  > 0
\]
are described.

\section{Ferromagnetic solution}

The two-dimensional square lattice extension of the two-sites model can be studied by the means of Mean Field Theory. We have:
\[
\begin{aligned}
	\hat n_{i\uparrow} \hat n_{i\downarrow} &= \left( \ev{\hat n_{i\uparrow}} + \delta \hat n_{i\uparrow} \right) \left( \ev{\hat n_{i\downarrow}} + \delta \hat n_{i\downarrow} \right) \\
	&\simeq \ev{\hat n_{i\uparrow}} \ev{\hat n_{i\downarrow}} +  \delta \hat n_{i\uparrow} \ev{\hat n_{i\downarrow}} + \ev{\hat n_{i\uparrow}} \delta \hat n_{i\downarrow} + \mathcal{O} \left(\delta n^2\right) \\
	&= - \ev{\hat n_{i\uparrow}} \ev{\hat n_{i\downarrow}} + \hat n_{i\uparrow} \ev{\hat n_{i\downarrow}} + \ev{\hat n_{i\uparrow}} \hat n_{i\downarrow} + \mathcal{O} \left(\delta n^2\right)
\end{aligned}
\]
where $\delta \hat n_{i\sigma} \equiv \hat n_{i\sigma} - \ev{\hat n_{i\sigma}}$ and orders higher than first have been ignored, assuming negligible fluctuations around the equilibrium magnetization. The first term of the above three can be neglected, being a pure energy shift. The Mean-Field Theory ferromagnetic solution prescribes an uniformly magnetized lattice,
\[
	\ev{\hat n_{i\uparrow}} = n+m
	\qquad
	\ev{\hat n_{i\downarrow}} = n-m
\]
where $n$ is the site electron density and $m$ is the density unbalance, leading to a magnetization per site $2m$. The mean-field hamiltonian with these substitutions becomes:
\[
\begin{aligned}
	\hat H &\simeq 
	-t \sum_{\langle ij \rangle} \sum_\sigma \hat c_{i\sigma}^\dagger \hat c_{j\sigma}
	+ U \sum_i \left[
		\hat n_{i\uparrow} \ev{\hat n_{i\downarrow}} + \ev{\hat n_{i\uparrow}} \hat n_{i\downarrow} 
	\right] \\
	&= -t \sum_{\langle ij \rangle} \sum_\sigma \hat c_{i\sigma}^\dagger \hat c_{j\sigma}
	+ nU \sum_i \left[
		\hat n_{i\uparrow} + \hat n_{i\downarrow} 
	\right] + mU \sum_i \left[
		\hat n_{i\uparrow} - \hat n_{i\downarrow} 
	\right]
\end{aligned}
\]
Fourier transforming,
\[
\begin{aligned}
	-t \sum_{\langle ij \rangle} \sum_\sigma \hat c_{i\sigma}^\dagger \hat c_{j\sigma} &= -2t \sum_{\mathbf{k}\sigma} \left[
		\cos(k_x) + \cos(k_y)
	\right] \hat n_{\mathbf{k}\sigma} \\
	nU \sum_i \left[
		\hat n_{i\uparrow} + \hat n_{i\downarrow} 
	\right] &= nU \sum_{\mathbf{k}\sigma} \hat n_{\mathbf{k}\sigma} \\
	mU \sum_i \left[
		\hat n_{i\uparrow} - \hat n_{i\downarrow} 
	\right] &= mU \sum_{\mathbf{k}\sigma} \left[
		\hat n_{\mathbf{k}\uparrow} - \hat n_{\mathbf{k}\downarrow}
	\right]
\end{aligned}
\]
having used adimensional lattice momenta. For a square lattice, the Brillouin Zone is delimited by
\[
	\mathbf{k} \in [-\pi,\pi] \times [-\pi,\pi]
\]
The hopping single-state energy is given by
\[
	\epsilon_{\mathbf{k}}^{(0)} = -2t \left[
		\cos(k_x) + \cos(k_y)
	\right]
\]
represented as a band in Fig.~\ref{fig:ferromagnetic-3d-band}. At $U=0$, the mean-field ferromagnetic state fills the band bottom-up. The single-state energy becomes:
\[
\begin{aligned}
	\epsilon_{\mathbf{k}\uparrow} &= U \left(
		n+m
	\right) - 2t \left[
		\cos(k_x) + \cos(k_y)
	\right] \\
	\epsilon_{\mathbf{k}\downarrow} &= U \left(
		n-m
	\right) - 2t \left[
		\cos(k_x) + \cos(k_y)
	\right]
\end{aligned}
\]
Now it is a matter of finding the optimal value for $m$, minimizing the total energy at fixed filling $\rho = 2n$. Consider the half-filling situation. An unpolarized system will have $n=1/4$, $m=0$. A perfectly up-ferromagnetic system, $n=1/4$, $m=1/4$. \todo

\begin{figure}
	\centering
	\begin{tikzpicture}
	\begin{axis}[
			axis on top,
			axis lines=center,
			xmin=-1.3, xmax=1.3,
			ymin=-1.3, ymax=1.3,
			zmin=-2.5, zmax=2.5,
			xtick={-1,1},
			ytick={-1,1},
			ztick={-2,2},
			xticklabels={$-\pi$,$\pi$},
			yticklabels={$-\pi$,$\pi$},
			zticklabels={$-2t$,$2t$},
			xlabel={$k_x$},
			ylabel={$k_y$},
			zlabel={$\epsilon_{\mathbf{k}}^{(0)}$},
			xlabel style={anchor=west},
			ylabel style={anchor=south west},
			zlabel style={anchor=south},
			width=\textwidth,
			view/az=15,
			view/el=30
		]
		\addplot3[
			domain=-1:0,
			color=tabred,
			dashed,
		]
		(
			{x},
			{-1-x},
			{0}
		);
		\addplot3[
			domain=0:1,
			color=tabred,
			dashed,
		]
		(
			{x},
			{-1+x},
			{0}
		);
		\addplot3[
			domain=-1:0,
			color=tabred,
			dashed,
		]
		(
			{x},
			{1+x},
			{0}
		);
		\addplot3[
			domain=0:1,
			color=tabred,
			dashed,
		]
		(
			{x},
			{1-x},
			{0}
		);
		
		\addplot3[
			domain=-1:1,
			samples=25,
			smooth,
			surf,
			opacity=0.2
		] { -cos(deg(pi*x))-cos(deg(pi*y)) };
		
		
	\end{axis}
\end{tikzpicture}
	\caption{Depiction of the Hubbard square lattice hopping band $\epsilon_{\mathbf{k}}^{(0)} = -2t[\cos(k_x) + \cos(k_y)]$. The red lines mark the zero-energy intersection.}
	\label{fig:ferromagnetic-3d-band}
\end{figure}

\section{Antiferromagnetic solution}