\chapter{Mean-Field Theory in Hubbard lattices}\label{appendix:mean-field-hubbard}

In this Appendix the Mean-Field solutions to the Hubbard hamiltonian,
\[
	\hat H = 
	-t \sum_{\langle ij \rangle} \sum_\sigma \hat c_{i\sigma}^\dagger \hat c_{j\sigma}
	+ U \sum_i \hat n_{i\uparrow} \hat n_{i\downarrow}
	\qquad
	t, U  > 0
\]
are described. The discussion is limited to the two-dimensional square lattice. The two-dimensional square lattice extension of the two-sites model can be studied by the means of Mean Field Theory. We have:
\[
\begin{aligned}
	\hat n_{i\uparrow} \hat n_{i\downarrow} &= \left( \ev{\hat n_{i\uparrow}} + \delta \hat n_{i\uparrow} \right) \left( \ev{\hat n_{i\downarrow}} + \delta \hat n_{i\downarrow} \right) \\
	&\simeq \ev{\hat n_{i\uparrow}} \ev{\hat n_{i\downarrow}} +  \delta \hat n_{i\uparrow} \ev{\hat n_{i\downarrow}} + \ev{\hat n_{i\uparrow}} \delta \hat n_{i\downarrow} + \mathcal{O} \left(\delta n^2\right) \\
	&= - \ev{\hat n_{i\uparrow}} \ev{\hat n_{i\downarrow}} + \hat n_{i\uparrow} \ev{\hat n_{i\downarrow}} + \ev{\hat n_{i\uparrow}} \hat n_{i\downarrow} + \mathcal{O} \left(\delta n^2\right)
\end{aligned}
\]
where $\delta \hat n_{i\sigma} \equiv \hat n_{i\sigma} - \ev{\hat n_{i\sigma}}$ and orders higher than first have been ignored, assuming negligible fluctuations around the equilibrium single-site population. The first term of the above three can be neglected at fixed particles number, being a pure energy shift. 

\section{Ferromagnetic solution}

The Mean-Field Theory ferromagnetic solution prescribes an uniformly magnetized lattice,
\[
	\ev{\hat n_{i\uparrow}} = n+m
	\qquad
	\ev{\hat n_{i\downarrow}} = n-m
\]
where $n$ is the site electron density and $m$ is the density unbalance, leading to a magnetization per site $2m$. The mean-field hamiltonian with these substitutions becomes:
\[
\begin{aligned}
	\hat H &\simeq 
	-t \sum_{\langle ij \rangle} \sum_\sigma \hat c_{i\sigma}^\dagger \hat c_{j\sigma}
	+ U \sum_i \left[
		\hat n_{i\uparrow} \ev{\hat n_{i\downarrow}} + \ev{\hat n_{i\uparrow}} \hat n_{i\downarrow} 
	\right] \\
	&= -t \sum_{\langle ij \rangle} \sum_\sigma \hat c_{i\sigma}^\dagger \hat c_{j\sigma}
	+ nU \sum_i \left[
		\hat n_{i\uparrow} + \hat n_{i\downarrow} 
	\right] + mU \sum_i \left[
		\hat n_{i\uparrow} - \hat n_{i\downarrow} 
	\right]
\end{aligned}
\]
Fourier transforming,
\[
\begin{aligned}
	-t \sum_{\langle ij \rangle} \sum_\sigma \hat c_{i\sigma}^\dagger \hat c_{j\sigma} &= -2t \sum_{\mathbf{k}\sigma} \left[
		\cos(k_x) + \cos(k_y)
	\right] \hat n_{\mathbf{k}\sigma} \\
	nU \sum_i \left[
		\hat n_{i\uparrow} + \hat n_{i\downarrow} 
	\right] &= nU \sum_{\mathbf{k}\sigma} \hat n_{\mathbf{k}\sigma} \\
	mU \sum_i \left[
		\hat n_{i\uparrow} - \hat n_{i\downarrow} 
	\right] &= mU \sum_{\mathbf{k}\sigma} \left[
		\hat n_{\mathbf{k}\uparrow} - \hat n_{\mathbf{k}\downarrow}
	\right]
\end{aligned}
\]
having used adimensional lattice momenta. For a square lattice, the Brillouin Zone is delimited by
\[
	\mathbf{k} \in [-\pi,\pi] \times [-\pi,\pi]
\]
The hopping single-state energy is given by
\[
	\epsilon_{\mathbf{k}}^{(0)} = -2t \left[
		\cos(k_x) + \cos(k_y)
	\right]
\]
represented as a band in Fig.~\ref{appfig:ferromagnetic-3d-band}. At $U=0$, the mean-field ferromagnetic state fills the band bottom-up. The single-state energy becomes:
\[
\begin{aligned}
	\epsilon_{\mathbf{k}\uparrow} &= U \left(
		n+m
	\right) - 2t \left[
		\cos(k_x) + \cos(k_y)
	\right] \\
	\epsilon_{\mathbf{k}\downarrow} &= U \left(
		n-m
	\right) - 2t \left[
		\cos(k_x) + \cos(k_y)
	\right]
\end{aligned}
\]
Now it is a matter of finding the optimal value for $m$, minimizing the total energy at fixed filling $\rho = 2n$. Notice that said minimization is performed parametrically varying the magnetization $m$, inside the ferromagnetic-polarized space. As it turns out, for strong local repulsion $U/t \gg 1$, antiferromagnetic ordering is preferred. Comparison is needed in order to assess which magnetic ordering is preferred.

Consider the half-filling situation. An unpolarized system will have $n=1/4$, $m=0$: this implies $\ev{\hat n_{i\uparrow}} = \ev{\hat n_{i\downarrow}} = 1/4$. A perfectly up-ferromagnetic system, $n=1/4$, $m=1/4$: then $\ev{\hat n_{i\uparrow}} = 1/2$ and $\ev{\hat n_{i\downarrow}} = 0$. \todo

\begin{figure}
	\centering
	\begin{tikzpicture}
	\begin{axis}[
			axis on top,
			axis lines=center,
			xmin=-1.3, xmax=1.3,
			ymin=-1.3, ymax=1.3,
			zmin=-2.5, zmax=2.5,
			xtick={-1,1},
			ytick={-1,1},
			ztick={-2,2},
			xticklabels={$-\pi$,$\pi$},
			yticklabels={$-\pi$,$\pi$},
			zticklabels={$-2t$,$2t$},
			xlabel={$k_x$},
			ylabel={$k_y$},
			zlabel={$\epsilon_{\mathbf{k}}^{(0)}$},
			xlabel style={anchor=west},
			ylabel style={anchor=south west},
			zlabel style={anchor=south},
			width=\textwidth,
			view/az=15,
			view/el=30,
		]
		\addplot3[
			domain=-1:0,
			color=tabred,
			dashed,
		]
		(
			{x},
			{-1-x},
			{0}
		);
		\addplot3[
			domain=0:1,
			color=tabred,
			dashed,
		]
		(
			{x},
			{-1+x},
			{0}
		);
		\addplot3[
			domain=-1:0,
			color=tabred,
			dashed,
		]
		(
			{x},
			{1+x},
			{0}
		);
		\addplot3[
			domain=0:1,
			color=tabred,
			dashed,
		]
		(
			{x},
			{1-x},
			{0}
		);
		
		\addplot3[
			domain=-1:1,
			samples=25,
			smooth,
			surf,
			opacity=0.2,
			colormap name=tab
		] { -cos(deg(pi*x))-cos(deg(pi*y)) };
	\end{axis}
\end{tikzpicture}
	\caption{Depiction of the Hubbard square lattice hopping band $\epsilon_{\mathbf{k}}^{(0)} = -2t[\cos(k_x) + \cos(k_y)]$. The red lines mark the zero-energy intersection.}
	\label{appfig:ferromagnetic-3d-band}
\end{figure}

{\color{tabred} Unclear: numerically, it turns out the paramagnetic phase ($m=0$) is preferred. Let $\Delta \equiv Um$ and ignore the constant contribution to energies $Un$: graphically, the $\uparrow$ band is shifted by $\Delta$, the $\downarrow$ band by $-\Delta$. At half-filling the Fermi energy remains fixed. For each quadrant (top view of the bands), the DoS is inversion-symmetric with respect to the anti-diagonal (red lines in Fig.~\ref{appfig:ferromagnetic-3d-band}), thus filling the bands bottom-up while performing the shifts should leave the total energy unchanged. Why is $m=0$ preferred?}

\section{Antiferromagnetic solution}

Consider now an AF mean-field solution. Let me change notation for a brief moment, indicating each site as
\[
	i \to \mathbf{r} = (x,y)
	\qquad
	x,y \in \mathbb{N}
\]
The mean-field AF solution at half-filling is the uniform-modulated magnetization
\[
	m_\mathbf{r} = (-1)^{x+y} m
	\qquad
	m \in [-1,1]
\]
and a mean-field Ansatz
\[
	\ev{\hat n_{\mathbf{r}\uparrow}} = n+m_\mathbf{r}
	\qquad
	\ev{\hat n_{\mathbf{r}\downarrow}} = n-m_\mathbf{r}
\]
With respect to the solution presented above, the only detail changing is the last term,
\[
	\hat H = -t \sum_{\langle \mathbf{r}\mathbf{r}' \rangle} \sum_\sigma \hat c_{\mathbf{r}\sigma}^\dagger \hat c_{\mathbf{r}'\sigma}
	+ nU \sum_\mathbf{r} \left[
		\hat n_{\mathbf{r}\uparrow} + \hat n_{\mathbf{r}\downarrow}
	\right] + mU \sum_\mathbf{r} (-1)^{x+y} \left[
		\hat n_{\mathbf{r}\uparrow} - \hat n_{\mathbf{r}\downarrow}
	\right]
\]
Fourier-transforming, the phase factor can be absorbed in the destruction operator inside of $\hat n_{\mathbf{r}\sigma}$:
\[
\begin{aligned}
	\sum_\mathbf{r} (-1)^{x+y} \hat n_{\mathbf{r}\sigma} &= \sum_\mathbf{r} (-1)^{x+y} \hat c_{\mathbf{r}\sigma}^\dagger \hat c_{\mathbf{r}\sigma} \\
	&= 
	\sum_\mathbf{r} e^{i \bm{\pi} \cdot \mathbf{r}}
	\frac{1}{N} \sum_{\mathbf{k} \in \mathrm{BZ}} e^{i \mathbf{k} \cdot \mathbf{r}} \hat c_{\mathbf{k}\sigma}^\dagger \frac{1}{N} \sum_{\mathbf{k}' \in \mathrm{BZ}} e^{-i \mathbf{k}' \cdot \mathbf{r}} \hat c_{\mathbf{k}'\sigma} \\
	&= \sum_{\mathbf{k} \in \mathrm{BZ}} \sum_{\mathbf{k}' \in \mathrm{BZ}} \hat c_{\mathbf{k}\sigma}^\dagger \hat c_{\mathbf{k}'\sigma} \frac{1}{N^2} \sum_\mathbf{r} e^{-i [\mathbf{k}' - (\mathbf{k} + \bm{\pi}) ] \cdot \mathbf{r}} \\
	&= \sum_{\mathbf{k} \in \mathrm{BZ}} \hat c_{\mathbf{k}\sigma}^\dagger \hat c_{\mathbf{k}+\bm{\pi}\sigma}
\end{aligned}
\]
where $\bm{\pi} = (\pi,\pi)$. It follows:
\[
mU \sum_\mathbf{r} (-1)^{x+y} \left[
		\hat n_{\mathbf{r}\uparrow} - \hat n_{\mathbf{r}\downarrow}
	\right] = \Delta \sum_{\mathbf{k} \in \mathrm{BZ}} \left[
		\hat c_{\mathbf{k}\uparrow}^\dagger \hat c_{\mathbf{k}+\bm{\pi}\uparrow} - \hat c_{\mathbf{k}\downarrow}^\dagger \hat c_{\mathbf{k}+\bm{\pi}\downarrow}
	\right]
	\qq{where}
	\Delta \equiv mU
\]
\begin{figure}
	\centering
	\subfloat[Alternative band.]{
		% It's easier to rotate coordinates and hide axis than
% define a tilted rectangular domain?

\begin{tikzpicture}
	\begin{axis}[
			axis on top,
			axis lines=none, % Support axis
			xmin=-1.3, xmax=1.3,
			ymin=-1.3, ymax=2.3,
			zmin=-2.5, zmax=2.5,
			view/az=65,
			view/el=30,
			width=0.7\textwidth
		]
		\addplot3[
			domain=-0.707:0.707,
			color=tabred,
		]
		(
			{x},
			{0.707},
			{0}
		);
		\addplot3[
			domain=-0.707:0.707,
			color=tabred,
		]
		(
			{x},
			{-0.707},
			{0}
		);
		\addplot3[
			domain y=-0.707:0.707,
			color=tabred,
		]
		(
			{0.707},
			{y},
			{0}
		);
		\addplot3[
			domain y=-0.707:0.707,
			color=tabred,
		]
		(
			{-0.707},
			{y},
			{0}
		);
		
		\draw[color=black, line width=0.2pt, -stealth]
			(0,0,0) -- (0.2,0.2,0) node[anchor=west, yshift=-0.2em]
				{\tiny $k_x$};
		\draw[color=black, line width=0.2pt, -stealth]
			(0,0,0) -- (-0.25,0.25,0) node[anchor=south, xshift=0.2em, yshift=-0.1em]
				{\tiny $k_y$};
		
		\node[color=tabred, anchor=south, yshift=0.2em]
			at (-0.707,0,0)
				{$\mathrm{MBZ}$};

		\draw[color=gray, -stealth]
			(0,0,0) -- (0,{sqrt(2)},0);

		\fill[color=tabred]
			(0,0,0) circle (1pt) node[anchor=east]
				{$\mathbf{0}$};
		
		\fill[color=tabblue]
			(0,{sqrt(2)},0) circle (1pt) node[anchor=west]
				{$\bm{\pi}$};
		
		\addplot3[
			domain=-0.707:0.707,
			y domain=-0.707:2.121,
			samples=25,
			smooth,
			surf,
			opacity=0.2,
			colormap name=tab
		] { -cos(deg(
				pi * (x-y)/sqrt(2)	% Rotated coordinates
			))-cos(deg(
				pi* (x+y)/sqrt(2)	% Rotated coordinates
			)) };
	\end{axis}
\end{tikzpicture}
		\label{appsubfig:alternative-ferromagnetic-3d-band}
	}
	\subfloat[Contour plot.]{
		\begin{tikzpicture}
	\begin{axis}[
			axis x line=center,
			axis y line=center,
			axis z line=none,
			xmin=-2.3, xmax=2.3,
			ymin=-2.3, ymax=2.3,
			zmin=-2.5, zmax=2.5,
			xtick={1},
			ytick={1},
			xticklabel={$\pi$},
			yticklabel={$\pi$},
			xlabel={$k_x$},
			ylabel={$k_y$},
			xlabel style={anchor=west},
			ylabel style={anchor=south},
			view/az=0,
			view/el=90,
			width=0.4\textwidth,
			height=0.4\textwidth
		]
		\addplot3[
			domain=-1:0,
			color=tabred,
			dashed,
		]
		(
			{x},
			{-1-x},
			{0}
		);
		\addplot3[
			domain=0:1,
			color=tabred,
			dashed,
		]
		(
			{x},
			{-1+x},
			{0}
		);
		\addplot3[
			domain=-1:0,
			color=tabred,
			dashed,
		]
		(
			{x},
			{1+x},
			{0}
		);
		\addplot3[
			domain=0:1,
			color=tabred,
			dashed,
		]
		(
			{x},
			{1-x},
			{0}
		);
		
		\addplot3[
			domain=-2:2,
			samples=25,
			smooth,
			contour filled={number=10},
			opacity=0.45,
		] { -cos(deg(pi*x))-cos(deg(pi*y)) };
		
		\node[color=tabred, anchor=south east, xshift=0.3em]
			at (-1,0,0)
				{\tiny $\mathrm{MBZ}$};
		
		\draw[color=gray, -stealth]
			(0,0,0) -- (1,1,0);
		
		\fill[color=tabred]
			(0,0,0) circle (0pt) node[anchor=south east]
				{$\mathbf{0}$};
		
		\fill[color=tabblue]
			(1,1,0) circle (1pt) node[anchor=west]
				{$\bm{\pi}$};
		
	\end{axis}

	% Fine tuned
	\fill[color=tabred]
		(2.21,2.21) circle (1pt);
	
\end{tikzpicture}
		\label{appsubfig:ferromagnetic-contour}
	}
	\caption{Alternative depiction of the Hubbard square lattice hopping band previously reported in Fig.~\ref{appfig:ferromagnetic-3d-band}. The Magnetic Brillouin Zone ($\mathrm{MBZ}$) is delimited by the zero-energy contour and is indicated in figure. As it is evident, energy sign flips by taking a $(\pi,\pi)$ translation in $\mathbf{k}$ space.}
	\label{appfig:alternative-ferromagnetic-band}
\end{figure}

Consider the band of Fig.~\ref{appfig:ferromagnetic-3d-band} at half-filling. As does \citeauthor{fabrizio2022course} \cite{fabrizio2022course}, the area delimited externally by the solid lines at zero energy is denominated ``Magnetic Brillouin Zone'' ($\mathrm{MBZ}$). The periodicity of $\mathbf{k}$ space guarantees that the full $\mathrm{BZ}$ can be taken as well to be the one of Fig.~\ref{appsubfig:alternative-ferromagnetic-3d-band}. Then:
\[
\begin{aligned}
	\sum_{\mathbf{k} \in \mathrm{BZ}}
	\hat c_{\mathbf{k}\uparrow}^\dagger \hat c_{\mathbf{k}+\bm{\pi}\uparrow} &= \sum_{\mathbf{k} \in \mathrm{MBZ}}
	\left[
		\hat c_{\mathbf{k}\uparrow}^\dagger \hat c_{\mathbf{k}+\bm{\pi}\uparrow} + \hat c_{\mathbf{k}+\bm{\pi}\uparrow}^\dagger \hat c_{\mathbf{k}+2\bm{\pi}\uparrow}
	\right] \\
	&= \sum_{\mathbf{k} \in \mathrm{MBZ}}
	\left[
		\hat c_{\mathbf{k}\uparrow}^\dagger \hat c_{\mathbf{k}+\bm{\pi}\uparrow} + \hat c_{\mathbf{k}+\bm{\pi}\uparrow}^\dagger \hat c_{\mathbf{k}\uparrow}
	\right]
\end{aligned}
\]
and the same applies for spin $\downarrow$. Periodicity by shifts $2\bm{\pi}$ has been used. Now, define the Nambu spinors:
\[
	\hat \Psi_{\mathbf{k}\sigma} \equiv \begin{bmatrix}
		\hat c_{\mathbf{k}\sigma}^\dagger \\
		\hat c_{\mathbf{k}+\bm{\pi}\sigma}^\dagger 
	\end{bmatrix}
\]
and a spin-wise gap,
\[
	\Delta_\uparrow = \Delta
	\qquad
	\Delta_\downarrow = -\Delta
\]
At fixed filling, the $U$ term is a pure energy shift, thus will be neglected. The kinetic term transforms as
\[
\begin{aligned}
	-t \sum_{\langle ij \rangle} \sum_\sigma \hat c_{i\sigma}^\dagger \hat c_{j\sigma} &= \sum_{\mathbf{k} \in \mathrm{BZ}} \sum_\sigma \epsilon_\mathbf{k}^{(0)} \hat c_{\mathbf{k}\sigma}^\dagger \hat c_{\mathbf{k}\sigma} \\ 
	&= \sum_{\mathbf{k} \in \mathrm{MBZ}} \sum_\sigma \left[
		\epsilon_\mathbf{k}^{(0)} \hat c_{\mathbf{k}\sigma}^\dagger \hat c_{\mathbf{k}\sigma}
		+ \epsilon_{\mathbf{k}+\bm{\pi}}^{(0)} \hat c_{\mathbf{k}+\bm{\pi}\sigma}^\dagger \hat c_{\mathbf{k}+\bm{\pi}\sigma}
	\right] \\
	&= \sum_{\mathbf{k} \in \mathrm{MBZ}} \sum_\sigma \epsilon_{\mathbf{k}}^{(0)}  \left[
		\hat c_{\mathbf{k}\sigma}^\dagger \hat c_{\mathbf{k}\sigma}
		- \hat c_{\mathbf{k}+\bm{\pi}\sigma}^\dagger \hat c_{\mathbf{k}+\bm{\pi}\sigma}
	\right] \\ 
\end{aligned}
\]
In the second passage, the sum over the full $\mathrm{BZ}$ was written considering that the entirety of the zone is given by all the points in the $\mathrm{MBZ}$ plus their conjugates obtained by a $\bm{\pi}$ shift in the flipped band. As depicted in Fig.~\ref{appsubfig:alternative-ferromagnetic-3d-band}, kinetic energy is anti-periodic in $\mathbf{k}$ space by a vector $\bm{\pi}$. This anti-periodicity accounts for the minus sign arising in the third passage. The hamiltonian is then given by:
\[
	\hat H = \sum_{\mathbf{k} \in \mathrm{MBZ}} \sum_\sigma \hat \Psi_{\mathbf{k}\sigma}^\dagger \hat h_{\mathbf{k}\sigma} \hat \Psi_{\mathbf{k}\sigma}
	\qq{being}
	\hat h_{\mathbf{k}\sigma} \equiv \begin{bmatrix}
		\epsilon_\mathbf{k}^{(0)} & \Delta_\sigma \\
		\Delta_\sigma & - \epsilon_\mathbf{k}^{(0)}
	\end{bmatrix}
\]
Notice: the Nambu hamiltonian is a $2\times2$ matrix over the $\mathrm{MBZ}$ -- which is half the full $\mathrm{BZ}$, coherently with a solution which essentially bipartites the lattice giving back a double sized unit cell.

The system ground-state \todo