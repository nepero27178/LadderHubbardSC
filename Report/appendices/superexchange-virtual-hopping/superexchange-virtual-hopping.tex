\chapter{Superexchange and virtual hopping in Hubbard lattices}\label{appendix:superexchange-virtual-hopping}

A key mechanism in \AF phase formation in Hubbard lattice is superexchange. The \AF phase is stabilized by spin fluctuations and second-order virtual hopping. The mechanism becomes clear enough by considering a $2$-sites Hubbard toy model.

\section{Virtual hopping in the $2$-sites Hubbard lattice}

\begin{figure}
	\centering
	\def\xSeparation{2}			% Lattice spacing
\def\ySeparation{1.5}		% Rows spacing
\def\angle{90}				% Arrows angle (0 is horizontal)
\def\arrowLength{0.5}		% Arrows length
\def\onSiteSeparation{0.1}  % Arrows horizontal separation on site
\begin{tikzpicture}
	
	% Lattice
	\foreach \i in {0,1,2}{
		\filldraw[color=lightgray, fill=lightgray] 
			(0,{\i*\ySeparation}) circle (1.5pt) 
			--++ 
			({\xSeparation},0) circle (1.5pt);
	}
	
	% Local repulsion
	\node[anchor=west, align=left, xshift=1cm]
		at (
			{0 + \xSeparation},
			{0 + 2 * \ySeparation}
		) {Local repulsion};
	
	\fill[color=tabred!80, path fading=fade out]
		(
			{0},
			{2 * \ySeparation}
		) circle (10pt) node (LocRep) {};
	% Label must be defined outside not to conflict with path fading
	
	\node[color=tabred, anchor=east, xshift=-0.25cm]
		at (LocRep) {\small $U$};
	
	\draw[color=tabblue, -stealth]
		(
			{0 - \onSiteSeparation/2 - \arrowLength/2 * cos(\angle)},
			{0 + 2 * \ySeparation - \arrowLength/2 * sin(\angle)}
		) --++ (
			{\arrowLength * cos(\angle)},
			{\arrowLength * sin(\angle)}
		);
	\draw[color=tabblue, stealth-]
		(
			{0 + \onSiteSeparation/2 - \arrowLength/2 * cos(\angle)},
			{0 + 2 * \ySeparation - \arrowLength/2 * sin(\angle)}
		) --++ (
			{\arrowLength * cos(\angle)},
			{\arrowLength * sin(\angle)}
		);
	
	% Single hop
	\node[anchor=west, align=left, xshift=1cm]
		at (
			{0 + \xSeparation},
			{0 + \ySeparation}
		) {Single hop};
	\draw[color=tabblue, -stealth]
		(
			{0 - \arrowLength/2 * cos(\angle)},
			{0 + \ySeparation - \arrowLength/2 * sin(\angle)}
		) --++ (
			{\arrowLength * cos(\angle)},
			{\arrowLength * sin(\angle)}
		);
	
	\draw[color=tabgreen, dashed] 
		(
			{0 + \arrowLength/2 * cos(\angle)},
			{\ySeparation}
		) edge [-stealth, bend left=30]
			node[midway, anchor=south]
				{\small $-t$}
		(
			{0 + \xSeparation + \arrowLength/2 * cos(\angle)},
			{\ySeparation}
		);
	
	% Double hop
	\node[anchor=west, align=left, xshift=1cm]
		at (
			{0 + \xSeparation},
			{0}
		) {Double hop};
	\draw[color=tabblue, -stealth]
		(
			{0 - \arrowLength/2 * cos(\angle)},
			{0 - \arrowLength/2 * sin(\angle)}
		) --++ (
			{\arrowLength * cos(\angle)},
			{\arrowLength * sin(\angle)}
		);
	\draw[color=tabblue, stealth-]
		(
			{0 + \xSeparation - \arrowLength/2 * cos(\angle)},
			{0 - \arrowLength/2 * sin(\angle)}
		) --++ (
			{\arrowLength * cos(\angle)},
			{\arrowLength * sin(\angle)}
		);
		
	\draw[color=tabgreen, dashed] 
		(
			{0 + \arrowLength/2 * cos(\angle)},
			{0 + \arrowLength/2 * sin(\angle)}
		) edge [-stealth, bend left=30]
			node[midway, anchor=south]
				{\small $-t$}
		(
			{0 + \xSeparation + \arrowLength/2 * cos(\angle)},
			{0 + \arrowLength/2 * sin(\angle)}
		)
		(
			{0 - \arrowLength/2 * cos(\angle)},
			{0 - \arrowLength/2 * sin(\angle)}
		) edge [stealth-, bend right=30]
			node[midway, anchor=north]
				{\small $-t$}
		(
			{0 + \xSeparation - \arrowLength/2 * cos(\angle)},
			{0 - \arrowLength/2 * sin(\angle)}
		);
\end{tikzpicture}
	\caption{Two sites Hubbard model.}
	\label{appfig:two-sites-hubbard-model}
\end{figure}

Consider the toy model:
\[
	\hat H = 
	-t \left\lbrace \hat c_{1\uparrow}^\dagger \hat c_{2\uparrow} + \hat c_{1\downarrow}^\dagger \hat c_{2\downarrow} + \mathrm{h}.\mathrm{c}. \right\rbrace
	+ U \left\lbrace \hat n_{1\uparrow} \hat n_{1\downarrow} + \hat n_{2\uparrow} \hat n_{2\downarrow} \right\rbrace
\]
with $i=1,2$ the site index. The two sites are represented in Fig.~\ref{appfig:two-sites-hubbard-model}. The two competing processes are: 
\begin{enumerate}
	\item electrons inter-sites hopping with amplitude $-t$;
	\item local repulsion $+U$, acting when two anti-aligned electrons reside on the same site; 
\end{enumerate}

For an half-filled system, the Hilbert space is six-dimensional. I use the notation $\ket{n_{1\uparrow} n_{1\downarrow} n_{2\uparrow} n_{2\downarrow}}$ to indicate the six computational basis states:
\[
\begin{aligned}
	\ket{\psi_1} &\equiv \ket{1010} \\
	\ket{\psi_2} &\equiv \ket{1100} \\
\end{aligned}
\qquad
\begin{aligned}
	\ket{\psi_3} &\equiv \ket{1001} \\
	\ket{\psi_4} &\equiv \ket{0110} \\
\end{aligned}
\qquad
\begin{aligned}
	\ket{\psi_5} &\equiv \ket{0011} \\
	\ket{\psi_6} &\equiv \ket{0101} \\
\end{aligned}
\]
For example, the top panel of Fig.~\ref{appfig:two-sites-hubbard-model} shows state $\ket{\psi_2}$. 

\subsection{Exact solution of the half-filled model}

The hamiltonian matrix is directly evaluated in this basis
\[
	H_{ij} = \bra{\psi_i}{\hat H}\ket{\psi_j}
	\quad\implies\quad
	H = \begin{bmatrix}
		0 &    &    &    &    &    \\
		&  U & -t & -t &    &    \\
		& -t &    &    & -t &    \\
		& -t &    &    & -t &    \\
		&    & -t & -t &  U &    \\
		&    &    &    &    &  0 \\  
	\end{bmatrix}
\]
Empty slots in the matrix stand for zeros. Evidently the states $\ket{\psi_1}$ (both up) and $\ket{\psi_6}$ (both down) are zero-energy eigenstates. These states cannot realize electrons hopping because of Pauli principle. The internal $4 \times 4$ matrix is readily diagonalized by the means of a change of basis $V$,
\[
	V \begin{bmatrix}
		 U & -t & -t &     \\
		-t &    &    & -t  \\
		-t &    &    & -t  \\
		   & -t & -t &  U  \\
	\end{bmatrix} V^\dagger = \begin{bmatrix}
	E^- &   & 	& 	 \\
	    & 0 & 	&	 \\
		&   & U & 	 \\
		&	&	& E^+
	\end{bmatrix}
\]

\begin{table}
	\centering
	\begin{tabular}{c c c}
		Structure & Eigenstate & Energy \\
		\midrule
		Spin-$1/2$ singlet & $\displaystyle \frac{\ket{\phi_3} - \ket{\phi_4}}{\sqrt{2}}$ & $\displaystyle E^- = \frac{U}{2} - \sqrt{\frac{U^2}{4} + 4t^2}$ \\
		&&\\ % Extra space
		Spin-$1/2$ triplet & $\ket{\phi_1}, \displaystyle \frac{\ket{\phi_3} + \ket{\phi_4}}{\sqrt{2}}, \ket{\phi_6}$ & 0 \\
		&&\\ % Extra space
		&$\displaystyle \frac{\ket{\phi_2}-\ket{\phi_5}}{\sqrt{2}}$ & $U$\\
		&&\\ % Extra space
		&$\displaystyle \frac{\ket{\phi_2}+\ket{\phi_5}}{\sqrt{2}}$ & $\displaystyle E^+ = \frac{U}{2} + \sqrt{\frac{U^2}{4} + 4t^2}$ \\
	\end{tabular}
	\caption{List of exact eigenstates and relative energies for the $2$-sites half-filled Hubbard model.}
	\label{apptab:two-sites-hubbard-model-eigenstates}
\end{table}

Tab.~\ref{apptab:two-sites-hubbard-model-eigenstates} shows the eigenvectors and relative eigenvalues obtained from diagonalization. The ground-state is the singlet state,
\[
	\frac{\ket{\phi_3} - \ket{\phi_4}}{\sqrt{2}} = \frac{\ket{1010} - \ket{0101}}{\sqrt{2}} = \frac{\ket{\uparrow\downarrow} - \ket{\downarrow\uparrow}}{\sqrt{2}}
\]
of energy
\[
	E^- = \frac{U}{2} - \sqrt{\frac{U^2}{4} + 4t^2} \simeq - \frac{4t^2}{U}
\]
the latter equality being true if $U \gg t$ (strong repulsion limit). The singlet state pairs with a spatially-symmetric (nodeless) wavefunction. The entire triplet (second row of Tab.~\ref{apptab:two-sites-hubbard-model-eigenstates}) remains at zero energy. Excited states are anti-symmetrized and symmetrized version of the polarized states $\ket{\phi_1}$ and $\ket{\phi_6}$.

\subsection{Virtual hopping}

The key feature of the singlet state is the one represented in the bottom panel of Fig.~\ref{appfig:two-sites-hubbard-model}: if the two electrons occupy separate sites and are anti-aligned, both ``see'' the other site as empty, thus free to hop to. \todo