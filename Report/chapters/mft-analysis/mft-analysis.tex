\chapter{Mean-field theory analysis of the EHM}\label{chap:mft-analysis}

This chapter is devoted to develop a Mean Field Theory (MFT) approximation of the Extended Hubbard model (EHM) of Eq.~\eqref{eq:extended-hubbard-model},
\[
	\hat H =
	\underbrace{
		-t \sum_{\langle ij \rangle} \sum_\sigma \hat c_{i\sigma}^\dagger \hat c_{j\sigma}
	}_{\hat H_t} \underbrace{
		+U \sum_i \hat n_{i\uparrow} \hat n_{i\downarrow}
		\vphantom{
			\sum_{\langle ij \rangle}
		}
	}_{\hat H_U}
	\underbrace{
		- V \sum_{\langle ij \rangle} \sum_{\sigma \sigma'} \hat n_{i\sigma} \hat n_{j\sigma'}
	}_{\hat H_V}
\]
Mean Field Theory (MFT) is a widely used and simple theoretical tool, often sufficient to describe the leading orders in phase transition phenomena of Many-Body Physics. Here MFT is employed to discuss both the effects of the non-local term $\hat H_V$ onto the AF phase, as well as the insurgence of anisotropic superconductivity -- following the path of Bardeen-Cooper-Schrieffer (BCS) theory in describing conventional $s$-wave superconductivity. As will be thoroughly described, the lattice spatial structure directly influences the topology of the gap function, giving rise to anisotropic pairing. Sec.~\ref{sec:mft-analysis-non-local-source} studies the non-local attraction $\hat H_V$ in real-space, describing how such interaction can contribute to the hamiltonian as a symmetry-breaking term in given channels. In the following sections, we move to specific channels and study theoretically and numerically the effect of non-local attraction.

\section{Mean-Field theory real space description}\label{sec:mft-analysis-non-local-source}

The general aim is to study the phase diagram of the model by comparing ground-state energies of different phases. The phases we consider here for the EHM are the Anti-Ferromagnetic ordering (AF), given by a non-uniform distribution of charge in each spin sector, and the Superconducting phase, described by a uniformly distributed charge allowing for Cooper pairing instabilities. For the EHM, three symmetries are ``brekable'':
\begin{enumerate}
	\item (Crystal) translational invariance. By breaking explicitly this symmetry, the obtained state must show a Charge-Density Wave (CDW) ordering;
	\item $\mathrm{U}(1)$ charge conservation. By breaking this symmetry, we allow for Cooper fluctuations originating superconductivity;
	\item $\mathrm{SU}(2)$ spin conservation. By breaking this symmetry, we take in all the processes not conserving spin (such as site hop \textit{plus} spin flip).
\end{enumerate}
As a general rule, $\mathrm{SU}(2)$ symmetry will be preserved while charge and space symmetries will be alternatively broken, in order to generate a pure AF phase or a pure superconducting phase. ``To preserve one symmetry'' means to set to zero all operators which break said symmetry, due to selection rules.

App.~\ref{appendix:mean-field-hubbard} describes in detail the MFT treatment of the pure Hubbard model, $\hat H_t + \hat H_U$; the key passage is there given by the approximation
\begin{equation}\label{eq:mft-pure-hubbard-wick-decomposition}
	\hat n_{i\uparrow} \hat n_{i\downarrow} \simeq \hat n_{i\uparrow} \langle \hat n_{i\downarrow} \rangle + \langle \hat n_{i\uparrow} \rangle \hat n_{i\downarrow} + (\mathrm{constants})
\end{equation}
from which the AF structure is simply recovered. However, to perform the above approximation coherently, we are implementing Wick's Theorem on the generic term:
\begin{equation}\label{eq:mft-extended-hubbard-wick-decomposition}
	\langle
		\hat c_{i\sigma}^\dagger \hat c_{j\sigma'}^\dagger \hat c_{j\sigma'} \hat c_{i\sigma}
	\rangle \simeq \underbrace{
		\langle 
			\hat c_{i\sigma}^\dagger \hat c_{j\sigma'}^\dagger
		\rangle \langle	
			\hat c_{j\sigma'} \hat c_{i\sigma} 
		\rangle 
	}_{\text{Cooper}}
	- 
	\underbrace{
		\langle 
			\hat c_{i\sigma}^\dagger \hat c_{j\sigma'}
		\rangle \langle	
			\hat c_{j\sigma'}^\dagger \hat c_{i\sigma} 
		\rangle 
	}_{\text{Fock}}
	+ 
	\underbrace{
		\langle 
		\hat c_{i\sigma}^\dagger \hat c_{i\sigma}
		\rangle \langle	
		\hat c_{j\sigma'}^\dagger \hat c_{j\sigma'} 
		\rangle
	}_{\text{Hartree}}
\end{equation}
As a first approximation, the theorem is assumed to hold (which, in a $\mathrm{BCS}$-like fashion, is equivalent to assuming for the ground-state to be a coherent state). Of the three terms above, the AF phase description fo the pure Hubbard Model only allows for the Hartree term to be non-zero. Cooper fluctuations are suppressed, because only the translational invariance is broken, and the Fock term is null as well because if $i=j$ and $\sigma'=\overline{\sigma}$ (as is for the local interaction, which contains the operator $\hat n_{i\uparrow} \hat n_{i\downarrow}$) the expectation values involved are describing a process breaking $\mathrm{SU}(2)$ spin symmetry. Thus, correctly, the Wick's decomposition of Eq.~\eqref{eq:mft-pure-hubbard-wick-decomposition} only involves Hartree-terms of Eq.~\eqref{eq:mft-extended-hubbard-wick-decomposition}. In general, however, the three terms need to be considered altogether: this is what is done in the next section.

\subsection{The non-local term as a source of symmetry-breaking interactions}

\begin{figure}
	\centering
	\begin{tikzpicture}
	\fill[color=lightgray] 
		(0,0) circle (1.5pt)
			node[anchor=south east, color=black]
				{$i$}
		(1,0) circle (1.5pt)
			node[anchor=west, color=black]
				{$i+\delta_x$}
		(-1,0) circle (1.5pt)
			node[anchor=east, color=black]
				{$i-\delta_x$}
		(0,1) circle (1.5pt)
			node[anchor=south, color=black]
				{$i+\delta_y$}
		(0,-1) circle (1.5pt)
			node[anchor=north, color=black]
				{$i-\delta_y$};
		
	\draw[color=lightgray] 
		(-1,0) -- (1,0)
		(0,-1) -- (0,1);
\end{tikzpicture}
	\caption{Schematic representation of the four NNs of a given site $i$ for a planar square lattice.}
	\label{fig:square-nearest-neighbors}
\end{figure}

Consider now the NN non-local term,
\begin{equation}\label{eq:extended-hubbard-nonlocal-interaction}
	\hat H_V \equiv - V \sum_{\langle ij \rangle} \sum_{\sigma \sigma'} \hat n_{i\sigma} \hat n_{j\sigma'}
\end{equation}
Evidently the hamiltonian can be decomposed in various spin terms,
\[
\begin{aligned}
	\hat H_V &= \sum_{\sigma \sigma'} \hat H_V^{\sigma\sigma'} \\
	&= \underbrace{
		\hat H_V^{\uparrow\uparrow} + \hat H_V^{\downarrow\downarrow}
	}_\text{Same-spin} + \underbrace{
		\hat H_V^{\uparrow\downarrow} + \hat H_V^{\downarrow\uparrow}
	}_\text{Opposite-spin}
\end{aligned}
\]
Evidently, to carry out a summation over nearest neighbors $\ev{ij}$ of a square lattice means precisely to sum over all links of the lattice. Then we can identify the generic opposite-spin (o.s.) term $\hat H_V^{\sigma \overline{\sigma}}$ as the one collecting the $\sigma$ operators of sublattice $\mathcal{S}_a$ and $\overline{\sigma}$ operators of sublattice $\mathcal{S}_b$. 
%At half-filling, as described in Sec.~\ref{sec:antiferromagnetic-ordering-hubbard}, the ground-state leading contribution will be the antiferromagnetic state, with the square lattice decomposed in two oppositely polarized square lattices with spacing increased by a factor $\sqrt{2}$. Then, it is to be expected that on this configuration the same-spin (s.s.) contributions are suppressed\footnote{
%	This is also due to superexchange stabilization: the triplet contribution to hamiltonian is suppressed, and this cancels out the ferromagnetic terms $\hat H_V^{\sigma\sigma}$ while privileging the singlet configuration of the anti-ferromagnetic terms $\hat H_V^{\sigma\overline{\sigma}}$.
%}. Anyways, the calculation will be carried out considering both terms.
The o.s. non-local interactions can be written as a sum of terms over just one of the two sublattices $\mathcal{S}_a$ and $\mathcal{S}_b$, oppositely polarized in the AF configuration (see Fig.~\ref{subfig:antiferromagnet-sublattices})
\[
	\begin{aligned}
		\hat H_V^{(\mathrm{o.s.})} &= \overbrace{
			\sum_{i \in \mathcal{S}_a} \hat h_V^{(i)}
		}^{\hat H_V^{\uparrow\downarrow}} + \overbrace{
			\sum_{i \in \mathcal{S}_b} \hat h_V^{(i)}
		}^{\hat H_V^{\downarrow\uparrow}} \hspace{0.1\textwidth}
		\hat h_V^{(i)} = -V \sum_{\ell = x,y} \left(
		\hat n_{i\uparrow} \hat n_{i+\delta_\ell \downarrow} + \hat n_{i\uparrow} \hat n_{i-\delta_\ell \downarrow} 
		\right)  \\
		&= \sum_{i \in \mathcal{S}} \hat h_V^{(i)}
	\end{aligned}
\]
Here the notation of Fig.~\ref{subfig:antiferromagnet-sublattices} is used. The two-dimensional lattice is regular-square. For each site $i$ in a given sublattice, the nearest neighbors sites are four -- all in the other sublattice. The notation used is $i \pm \delta_x$, $i \pm  \delta_y$ as in Fig.~\ref{fig:square-nearest-neighbors}. Similarly, the same-spin (s.s.) hamiltonian decomposes as
\[
	\hat H_V^{(\mathrm{s.s.})} = -V \sum_{i \in \mathcal{S}_a} \sum_{\ell = x,y} \sum_\sigma \left(
		\hat n_{i\sigma} \hat n_{i + \delta_\ell \sigma} + \hat n_{i\sigma} \hat n_{i - \delta_\ell \sigma} 
	\right) 
\]
Note here the summation only on one sublattice. 
%As will be shown in Sec.~\ref{subsec:mft-self-consitency-cooper-decomposition}, under MFT it makes sense to approximate
%\[
%	\hat H_V \simeq \hat{H}_V^{(\mathrm{AF})}
%\]
%thus neglecting ferromagnetic contribution to Cooper instability.
The non-local interaction contribution to energy, as a function of the $T=0$ full hamiltonian ground-state\footnote{
	Extensions to finite temperatures is simple: minimization must be carried out on free energy, while expectation values must be taken in a thermodynamic fashion.
} $\ket{\Psi}$, is given by
\[
\begin{aligned}
	E_V [\Psi] &= \mel{\Psi}{\hat H_V}{\Psi} \\
	&= -V \sum_{\ev{ij}} \sum_{\sigma\sigma'} \langle
		\hat n_{i\sigma} \hat n_{j\sigma'}
	\rangle \\
	&= \underbrace{
		-V \sum_{\ev{ij}} \sum_{\sigma} \langle
			\hat n_{i\sigma} \hat n_{j\sigma}
		\rangle
	}_{\mathrm{s.s.}} \underbrace{
		-V \sum_{\ev{ij}} \sum_{\sigma} \langle
			\hat n_{i\sigma} \hat n_{j\overline{\sigma}}
		\rangle
	}_{\mathrm{o.s.}}
\end{aligned}
\]
Shorthand notation has been used: $\mel{\Psi}{\cdot}{\Psi} = \langle \cdot \rangle$. 
The ground-state must realize the condition
\[
\fdv{}{\bra{\Psi}} E[\Psi] = 0
\]
being $E[\Psi]$ the total energy (made up of the three terms of couplings $t$, $U$ and $V$). {\color{tabred}[Expand derivation?]} The functional derivative must be carried out in a variational fashion including a Lagrange multiplier, the latter accounting for state-norm conservation, as is done normally in deriving the Hartree-Fock approximation for the eigenenergies of the electron liquid \cite{grosso2014solid, giuliani2005quantum}. 

\paragraph{Opposite-spin terms.}

Consider first the o.s. terms: take e.g. the term $\hat n_{i\uparrow} \hat n_{i + \delta_x \downarrow}$. As in Eq.~\eqref{eq:mft-extended-hubbard-wick-decomposition}, Wick's Theorem states that, if the expectation value is performed onto a coherent state,
\[
\begin{aligned}
	\langle 
		\hat n_{i\uparrow} \hat n_{i + \delta_x \downarrow}
	\rangle &= \langle 
		\hat c_{i\uparrow}^\dagger \hat c_{i + \delta_x \downarrow}^\dagger \hat c_{i + \delta_x \downarrow} \hat c_{i\uparrow} 
	\rangle \\
	&= 
	\underbrace{
		\langle 
			\hat c_{i\uparrow}^\dagger \hat c_{i + \delta_x \downarrow}^\dagger
		\rangle \langle	
			\hat c_{i + \delta_x \downarrow} \hat c_{i\uparrow} 
		\rangle 
	}_{\text{Cooper}}
	- 
	\underbrace{
		\langle 
			\hat c_{i\uparrow}^\dagger \hat c_{i + \delta_x \downarrow}
		\rangle \langle	
			\hat c_{i + \delta_x \downarrow}^\dagger \hat c_{i\uparrow} 
		\rangle 
	}_{\text{Fock}}
	+ 
	\underbrace{
		\langle 
			\hat c_{i\uparrow}^\dagger \hat c_{i\uparrow}
		\rangle \langle	
			\hat c_{i + \delta_x \downarrow}^\dagger \hat c_{i + \delta_x \downarrow} 
		\rangle
	}_{\text{Hartree}}
\end{aligned}
\]
Identical decompositions are given for all others NNs. Of the three terms above:
\begin{itemize}
	\item The Coooper term breaks $\mathrm{U}(1)$ charge symmetry, allowing for supeconducting instabilities;
	\item The Fock term breaks the $\mathrm{SU}(2)$ symmetry, because it accounts for a site hop \textit{plus} spin flip process;
	\item The Hartree term breaks translational invariance, because the mean-field to interact with is given by the local density;
\end{itemize}
Then, to look for AF instability only the Hartree term is to be accounted; instead, in superconducting instability only the Cooper term contributes. The Fock term we always assume to be suppressed, preserving $\mathrm{SU}(2)$ symmetry. Note that for superconducting instabilities, due to superexchange mechanism (as explained in App.~\ref{appendix:mean-field-hubbard}) the o.s. term account for singlet pairing as well as zero-spin triplet pairing. Which channel is preferred, is a matter of thermodynamic advantage.

\paragraph{Same-spin terms.} Consider then the same-spin terms: take e.g. the term $\hat n_{i\uparrow} \hat n_{i + \delta_x \uparrow}$. As above,
\[
\begin{aligned}
	\langle 
	\hat n_{i\uparrow} \hat n_{i + \delta_x \uparrow}
	\rangle &= \langle 
	\hat c_{i\uparrow}^\dagger \hat c_{i + \delta_x \uparrow}^\dagger \hat c_{i + \delta_x \uparrow} \hat c_{i\uparrow} 
	\rangle \\
	&= 
	\underbrace{
		\langle 
		\hat c_{i\uparrow}^\dagger \hat c_{i + \delta_x \uparrow}^\dagger
		\rangle \langle	
		\hat c_{i + \delta_x \uparrow} \hat c_{i\uparrow} 
		\rangle 
	}_{\text{Cooper}}
	- 
	\underbrace{
		\langle 
		\hat c_{i\uparrow}^\dagger \hat c_{i + \delta_x \uparrow}
		\rangle \langle	
		\hat c_{i + \delta_x \uparrow}^\dagger \hat c_{i\uparrow} 
		\rangle 
	}_{\text{Fock}}
	+ 
	\underbrace{
		\langle 
		\hat c_{i\uparrow}^\dagger \hat c_{i\uparrow}
		\rangle \langle	
		\hat c_{i + \delta_x \uparrow}^\dagger \hat c_{i + \delta_x \uparrow} 
		\rangle
	}_{\text{Hartree}}
\end{aligned}
\]
Identical consideration as in the above paragraph hold for each term. The only difference with the o.s. terms is given by the Fock term: since the spin-flip process is absent, now the Fock fluctuations actually contribute to the AF phase as an effective NN hopping term {\color{tabred}[Unclear: does the renormalization happen also for the Cooper phase?]}. As a final remark, notice that the superconducting instabilities of the s.s. terms account only for triplet pairing. The only possible superconducting ordering established by the means of these terms is odd in real space. Then $s$-wave and $d$-wave superconductivity cannot establish in this channel; $p_\ell$-wave superconductivity, instead, can.

\subsection{Reciprocal-space transform of the non-local interaction}

It is useful to derive analytically the reciprocal-space form of the non-local attraction. Consider a generic bond, say, the one connecting sites $j$ and $j\pm\delta_\ell$ (variable $i$ is here referred to as the imaginary unit to avoid confusion). $\mathbf{x}_j$ is the $2$D notation for the position of site $j$, while $\bm{\delta}_\ell$ is the $2$D notation for the lattice spacing previously indicated as $\delta_\ell$. Fourier transform it according to the convention
\[
\hat c_{j\sigma} = \frac{1}{\sqrt{L_xL_y}} \sum_{\mathbf{k} \in \mathrm{BZ}} e^{-i \mathbf{k} \cdot \mathbf{x}_j} \hat c_{\mathbf{k}\sigma}
\]
Then:
\[
\begin{aligned}
	\hat n_{j\sigma} \hat n_{j \pm \delta_\ell \sigma'} &= \hat c_{j\sigma}^\dagger \hat c_{j \pm \delta_\ell \sigma'}^\dagger \hat c_{j \pm \delta_\ell \sigma'} \hat c_{j\sigma} \\
	&= \frac{1}{(L_xL_y)^2} \sum_{\nu=1}^4 \sum_{\mathbf{k}_\nu \in \mathrm{BZ}} e^{i \left[ (\mathbf{k}_1 + \mathbf{k}_2) - (\mathbf{k}_3 + \mathbf{k}_4) \right] \cdot \mathbf{x}_j} e^{\pm i(\mathbf{k}_2-\mathbf{k}_3) \cdot \bm{\delta}_\ell}  \hat c_{\mathbf{k}_1 \sigma}^\dagger \hat c_{\mathbf{k}_2 \sigma'}^\dagger \hat c_{\mathbf{k}_3 \sigma'} \hat c_{\mathbf{k}_4\sigma}
\end{aligned}
\]
Then, the interaction at site $j$, spin $\sigma$ with its NNs at spin $\sigma'$ -- indicated as $(j\sigma\sigma')$ -- is given by
\[
\begin{aligned}
	(j\sigma\sigma') &= -V \sum_{\ell = x,y} \sum_{\delta = \pm \delta_\ell} \hat n_{j\sigma} \hat n_{j \pm \delta_\ell \sigma'} \\
	&= -\frac{V}{(L_xL_y)^2} \sum_{\ell = x,y} \sum_{\nu=1}^4 \sum_{\mathbf{k}_\nu \in \mathrm{BZ}} e^{i \left[ (\mathbf{k}_1 + \mathbf{k}_2) - (\mathbf{k}_3 + \mathbf{k}_4) \right] \cdot \mathbf{x}_j} \\
	&\hspace{0.3\textwidth} \times \left(
	e^{ i(\mathbf{k}_2-\mathbf{k}_3) \cdot \bm{\delta}_\ell} + e^{ -i(\mathbf{k}_2-\mathbf{k}_3) \cdot \bm{\delta}_\ell} 
	\right)
	\hat c_{\mathbf{k}_1 \sigma}^\dagger \hat c_{\mathbf{k}_2 \sigma'}^\dagger \hat c_{\mathbf{k}_3 \sigma'} \hat c_{\mathbf{k}_4\sigma} \\
	&= -\frac{2V}{(L_xL_y)^2} \sum_{\ell = x,y} \sum_{\nu=1}^4 \sum_{\mathbf{k}_\nu \in \mathrm{BZ}} e^{i \left[ (\mathbf{k}_1 + \mathbf{k}_2) - (\mathbf{k}_3 + \mathbf{k}_4) \right] \cdot \mathbf{x}_j} \cos\left[
	(\mathbf{k}_2-\mathbf{k}_3) \cdot \bm{\delta}_\ell
	\right]	\hat c_{\mathbf{k}_1 \sigma}^\dagger \hat c_{\mathbf{k}_2 \sigma'}^\dagger \hat c_{\mathbf{k}_3 \sigma'} \hat c_{\mathbf{k}_4\sigma}
\end{aligned}
\]
The full non-local interaction is given by summing over all sites of $\mathcal{S}_a$ (which is, half the sites of $\mathcal{S}$). This gives back momentum conservation,
\[
	\frac{1}{L_xL_y} \sum_{j \in \mathcal{S}_a} e^{i \left[ (\mathbf{k}_1 + \mathbf{k}_2) - (\mathbf{k}_3 + \mathbf{k}_4) \right] \cdot \mathbf{x}_j} = \frac{1}{2} \delta_{\mathbf{k}_1 + \mathbf{k}_2 = \mathbf{k}_3 + \mathbf{k}_4}
\]
Let $\mathbf{k}_1 + \mathbf{k}_2 = \mathbf{k}_3 + \mathbf{k}_4 = \mathbf{K}$, and define $\mathbf{k}$, $\mathbf{k}'$ such that
\[
	\mathbf{k}_1 \equiv \mathbf{K} + \mathbf{k} 
	\qquad
	\mathbf{k}_2 \equiv \mathbf{K} - \mathbf{k} 
	\qquad
	\mathbf{k}_3 \equiv \mathbf{K} - \mathbf{k}' 
	\qquad
	\mathbf{k}_4 \equiv \mathbf{K} + \mathbf{k}'
	\qquad
	\delta \mathbf{k} \equiv \mathbf{k}-\mathbf{k}'
\]
Sums over these variables must be intended as over the Brillouin Zone ($\mathrm{BZ}$). Then, finally
\begin{align}
	\hat H_V &= \sum_{j \in \mathcal{S}_a} \sum_{\sigma\sigma'} (j\sigma\sigma') \nonumber \\
	&= - \frac{V}{L_x L_y} \sum_{\ell = x,y} \sum_{\mathbf{K}, \mathbf{k}, \mathbf{k}'} \cos\left(
		\delta \mathbf{k} \cdot \bm{\delta}_\ell
	\right)	\hat c_{\mathbf{K}+\mathbf{k} \sigma}^\dagger \hat c_{\mathbf{K}-\mathbf{k} \sigma'}^\dagger \hat c_{\mathbf{K}-\mathbf{k}' \sigma'} \hat c_{\mathbf{K}+\mathbf{k}'\sigma} \nonumber \\
	&= - \frac{V}{L_x L_y} \sum_{\mathbf{K}, \mathbf{k}, \mathbf{k}'} \left[
	\cos \left(
		\delta k_x
	\right)	+ \cos \left(
		\delta k_y
	\right)	
	\right]	\hat c_{\mathbf{K}+\mathbf{k} \sigma}^\dagger \hat c_{\mathbf{K}-\mathbf{k} \sigma'}^\dagger \hat c_{\mathbf{K}-\mathbf{k}' \sigma'} \hat c_{\mathbf{K}+\mathbf{k}'\sigma} \label{eq:reciprocal-space-non-local-interaction-explicit}
\end{align}
Different Wick contraction schemes lead to different results:
\begin{align}
	&\wick{
		\c
		c_{\mathbf{K}+\mathbf{k} \sigma}^\dagger 
		\c
		c_{\mathbf{K}-\mathbf{k} \sigma'}^\dagger c_{\mathbf{K}-\mathbf{k}' \sigma'} c_{\mathbf{K}+\mathbf{k}'\sigma}
	} &&\text{Cooper contraction} \label{eq:reciprocal-space-cooper-contraction} \\
	&\wick{
		\c
		c_{\mathbf{K}+\mathbf{k} \sigma}^\dagger 
		c_{\mathbf{K}-\mathbf{k} \sigma'}^\dagger
		\c
		c_{\mathbf{K}-\mathbf{k}' \sigma'} c_{\mathbf{K}+\mathbf{k}'\sigma}
	} &&\text{Fock contraction} \label{eq:reciprocal-space-fock-contraction} \\
	&\wick{
		\c
		c_{\mathbf{K}+\mathbf{k} \sigma}^\dagger 
		c_{\mathbf{K}-\mathbf{k} \sigma'}^\dagger c_{\mathbf{K}-\mathbf{k}' \sigma'}
		\c
		c_{\mathbf{K}+\mathbf{k}'\sigma}
	} &&\text{Hartree contraction} \label{eq:reciprocal-space-hartree-contraction}
\end{align}
%To contract means to take an average over the ground state, which by symmetry must show zero net momentum and preserve $\mathrm{SU}(2)$ symmetry. This translates in the following requirements for different schemes:
%\begin{align}
%	&\wick{
%		\c c^\dagger_{\mathbf{k}_1\sigma_1}
%		\c c^\dagger_{\mathbf{k}_2\sigma_2}
%	} \neq 0
%	&\text{iff $\mathbf{k}_1 + \mathbf{k}_2 = \mathbf{0}$ and $\sigma_1 = \overline{\sigma_2}$} \label{eq:reciprocal-space-dagdag-requirements} \\
%	&\wick{
%		\c c^\dagger_{\mathbf{k}_1\sigma_1}
%		\c c_{\mathbf{k}_2\sigma_2}
%	} \neq 0
%	&\text{iff $\mathbf{k}_1 - \mathbf{k}_2 = \mathbf{0}$ and $\sigma_1 = \sigma_2$} \label{eq:reciprocal-space-dagnodag-requirements}
%\end{align}
%{\color{tabred}[Not sure]}
%which will be used later.
\todo

\section{Anti-Ferromagnetic instability} 

In this section, the effect of the non-local interaction on the antiferromagnetic phase is discussed. The MFT derivation for the ``standard'' Hubbard Model is discussed in App.~\ref{appendix:mean-field-hubbard}. Recalling the main results, the Anti-Ferromagnetic phase specified by the Ansatz \eqref{appeq:af-mean-field-ansatz} (which is explicitly breaking translational invariance in each spin sector, while preserving $\mathrm{SU}(2)$ and $\mathrm{U}(1)$ symmetries) reduces the hamiltonian to the form of Eq.~\eqref{appeq:hubbard-mean-field-hamiltonian}
\[
	\hat H_t + \hat H_U \stackrel{\mathrm{MFT}}{\simeq} -t \sum_{\langle \mathbf{r}\mathbf{r}' \rangle} \sum_\sigma \hat c_{\mathbf{r}\sigma}^\dagger \hat c_{\mathbf{r}'\sigma}
	+ nU \sum_\mathbf{r} \left[
	\hat n_{\mathbf{r}\uparrow} + \hat n_{\mathbf{r}\downarrow}
	\right] - mU \sum_\mathbf{r} (-1)^{x+y} \left[
	\hat n_{\mathbf{r}\uparrow} - \hat n_{\mathbf{r}\downarrow}
	\right]
\]
In reciprocal space, the hamiltonian decomposes as in Eq.~\eqref{appeq:hubbard-bogoliubov-hamiltonian},
\[
	\hat H_t + \hat H_U \stackrel{\mathrm{MFT}}{\simeq} \sum_{\mathbf{k} \in \mathrm{MBZ}} \sum_\sigma \hat \Psi_{\mathbf{k}\sigma}^\dagger h_{\mathbf{k}\sigma} \hat \Psi_{\mathbf{k}\sigma}
	\qq{being}
	h_{\mathbf{k}\sigma} \equiv \begin{bmatrix}
		\epsilon_\mathbf{k} & -\Delta_\sigma \\
		-\Delta_\sigma & - \epsilon_\mathbf{k}
	\end{bmatrix}
\]
and $\Delta_\uparrow = mU$, $\Delta_\downarrow = -mU$. Nambu spinorial formulation is used,
\[
	\hat \Psi_{\mathbf{k}\sigma} \equiv \begin{bmatrix}
		\hat c_{\mathbf{k}\sigma} \\
		\hat c_{\mathbf{k}+\bm{\pi}\sigma} 
	\end{bmatrix}
\]
and the free electrons energy is simply the tight binding energy
\[
	\epsilon_\mathbf{k} = -2t \left[
		\cos (k_x) + \cos (k_y)
	\right]
\]
which is spin-invariant. The MFT description of the model reduces to a gas of free ``$\gamma$-fermions'', described by the Nambu spinor of Eq.~\eqref{appeq:af-diagonalized-nambu-spinor},
\[
	\hat\Gamma_{\mathbf{k}\sigma} = W_{\mathbf{k}\sigma} \hat \Psi_{\mathbf{k}\sigma} = \begin{bmatrix}
		\hat \gamma_{\mathbf{k}\sigma}^{(-)} \\ \hat \gamma_{\mathbf{k}\sigma}^{(+)}
	\end{bmatrix}
\]
where
\[
	W_{\mathbf{k}\sigma} = \begin{bmatrix}
		- \sin \theta_{\mathbf{k}\sigma} & - \cos \theta_{\mathbf{k}\sigma} \\ 
		\cos \theta_{\mathbf{k}\sigma} & - \sin \theta_{\mathbf{k}\sigma}
	\end{bmatrix}
	\qq{and}
	\sin 2\theta_{\mathbf{k}\sigma} \equiv \frac{\Delta_\sigma}{E_\mathbf{k}}
\]
These fermions populate the two bands $\pm E_\mathbf{k} = \sqrt{\epsilon_\mathbf{k}^2 + \Delta^2}$.

Consider now the non-local interaction $\hat H_V$: since only translational invariance is broken in the AF phase, the only relevant contributions coming from Wick's decomposition are Hartree terms and the same-spin Fock term. The net effect obtained by including this interaction, as I will explain, is a renormalization of the various quantities,
\[
	\epsilon_\mathbf{k} \to \tilde{\epsilon}_{\mathbf{k}\sigma}
	\qquad
	E_\mathbf{k} \to \tilde{E}_{\mathbf{k}\sigma}
	\qquad
	\Delta_\sigma \to \tilde{\Delta}_{\mathbf{k}\sigma}
	\qquad
	\theta_{\mathbf{k}\sigma} \to \tilde{\theta}_{\mathbf{k}\sigma}
\]
The band energies renormalization is simply
\[
	\tilde{E}_{\mathbf{k}\sigma} \equiv \sqrt{\tilde{\epsilon}_{\mathbf{k}\sigma}^2 + |\tilde{\Delta}_{\mathbf{k}\sigma}|^2}
\]
The physical behavior is the same as for the pure Hubbard model. In next sections the different contributions to renormalization are treated.

\subsection{Hartree renormalization of chemical potential and gap}

The same-spin and opposite-spin non-local Hartree terms are
\[
	\underbrace{
		-V \sum_{\ev{ij}} \sum_\sigma \left[
			\langle 
				\hat n_{i\sigma}
			\rangle \hat n_{j\sigma} + \hat n_{i\sigma} \langle 
				\hat n_{j\sigma}
			\rangle
		\right]
	}_{\mathrm{s.s.}}
	\underbrace{
		-V \sum_{\ev{ij}} \sum_\sigma \left[
			\langle
				\hat n_{i\sigma}
			\rangle \hat n_{j\overline{\sigma}} + \hat n_{i\sigma} \langle 
				\hat n_{j\overline{\sigma}}
			\rangle
		\right]
	}_{\mathrm{o.s.}}
\]
Let $i \to \mathbf{r} = (x,y)$ and $j \to \mathbf{r}' = (x',y')$. Then, using the Ansatz of Eq.~\ref{appeq:af-mean-field-ansatz}, summarized as
\[
	\langle \hat n_{\mathbf{r}\sigma} \rangle = n - (-1)^{x+y+\delta_{\sigma=\uparrow}} m
\]
we get
\begin{multline*}
	\overbrace{
		-nV \sum_{\ev{\mathbf{r}\mathbf{r}'}} \sum_\sigma \left[
			\hat n_{\mathbf{r}'\sigma} + \hat n_{\mathbf{r}\sigma}
		\right] + mV \sum_{\ev{\mathbf{r}\mathbf{r}'}} \sum_\sigma (-1)^{\delta_{\sigma=\uparrow}} \left[
			(-1)^{x'+y'} \hat n_{\mathbf{r}'\sigma} + (-1)^{x+y} \hat n_{\mathbf{r}\sigma}
		\right]
	}^{\mathrm{s.s.}} \\
	\underbrace{
		-nV \sum_{\ev{\mathbf{r}\mathbf{r}'}} \sum_\sigma \left[
			\hat n_{\mathbf{r}'\overline{\sigma}} + \hat n_{\mathbf{r}\sigma}
		\right] + mV \sum_{\ev{\mathbf{r}\mathbf{r}'}} \sum_\sigma \left[
			(-1)^{x'+y'+\delta_{\overline{\sigma}=\uparrow}} \hat n_{\mathbf{r}'\overline{\sigma}} + (-1)^{x+y+\delta_{\sigma=\uparrow}} \hat n_{\mathbf{r}\sigma}
		\right]
	}_{\mathrm{o.s.}}
\end{multline*}
For a square lattice, if $\mathbf{r} = (x,y)$ and $\mathbf{r}' = (x',y')$ are NNs evidently
\[
	(-1)^{x'+y'} = (-1)^{x+y+1}
\]
Moreover,
\[
	(-1)^{\delta_{\overline{\sigma}=\uparrow}} = (-1)^{\delta_{\sigma=\uparrow}+1}
\]
We obtain
\begin{multline}
	\overbrace{
		-nV \sum_{\ev{\mathbf{r}\mathbf{r}'}} \sum_\sigma \left[
			\hat n_{\mathbf{r}\sigma} + \hat n_{\mathbf{r}'\sigma}
		\right] + mV \sum_{\ev{\mathbf{r}\mathbf{r}'}} \sum_\sigma (-1)^{x+y+\delta_{\sigma=\uparrow}} \left[
			\hat n_{\mathbf{r}\sigma} - \hat n_{\mathbf{r}'\sigma}
		\right]
	}^{\mathrm{s.s.}} \\
	\underbrace{
		-nV \sum_{\ev{\mathbf{r}\mathbf{r}'}} \sum_\sigma \left[
			\hat n_{\mathbf{r}\sigma} + \hat n_{\mathbf{r}'\overline{\sigma}}
		\right] + mV \sum_{\ev{\mathbf{r}\mathbf{r}'}} \sum_\sigma (-1)^{x+y+\delta_{\sigma=\uparrow}} \left[
			\hat n_{\mathbf{r}\sigma} + \hat n_{\mathbf{r}'\sigma}
		\right]
	}_{\mathrm{o.s.}}
	\label{eq:af-hartree-renormalization-intermediate}
\end{multline}
Now, since
\[
	\sum_{\ev{\mathbf{r}\mathbf{r}'}} \sum_\sigma \left[
		\hat n_{\mathbf{r}\sigma} + \hat n_{\mathbf{r}'\sigma}
	\right] = \sum_{\ev{\mathbf{r}\mathbf{r}'}} \sum_\sigma \left[
		\hat n_{\mathbf{r}\sigma} + \hat n_{\mathbf{r}'\overline{\sigma}}
	\right] = z \hat N
\]
with $z=4$ the lattice coordination factor, indicating the number of NNs per site, then the first and third terms of Expr.~\eqref{eq:af-hartree-renormalization-intermediate} contribute to a pure chemical potential shift. The renormalized chemical potential is:
\begin{equation}\label{eq:af-hartree-chemical-potential-renormalization}
	\tilde{\mu} \equiv \mu + 2znV
\end{equation}
The second and fourth terms of Expr.~\eqref{eq:af-hartree-renormalization-intermediate} are to be reduced to a renormalization of the gap function. Explicitly,
\begin{multline}
	mV \sum_{\ev{\mathbf{r}\mathbf{r}'}} \sum_\sigma (-1)^{x+y+\delta_{\sigma=\uparrow}} \left[
		\hat n_{\mathbf{r}\sigma} - \hat n_{\mathbf{r}'\sigma}
	\right] + mV \sum_{\ev{\mathbf{r}\mathbf{r}'}} \sum_\sigma (-1)^{x+y+\delta_{\sigma=\uparrow}} \left[
		\hat n_{\mathbf{r}\sigma} + \hat n_{\mathbf{r}'\sigma}
	\right] \\
	= -2zmV \sum_\mathbf{r} (-1)^{x+y} \left[
		\hat n_{\mathbf{r}\uparrow} - \hat n_{\mathbf{r}\downarrow}
	\right] \label{eq:af-hartree-renormalization-intermediate-2}
\end{multline}
Consider now the last term of the pure Hubbard model under MFT approximations of Eq.~\eqref{appeq:hubbard-mean-field-hamiltonian},
\[
	- mU \sum_\mathbf{r} (-1)^{x+y} \left[
		\hat n_{\mathbf{r}\uparrow} - \hat n_{\mathbf{r}\downarrow}
	\right]
	\qq{(Local gap)}
\]
Expr.~\eqref{eq:af-hartree-renormalization-intermediate-2} is formally identical, thus we obtain a contribution to the renormalization of the AF gap,
\begin{equation}\label{eq:af-hartree-gap-os-renormalization}
	\Delta \to \Delta + 2zmV + (\text{s.s. contribution})
\end{equation}
This, together with Eq.~\eqref{eq:af-hartree-chemical-potential-renormalization}, concludes the non-local Hartree reparametrization of the hamiltonian. Next section is devoted to analyzing the effect of the Fock term.

\subsection{Fock renormalization of the hopping amplitude}

From Wick's decomposition of $\hat H_V$, the only allowed Fock term comes from the same-spin part due to $\mathrm{SU}(2)$ symmetry selection rules. Said hamiltonian term is
\[
	V \sum_{\ev{ij}} \sum_\sigma \left[
		\langle
			\hat c_{i\sigma}^\dagger \hat c_{j\sigma}
		\rangle \hat c_{j\sigma}^\dagger  \hat c_{i\sigma} + \mathrm{h.c.}
	\right]
\]
(note the $+$ sign in front of it). A bond-wise hopping amplitude can be defined,
\[
	\tilde{t}_{ij\sigma} \equiv t - V \langle
		\hat c_{j\sigma}^\dagger \hat c_{i\sigma}
	\rangle
\]
Then:
\[
	\hat H_t + V \sum_{\ev{ij}} \sum_\sigma \left[
		\langle
			\hat c_{i\sigma}^\dagger \hat c_{j\sigma}
		\rangle \hat c_{j\sigma}^\dagger  \hat c_{i\sigma} + \mathrm{h.c.}
	\right] = - \sum_{\ev{ij}} \sum_\sigma 
	\left[
		\tilde{t}_{ij\sigma} \hat c_{i\sigma}^\dagger \hat c_{j\sigma} + \mathrm{h.c.}
	\right]
\]
In reciprocal space, the effective hopping must be transformed as well. Consider the Fourier Transform given in Eq.~\eqref{eq:reciprocal-space-non-local-interaction-explicit},
\begin{multline}
	V \sum_{\ev{ij}} \sum_\sigma \left[
		\langle
			\hat c_{i\sigma}^\dagger \hat c_{j\sigma}
		\rangle \hat c_{j\sigma}^\dagger  \hat c_{i\sigma} + \mathrm{h.c.}
	\right] \\
	= \frac{2V}{L_x L_y} \sum_{\mathbf{K}, \mathbf{k}, \mathbf{k}'} \sum_\sigma \left[
		\cos \left(
			\delta k_x
			\right)	+ \cos \left(
			\delta k_y
		\right)	
	\right]	
	\langle
		\hat c_{\mathbf{K}+\mathbf{k} \sigma}^\dagger 
		\hat c_{\mathbf{K}-\mathbf{k}' \sigma}
	\rangle
	\hat c_{\mathbf{K}-\mathbf{k} \sigma}^\dagger  \hat c_{\mathbf{K}+\mathbf{k}'\sigma} \label{eq:reciprocal-space-non-local-interaction-fock-intermediate}
\end{multline}
where the $2$ prefactor comes from recognizing that the $\mathrm{h.c.}$ generates an identical contribution to the full sum.

In order to proceed, it is now necessary to understand how the AF phase is realized in reciprocal space. As is exposed in App.~\ref{appendix:mean-field-hubbard}, to impose an AF Ansatz of the form
\[
	\langle \hat n_{\mathbf{r}\sigma} \rangle = n - (-1)^{x+y+\delta_{\sigma=\uparrow}} m
	\quad\implies\quad
	\langle \hat n_{\mathbf{k}\sigma} \rangle = n\delta_{\mathbf{k}=\mathbf{0}} - (-1)^{\delta_{\sigma=\uparrow}} m\delta_{\mathbf{k}=\bm{\pi}}
\]
leads to an AF ground-state of free fermions at temperature $\beta$ described by the Nambu spinor of Eq.~\eqref{appeq:af-diagonalized-nambu-spinor}. All parameters are renormalized, thus we must account for renormalized band energies $\pm \tilde{E}_{\mathbf{k}\sigma}$ as well. The ground-state is realized by simply populating the two bands $\pm \tilde{E}_{\mathbf{k}\sigma}$ as
\[
	\bigotimes_{\mathbf{k}\in\mathrm{BZ}} \bigotimes_\sigma \left[
		\left(\hat \gamma_{\mathbf{k}\sigma}^{(-)}\right)^\dagger f(-\tilde{E}_\mathbf{k};\beta,\mu) + \left(\hat \gamma_{\mathbf{k}\sigma}^{(+)}\right)^\dagger f(\tilde{E}_\mathbf{k};\beta,\mu)
	\right] \ket{\Omega}
\]
The $\hat\gamma$ operators are normalized superpositions of two $\hat c$ operators at points in reciprocal space separated by a $\bm{\pi}$ shift. It follows that the above state is ultimately a superposition of many-body pure states, each of which has either the $\mathbf{k}\sigma$ state occupied \textit{or} the $\mathbf{k}+\bm{\pi}\sigma$ state. It follows that, when computing generically $\langle \hat c_{\mathbf{k}_1\sigma_1} \hat c_{\mathbf{k}_2\sigma_2} \rangle$, such expectation value can be non-zero if and only if $\sigma_1 = \sigma_2$ and $\mathbf{k}_1 = \mathbf{k}_2 + n\bm{\pi}$, being $n \in \mathbb{Z}$. Going back to Eq.~\eqref{eq:reciprocal-space-non-local-interaction-fock-intermediate}, this implies only two contributions are non-zero:
\[
	\mathbf{k} = -\mathbf{k}'
	\qq{or}
	\mathbf{k}+\bm{\pi} = -\mathbf{k}'
\]
Then Eq.~\eqref{eq:reciprocal-space-non-local-interaction-fock-intermediate} is reduced to:
\begin{multline*}
	\frac{2V}{L_x L_y} \sum_{\mathbf{K}, \mathbf{k}} \sum_\sigma \left[
		\cos \left(
			2k_x
		\right)	+ \cos \left(
			2k_y
		\right)	
	\right]	\\
	\left[
		\langle
			\hat c_{\mathbf{K}+\mathbf{k} \sigma}^\dagger 
			\hat 	c_{\mathbf{K}+\mathbf{k} \sigma}
		\rangle
		\hat c_{\mathbf{K}-\mathbf{k} \sigma}^\dagger  \hat c_{\mathbf{K}-\mathbf{k}\sigma}
		-
		\langle
			\hat c_{\mathbf{K}+\mathbf{k} \sigma}^\dagger 
			\hat 	c_{\mathbf{K}+\mathbf{k}+\bm{\pi} \sigma}
		\rangle
		\hat c_{\mathbf{K}-\mathbf{k} \sigma}^\dagger  \hat c_{\mathbf{K}-\mathbf{k}-\bm{\pi}\sigma}
	\right]
\end{multline*}
The sum over $\mathbf{K}$ produces $L_xL_y$ identical terms, and is thus absorbed leaving (after an irrelevant sign change)
\begin{equation}\label{eq:reciprocal-space-non-local-interaction-fock-intermediate-2}
	2V \sum_{\mathbf{k} \in \mathrm{BZ}} \sum_\sigma \left[
		\cos \left(
			2k_x
		\right)	+ \cos \left(
			2k_y
		\right)	
	\right]
	\left[
		\langle
			\hat c_{-\mathbf{k} \sigma}^\dagger 
			\hat c_{-\mathbf{k} \sigma}
		\rangle
		\hat c_{\mathbf{k} \sigma}^\dagger  \hat c_{\mathbf{k}\sigma}
		-
		\langle
			\hat c_{-\mathbf{k} \sigma}^\dagger 
			\hat 	c_{-(\mathbf{k}+\bm{\pi}) \sigma}
		\rangle
		\hat c_{\mathbf{k} \sigma}^\dagger  \hat c_{\mathbf{k}+\bm{\pi}\sigma}
	\right]
\end{equation}
Let me go through this last expression.

\paragraph{Tight-binding spectrum and chemical potential renormalization.}
The first term in square brackets of Eq.~\eqref{eq:reciprocal-space-non-local-interaction-fock-intermediate-2} contributes for $\mathbf{k}=\mathbf{0},\bm{\pi}$ coherently with the MF Ansatz. Since
\[
	\sum_\ell \cos \left(
		2\mathbf{k} \cdot \bm{\delta}_\ell
	\right)	\Big|_{\mathbf{k}=\mathbf{0},\bm{\pi}}= \frac{z}{2}
\]
the single particle energy $\epsilon_\mathbf{k}$ takes a $znV$ contribution at wavevector $\mathbf{0}$ from said term, and $(-1)^{\delta_{\sigma=\uparrow}} zmV$ at wavevector $\bm{\pi}$, obtaining:
\begin{equation}\label{eq:af-fock-tight-binding-ss-renormalization-raw}
	\epsilon_\mathbf{k} \to \epsilon_\mathbf{k} + zV \left[
		n\delta_{\mathbf{k}=\mathbf{0}} - (-1)^{\delta_{\sigma=\uparrow}} m \delta_{\mathbf{k}=\bm{\pi}}
	\right]
\end{equation}
Since the final form of the hamiltonian must be expressed restricting the sum to the MBZ, and $\epsilon_\mathbf{k} = - \epsilon_{\mathbf{k}+\bm{\pi}}$, one may write the hopping sector of the hamiltonian as
\[
	\sum_{\mathbf{k} \in \mathrm{MBZ}} \sum_\sigma \left[
		\left(
			\epsilon_\mathbf{k} + znV \delta_{\mathbf{k}=\mathbf{0}}
		\right) \hat c_{\mathbf{k}\sigma}^\dagger \hat c_{\mathbf{k}\sigma} + \left(
			-\epsilon_\mathbf{k} + (-1)^{\delta_{\sigma=\uparrow}} zmV \delta_{\mathbf{k}=\mathbf{0}}
		\right) \hat c_{\mathbf{k}+\bm{\pi}\sigma}^\dagger \hat c_{\mathbf{k}+\bm{\pi}\sigma}
	\right]
\]
\todo

\paragraph{Gap function renormalization.}
The second term in square brackets of Eq.~\eqref{eq:reciprocal-space-non-local-interaction-fock-intermediate-2} contributes instead to the gap renormalization, being out of diagonal in the $2\times2$ hamiltonian matrix. Restricting the sum to the MBZ and making use of reciprocal periodicity, it reduces to
\begin{multline*}
	- 2V \sum_{\mathbf{k} \in \mathrm{MBZ}} \sum_\sigma \left[
		\cos \left(
			2k_x
		\right)	+ \cos \left(
			2k_y
		\right)	
	\right] \\
	\left[
		\langle
			\hat c_{-\mathbf{k} \sigma}^\dagger 
			\hat c_{-(\mathbf{k}+\bm{\pi}) \sigma}
		\rangle
		\hat c_{\mathbf{k} \sigma}^\dagger  \hat c_{\mathbf{k}+\bm{\pi}\sigma} + \langle
			\hat c_{-(\mathbf{k}+\bm{\pi}) \sigma}^\dagger 
			\hat c_{-\mathbf{k} \sigma}
		\rangle
		\hat c_{\mathbf{k}+\bm{\pi} \sigma}^\dagger  \hat c_{\mathbf{k}\sigma}
	\right]
\end{multline*}
Use Eq.~\eqref{appeq:finite-temperature-order-parameter-derivation-intermediate},
\[
\begin{aligned}
	\langle
		\hat c_{-\mathbf{k} \sigma}^\dagger 
		\hat c_{-(\mathbf{k}+\bm{\pi}) \sigma}
	\rangle &= \langle 
		[
			\hat \Psi_{-\mathbf{k}\sigma}^\dagger
		]_1 [
			\hat \Psi_{-\mathbf{k}\sigma}
		]_2
	\rangle \\
	&= [
		W_{-\mathbf{k}\sigma}
	]_{1 1} [
		W_{-\mathbf{k}\sigma}^\dagger
	]_{2 1} f\left(
		-E_{-\mathbf{k}\sigma}; \beta,\mu
	\right) + [
		W_{-\mathbf{k}\sigma}
	]_{2 1} [
		W_{-\mathbf{k}\sigma}^\dagger
	]_{2 2} f\left(
		E_{-\mathbf{k}\sigma}; \beta,\mu
	\right) \\
	&= \frac{1}{2} \sin \left(2\theta_{-\mathbf{k}\sigma}\right) \tanh(\frac{\beta E_{-\mathbf{k}\sigma}}{2})
\end{aligned}
\]
%and then, making use of the Nambu spinorial formulation,
%\begin{multline*}
%	- 4V \sum_{\mathbf{k} \in \mathrm{MBZ}} \sum_\sigma \left[
%		\cos \left(
%			2k_x \delta_x
%		\right)	+ \cos \left(
%			2k_y \delta_y
%		\right)	
%	\right] \\
%	\left[
%		\langle
%			\hat \Psi_{-\mathbf{k} \sigma}^\dagger \tau^+
%			\hat \Psi_{-\mathbf{k}\sigma}
%		\rangle
%		\hat \Psi_{\mathbf{k} \sigma}^\dagger \tau^- \hat \Psi_{\mathbf{k}\sigma} + \langle
%			\hat \Psi_{-\mathbf{k} \sigma}^\dagger \tau^-
%			\hat \Psi_{-\mathbf{k}\sigma}
%		\rangle
%		\hat \Psi_{\mathbf{k} \sigma}^\dagger \tau^+ \hat \Psi_{\mathbf{k}\sigma}
%	\right]
%\end{multline*}
%\todo

\section{Superconducting instability}

This section is devoted to studying the superconducting phase of the system. The only symmetry we assume to break is the $\mathrm{U}(1)$ charge symmetry, thus allowing for superconducting fluctuations. The symmetry structure of the pairing mechanism determines the contributing Cooper fluctuations: for $s$-wave and $d$-wave superconductivity, only the o.s. Cooper term contributes; for $p_\ell$-wave superconductivity, the s.s. term contributes as well {\color{tabred}[Include hopping renormalization?]}. In the following sections, a derivation containing both Cooper terms is proposed.

{\color{tabred}
	[To be continued: separate singlet and triplet pairing channels, and describe triplet channel by itself by the means of four-components Nambu spinors. Use selection rules to set $\Delta^{(p_\ell)}=0$ in the singlet channel, in order to justify results obtained by a pure space-even simulation containing just the o.s. terms.]
}

\subsection{Mean-field treatment of the non-local term}

This approach leads to the conclusion that the (coherent) ground-state of the system must be an eigenstate of the mean-field effective hamiltonian:
\begin{equation}\label{eq:extended-hubbard-model-effective-intermediate}
	\begin{aligned}
		\hat H^{(\mathrm{e})} =
		&-t \sum_{\langle ij \rangle} \sum_\sigma \hat c_{i\sigma}^\dagger \hat c_{j\sigma}
		+ U \sum_{i \in \mathcal{S}} \hat n_{i\uparrow} \hat n_{i\downarrow} \\
		&- V \sum_{i \in \mathcal{S}} \sum_{\ell = x,y} \sum_{\delta = \pm \delta_\ell} \left[
			\langle 
				\hat c_{i\uparrow}^\dagger \hat c_{i + \delta \downarrow}^\dagger
			\rangle
			\hat c_{i + \delta \downarrow} \hat c_{i\uparrow} 
			+ \mathrm{h}.\mathrm{c}.
		\right]
	\end{aligned}
\end{equation}
Note that I am here summing over $i\in \mathcal{S}$: this is the same as considering both the $\uparrow\downarrow$ \textit{plus} the $\downarrow\uparrow$ terms of the o.s. hamiltonian involved in even-wave pairing. The pairing correlation function is defined across each bond as the pairing expectation
\[
	g_{ij} \equiv \langle 
		\hat c_{i\uparrow}^\dagger \hat c_{j\downarrow}^\dagger
	\rangle
\]
The effective hamiltonian reads:
\begin{equation}\label{eq:extended-hubbard-model-effective-intermediate-2}
	\hat H^{(\mathrm{e})} =
	-t \sum_{\langle ij \rangle} \sum_\sigma \hat c_{i\sigma}^\dagger \hat c_{j\sigma}
	+ U \sum_{i \in \mathcal{S}} \hat n_{i\uparrow} \hat n_{i\downarrow}
	- V \sum_{\langle ij \rangle} \left[
		g_{ij} \hat c_{j\downarrow} \hat c_{i\uparrow} + g_{ij}^* \hat c_{i\uparrow}^\dagger \hat c_{j\downarrow}^\dagger
	\right]
\end{equation}
As in standard $\mathrm{BCS}$ theory, this hamiltonian -- being quadratic in the electronic operators -- can be diagonalized via a Bogoliubov rotation. Superconducting pairing can arise both from the local $U$ term and from the non-local $V$ term. In next sections it is assumed the $V$ term generates dominant superconductivity via its weak non-local pairing.

\subsection{Mean-field treatment of the local term}

The mean-field description of the local (on-site) $U$ interaction is given in detail in App.~\ref{appendix:mean-field-hubbard}, along with a simple numerical analysis of the insurgence of antiferromagnetic ordering in a Hartree-Fock approximation scheme. Here the Cooper pairing is likewise assumed to dominate. Performing an analysis analogous to the one carried out in last section, we get the decoupling
\[
	U \sum_{i \in \mathcal{S}} \hat n_{i\uparrow} \hat n_{i\downarrow} \simeq U \sum_{i\sigma} \left[
		f_i \hat c_{i\downarrow} \hat c_{i\uparrow} + f_i^* \hat c_{i\uparrow}^\dagger \hat c_{i\downarrow}^\dagger
	\right]
\]
being
\[
	f_i \equiv \langle \hat c_{i\uparrow}^\dagger \hat c_{i\downarrow}^\dagger \rangle
\]
Collect $f$ and $g$ in the unique function of two variables:
\[
	C(i,j) = \begin{cases}
		f_i &\qq{if} i=j \\
		g_{ij} &\qq{if} \abs{i-j}=1 \\
		(\cdots) &\qq{otherwise}
	\end{cases}
\]
which expresses the generic correlator $\langle \hat c_{i\uparrow}^\dagger \hat c_{j\downarrow}^\dagger \rangle$. The correlator for $\abs{i-j}>1$ is left unexpressed, and supposed to be subdominant. The decoupled hamiltonian, apart from pure energy shifts and suppressed terms, is given by
\begin{align}
	\hat H^{(\mathrm{e})} =
	-t \sum_{\langle ij \rangle} \sum_\sigma \hat c_{i\sigma}^\dagger \hat c_{j\sigma}
	&+ U \sum_i \left[
	f_i \hat c_{i\downarrow} \hat c_{i\uparrow} + f_i^* \hat c_{i\uparrow}^\dagger \hat c_{i\downarrow}^\dagger
	\right] \nonumber \\
	&- V \sum_{\langle ij \rangle} \left[
	g_{ij} \hat c_{j\downarrow} \hat c_{i\uparrow} + g_{ij}^* \hat c_{i\uparrow}^\dagger \hat c_{j\downarrow}^\dagger
	\right] \label{eq:extended-hubbard-model-effective}
\end{align}
\todo

\subsection{Topological correlations}

\begin{figure}
	\centering
	\subfloat[Local $s$-wave]{
		\begin{tikzpicture}
	\fill[color=lightgray] 
	(0,0) circle (1.5pt)
		node[anchor=south west, color=black]
			{$1$}
	(1,0) circle (1.5pt)
		node[anchor=west]
			{$0$}
	(-1,0) circle (1.5pt)
		node[anchor=east]
			{$0$}
	(0,1) circle (1.5pt)
		node[anchor=south]
			{$0$}
	(0,-1) circle (1.5pt)
		node[anchor=north]
			{$0$};
			
	\node[anchor=center] 
		at (-1,1)
			{$\varphi^{(s)}_{ij}$};
	
	\draw[color=lightgray, dashed]
		(-1,0) -- (1,0)
		(0,-1) -- (0,1);
\end{tikzpicture}
		\label{subfig:s-wave-correlator}
	}
	\hspace{2em}
	\subfloat[Extended $s$-wave.]{
		\begin{tikzpicture}
	\fill[color=lightgray] 
	(0,0) circle (1.5pt)
	(1,0) circle (1.5pt)
		node[anchor=west, color=black]
			{$g^{(s)}$}
	(-1,0) circle (1.5pt)
		node[anchor=east, color=black]
			{$g^{(s)}$}
	(0,1) circle (1.5pt)
		node[anchor=south, color=black]
			{$g^{(s)}$}
	(0,-1) circle (1.5pt)
		node[anchor=north, color=black]
			{$g^{(s)}$};
	
	\draw[color=lightgray]
	(-1,0) -- (1,0)
	(0,-1) -- (0,1);
\end{tikzpicture}
		\label{subfig:s*-wave-correlator}
	}\\[3em]
	\subfloat[$p_x$-wave.]{
		\begin{tikzpicture}
	\fill[color=lightgray] 
	(0,0) circle (1.5pt)
		node[anchor=south west]
			{$0$}
	(1,0) circle (1.5pt)
		node[anchor=west, color=black]
			{$1$}
	(-1,0) circle (1.5pt)
		node[anchor=east, color=black]
			{$-1$}
	(0,1) circle (1.5pt)
		node[anchor=south]
			{$0$}
	(0,-1) circle (1.5pt)
		node[anchor=north]
			{$0$}
	(0,-1) circle (1.5pt)
		node[anchor=north, opacity=0]
			{$-1$}; % Aligment
			
	\node[anchor=center] 
		at (-1,1)
			{$\varphi^{(p_x)}_{ij}$};
	
	\draw[color=lightgray] 
		(-1,0) -- (1,0);
	\draw[color=lightgray, dashed] 
		(0,-1) -- (0,1);
\end{tikzpicture}
		\label{subfig:px-wave-correlator}
	}
	\hspace{2em}
	\subfloat[$p_y$-wave.]{
		\begin{tikzpicture}
	\fill[color=lightgray] 
	(0,0) circle (1.5pt)
		node[anchor=south west]
			{$0$}
	(1,0) circle (1.5pt)
		node[anchor=west]
			{$0$}
	(-1,0) circle (1.5pt)
		node[anchor=east]
			{$0$}
	(0,1) circle (1.5pt)
		node[anchor=south, color=black]
			{$1$}
	(0,-1) circle (1.5pt)
		node[anchor=north, color=black]
			{$-1$};
			
	\node[anchor=center] 
		at (-1,1)
			{$\varphi^{(p_y)}_{ij}$};
	
	\draw[color=lightgray, dashed] 
		(-1,0) -- (1,0);
	\draw[color=lightgray] 
		(0,-1) -- (0,1);
\end{tikzpicture}
		\label{subfig:py-wave-correlator}
	}
	\hspace{2em}
	\subfloat[$d_{x^2-y^2}$-wave.]{
		\begin{tikzpicture}
	\fill[color=lightgray] 
	(0,0) circle (1.5pt)
	(1,0) circle (1.5pt)
		node[anchor=west, color=black]
			{$g^{(d_{x^2-y^2})}$}
	(-1,0) circle (1.5pt)
		node[anchor=east, color=black]
			{$g^{(d_{x^2-y^2})}$}
	(0,1) circle (1.5pt)
		node[anchor=south, color=black]
			{$-g^{(d_{x^2-y^2})}$}
	(0,-1) circle (1.5pt)
		node[anchor=north, color=black]
			{$-g^{(d_{x^2-y^2})}$};
	
	\draw[color=lightgray] 
		(-1,0) -- (1,0)
		(0,-1) -- (0,1);
\end{tikzpicture}
		\label{subfig:d-wave-correlator}
	}
	\caption{Form factors at different topologies, as listed in Tab.~\ref{tab:x-wave-real-factors}. In figures five sites are represented: the hub site $i$ and its four NN. Solid lines represent non-zero values for $\varphi_{\bm{\delta}}$, while dashed lines represent vanishing factors.}
	\label{fig:x-wave-real-factors}
\end{figure}

Topology plays an important role in establishing SC, giving rise to anisotropic pairing as well as real space structures for the Cooper pairs. The correlator $g_{ij}$ is a function of position, specifically of its variables difference $\bm{\delta} \equiv \mathbf{x}_j-\mathbf{x}_i$. Over the square lattice with NN interaction, the latter can assume four values: $\bm{\delta} = \pm \bm{\delta}_x$, $\pm \bm{\delta}_y$. For a function of space defined over the four rim sites $\mathbf{x}_i \pm \bm{\delta}_\ell$ of Fig.~\ref{fig:square-nearest-neighbors}, various symmetry structures can be defined under the planar rotations group $\mathrm{SO}(2)$. In other words, the function $g_{\bm{\delta}}$  can be decomposed in planar harmonics (which are simply the sine-cosine basis). Equivalently, given two NN sites $i$, $j$
\[
	g_{ij} = \sum_\gamma g^{(\gamma)} \varphi_{ij}^{(\gamma)}
\]
where $g^{(\gamma)}$ are the $g_{ij}$ symmetries-decomposition coefficients while $\varphi_{ij}^{(\gamma)}$ are the form factors listed in Tab.~\ref{tab:x-wave-real-factors}, a simple orthonormal rearrangement of the harmonics basis.

\setlength{\extrarowheight}{0.5em}
\begin{table}
	\centering
	\begin{tabular}{r E l l}
		\textbf{Structure} & \multicolumn{2}{c}{\textbf{Form factor}} & \textbf{Graph} \\
		\midrule
		$s$-wave & $\varphi^{(s)}_{ij}$ & $\delta_{ij}$ & Fig.~\ref{subfig:s-wave-correlator} \\
		Extended $s$-wave & $\varphi^{(s^*)}_{ij}$ &
		$\delta_{j=i+\delta_x} + \delta_{j=i-\delta_x} + \delta_{j=i+\delta_y} + \delta_{j=i-\delta_y}$ & Fig.~\ref{subfig:s*-wave-correlator} \\
		$p_x$-wave & $\varphi^{(p_x)}_{ij}$ & $ 
		\delta_{j=i+\delta_x} - \delta_{j=i-\delta_x}$ & Fig.~\ref{subfig:px-wave-correlator} \\
		$p_y$-wave & $\varphi^{(p_y)}_{ij}$ & $ \delta_{j=i+\delta_y} - \delta_{j=i-\delta_y}$ & Fig.~\ref{subfig:py-wave-correlator} \\
		$d_{x^2-y^2}$-wave & $\varphi^{(d)}_{ij}$ & $
		\delta_{j=i+\delta_x} + \delta_{j=i-\delta_x} - \delta_{j=i+\delta_y} - \delta_{j=i-\delta_y}$ & Fig.~\ref{subfig:d-wave-correlator} 
	\end{tabular}
	\caption{First four spatial structures for the correlation function $C(i,j)$. In the middle column, all spatial dependence is included in the $\delta$s, while $f^{s}, g^{(\gamma)} \in \mathbb{C}$. The last column indicates the graph representation of each contribution given in Fig.~\ref{fig:wave-correlators}. Subscript $x^2-y^2$ is omitted for notational clarity.}
	\label{tab:x-wave-real-factors}
\end{table}
\setlength{\extrarowheight}{0em}

SC is established with a given symmetry -- which means, symmetry breaking in the phase transition proceeds in a specific channel. Conventional $\mathrm{BCS}$ superconductivity arises from the only possible spatial structure of the local pairing, $s$-wave -- here appearing as a local term (Fig.~\ref{subfig:s-wave-correlator}) and extended on a non-local term (Fig.~\ref{subfig:s*-wave-correlator}). Cuprates exhibit a tendency towards $d_{x^2-y^2}$ SC, while other materials towards $p$-wave types -- eventually with some chirality, as is the case for $p_x \pm i p_y$ SCs. To establish SC under a certain symmetry $\gamma$ means that Cooper pairs acquire said symmetry -- which implies, for correlations, $g^{(\gamma')} = g^{(\gamma)} \delta_{\gamma\gamma'}$. and $g_{ij} \propto \varphi_{ij}^{(\gamma)}$.

\section{Mean-Field theory reciprocal space description}\label{sec:mft-analysis-reciprocal-space}

In this $\mathrm{BCS}$-like approach, a self-consistent equation for the gap function must be retrieved in order to further investigate the model and extract the conditions for the formation of a superconducting phase with a given pairing topology. In order to do so, let me take a step back and perform explicitly the Fourier-transform of the various terms of Eq.~\ref{eq:extended-hubbard-model}.

\subsection{Kinetic term}

The kinetic part is trivial to transform. The followed convention is
\[
	\hat c_{j\sigma} = \frac{1}{\sqrt{L_xL_y}} \sum_{\mathbf{k} \in \mathrm{BZ}} e^{-i \mathbf{k} \cdot \mathbf{x}_j} \hat c_{\mathbf{k}\sigma}
\]
Calculation is carried out in App.~\ref{app:mean-field-hubbard}. Let
\[
	\epsilon_\mathbf{k} \equiv -2t \left[
	\cos(k_x \delta_x) + \cos(k_y \delta_y)
	\right]
\]
then we have
\[
\begin{aligned}
	-t \sum_{\langle ij \rangle} \sum_\sigma \hat c_{i\sigma}^\dagger \hat c_{j\sigma} &= \sum_{\mathbf{k}\sigma} \epsilon_\mathbf{k} \hat c_{\mathbf{k}\sigma}^\dagger \hat c_{\mathbf{k}\sigma} \\
	&= \sum_\mathbf{k} \epsilon_\mathbf{k} \left[
	\hat c_{\mathbf{k}\uparrow}^\dagger \hat c_{\mathbf{k}\uparrow} + \hat c_{\mathbf{k}\downarrow}^\dagger \hat c_{\mathbf{k}\downarrow}
	\right] \\
	&= \sum_\mathbf{k} \epsilon_\mathbf{k} \left[
	\hat c_{\mathbf{k}\uparrow}^\dagger \hat c_{\mathbf{k}\uparrow}- \hat c_{-\mathbf{k}\downarrow} \hat c_{-\mathbf{k}\downarrow}^\dagger 
	\right]
\end{aligned}
\]
In last passage I used fermionic anti-commutation rules and reversed the sign of the mute variable. This will become useful later.

\subsection{Non-local attraction}

{\color{tabred}

Consider a generic bond, say, the one connecting sites $j$ and $j\pm\delta_\ell$ (variable $i$ is here referred to as the imaginary unit to avoid confusion). $\mathbf{x}_j$ is the $2$D notation for the position of site $j$, while $\bm{\delta}_\ell$ is the $2$D notation for the lattice spacing previously indicated as $\delta_\ell$. Fourier transform it according to the convention
\[
	\hat c_{j\sigma} = \frac{1}{\sqrt{L_xL_y}} \sum_{\mathbf{k} \in \mathrm{BZ}} e^{-i \mathbf{k} \cdot \mathbf{x}_j} \hat c_{\mathbf{k}\sigma}
\]
Then:
\[
\begin{aligned}
	\hat n_{j\uparrow} \hat n_{j \pm \delta_\ell \downarrow} &= \hat c_{j\uparrow}^\dagger \hat c_{j \pm \delta_\ell \downarrow}^\dagger \hat c_{j \pm \delta_\ell \downarrow} \hat c_{j\uparrow} \\
	&= \frac{1}{(L_xL_y)^2} \sum_{\nu=1}^4 \sum_{\mathbf{k}_\nu \in \mathrm{BZ}} e^{i \left[ (\mathbf{k}_1 + \mathbf{k}_2) - (\mathbf{k}_3 + \mathbf{k}_4) \right] \cdot \mathbf{x}_j} e^{\pm i(\mathbf{k}_2-\mathbf{k}_3) \cdot \bm{\delta}_\ell}  \hat c_{\mathbf{k}_1 \uparrow}^\dagger \hat c_{\mathbf{k}_2 \downarrow}^\dagger \hat c_{\mathbf{k}_3 \downarrow} \hat c_{\mathbf{k}_4\uparrow}
\end{aligned}
\]
It follows,
\[
\begin{aligned}
	\hat h_V^{(j)} &= -\frac{V}{(L_xL_y)^2} \sum_{\ell = x,y} \sum_{\nu=1}^4 \sum_{\mathbf{k}_\nu \in \mathrm{BZ}} e^{i \left[ (\mathbf{k}_1 + \mathbf{k}_2) - (\mathbf{k}_3 + \mathbf{k}_4) \right] \cdot \mathbf{x}_j} \\
	&\hspace{0.3\textwidth} \times \left(
	e^{ i(\mathbf{k}_2-\mathbf{k}_3) \cdot \bm{\delta}_\ell} + e^{ -i(\mathbf{k}_2-\mathbf{k}_3) \cdot \bm{\delta}_\ell} 
	\right)
	\hat c_{\mathbf{k}_1 \uparrow}^\dagger \hat c_{\mathbf{k}_2 \downarrow}^\dagger \hat c_{\mathbf{k}_3 \downarrow} \hat c_{\mathbf{k}_4\uparrow} \\
	&= -\frac{2V}{(L_xL_y)^2} \sum_{\ell = x,y} \sum_{\nu=1}^4 \sum_{\mathbf{k}_\nu \in \mathrm{BZ}} e^{i \left[ (\mathbf{k}_1 + \mathbf{k}_2) - (\mathbf{k}_3 + \mathbf{k}_4) \right] \cdot \mathbf{x}_j} \cos\left[
	(\mathbf{k}_2-\mathbf{k}_3) \cdot \bm{\delta}_\ell
	\right]	\hat c_{\mathbf{k}_1 \uparrow}^\dagger \hat c_{\mathbf{k}_2 \downarrow}^\dagger \hat c_{\mathbf{k}_3 \downarrow} \hat c_{\mathbf{k}_4\uparrow}
\end{aligned}
\]
The full non-local interaction is given by summing over all sites of $\mathcal{S}$. This gives back momentum conservation,
\[
	\frac{1}{L_xL_y} \sum_{j \in \mathcal{S}} e^{i \left[ (\mathbf{k}_1 + \mathbf{k}_2) - (\mathbf{k}_3 + \mathbf{k}_4) \right] \cdot \mathbf{x}_j} = \delta_{\mathbf{k}_1 + \mathbf{k}_2 = \mathbf{k}_3 + \mathbf{k}_4}
\]
Let $\mathbf{k}_1 + \mathbf{k}_2 = \mathbf{k}_3 + \mathbf{k}_4 = \mathbf{K}$, and define $\mathbf{k}$, $\mathbf{k}'$ such that
\[
	\mathbf{k}_1 \equiv \mathbf{K} + \mathbf{k} 
	\qquad
	\mathbf{k}_2 \equiv \mathbf{K} - \mathbf{k} 
	\qquad
	\mathbf{k}_3 \equiv \mathbf{K} - \mathbf{k}' 
	\qquad
	\mathbf{k}_4 \equiv \mathbf{K} + \mathbf{k}'
	\qquad
	\delta \mathbf{k} \equiv \mathbf{k}-\mathbf{k}'
\]
Sums over these variables must be intended as over the Brillouin Zone ($\mathrm{BZ}$). Then, finally
\[
\begin{aligned}
	\hat H_V &\simeq \sum_{j \in \mathcal{S}} \hat h_V^{(j)} \\
	&= - \frac{2V}{L_x L_y} \sum_{\ell = x,y} \sum_{\mathbf{K}, \mathbf{k}, \mathbf{k}'} \cos\left(
	\delta \mathbf{k} \cdot \bm{\delta}_\ell
	\right)	\hat c_{\mathbf{K}+\mathbf{k} \uparrow}^\dagger \hat c_{\mathbf{K}-\mathbf{k} \downarrow}^\dagger \hat c_{\mathbf{K}-\mathbf{k}' \downarrow} \hat c_{\mathbf{K}+\mathbf{k}'\uparrow} \\
	&= - \frac{2V}{L_x L_y} \sum_{\ell = x,y} \sum_{\mathbf{K}, \mathbf{k}, \mathbf{k}'} \left[
		\cos \left(
			\delta k_x \delta_x
		\right)	+ \cos \left(
			\delta k_y \delta_y
		\right)	
	\right]	\hat c_{\mathbf{K}+\mathbf{k} \uparrow}^\dagger \hat c_{\mathbf{K}-\mathbf{k} \downarrow}^\dagger \hat c_{\mathbf{K}-\mathbf{k}' \downarrow} \hat c_{\mathbf{K}+\mathbf{k}'\uparrow} \\
\end{aligned}
\]
Eventually, in the second passage the prefactor $2$ can be absorbed by reintroducing the spin DoF\footnote{
	Justification can be given in two ways: either commutating appropriately the $\hat c$ operators, or by carrying out the previous space sums independently over the two sublattices.
}. 

}

Taking in the mean-field approximation (with Cooper pair symmetry breaking), we get
\[
	\hat c_{\mathbf{K}+\mathbf{k} \uparrow}^\dagger \hat c_{\mathbf{K}-\mathbf{k} \downarrow}^\dagger \hat c_{\mathbf{K}-\mathbf{k}' \downarrow} \hat c_{\mathbf{K}+\mathbf{k}'\uparrow} \simeq \langle
		\hat c_{\mathbf{K}+\mathbf{k} \uparrow}^\dagger \hat c_{\mathbf{K}-\mathbf{k} \downarrow}^\dagger
	\rangle \hat c_{\mathbf{K}-\mathbf{k}' \downarrow} \hat c_{\mathbf{K}+\mathbf{k}'\uparrow} + \hat c_{\mathbf{K}+\mathbf{k} \uparrow}^\dagger \hat c_{\mathbf{K}-\mathbf{k} \downarrow}^\dagger \langle 
		\hat c_{\mathbf{K}-\mathbf{k}' \downarrow} \hat c_{\mathbf{K}+\mathbf{k}'\uparrow}
	\rangle + \cdots
\]
Take e.g. $\langle \hat c_{\mathbf{K}+\mathbf{k} \uparrow}^\dagger \hat c_{\mathbf{K}-\mathbf{k} \downarrow}^\dagger \rangle$: the only non-zero contribution can come from the $\mathbf{K}=\mathbf{0}$ term, as will be discussed self-consistently in Sec.~\ref{subsec:mft-self-consitency-cooper-decomposition}. Then finally:
\[
\begin{aligned}
	\hat H_V \simeq - \sum_{\mathbf{k}, \mathbf{k}'}
	V_{\mathbf{k}\mathbf{k}'} \left[
		\langle 
			\hat \phi_\mathbf{k}^\dagger
		\rangle \hat \phi_{\mathbf{k}'} + \langle 
			\hat \phi_\mathbf{k}
		\rangle \hat \phi_{\mathbf{k}'}^\dagger
	\right]	
\end{aligned}
\]
having I defined the pairing operator 
\[
	\hat \phi_\mathbf{k} \equiv \hat c_{-\mathbf{k}\downarrow} \hat c_{\mathbf{k} \uparrow}
	\qquad
	\hat \phi_\mathbf{k}^\dagger \equiv \hat c_{\mathbf{k} \uparrow}^\dagger \hat c_{-\mathbf{k}\downarrow}^\dagger
\]
and the two-body potential
\[
	V_{\mathbf{k}\mathbf{k}'} = \frac{2V}{L_x L_y} \left[
		\cos \left(
			\delta k_x \delta_x
			\right)	+ \cos \left(
			\delta k_y \delta_y
		\right)	
	\right]
\]
Now, consider the term
\[
	\cos \left( \delta k_x \right)	+ \cos \left( \delta k_y \right) = \cos k_x \cos k_x' + \sin k_x \sin k_x' + \cos k_y \cos k_y' + \sin k_y \sin k_y'
\]
For the sake of readability, the notations
\[
	c_\ell \equiv \cos k_\ell
	\qquad
	s_\ell \equiv \sin k_\ell
	\qquad
	c_\ell' \equiv \cos	k_\ell'
	\qquad
	s_\ell' \equiv \sin k_\ell'
\]
are used. Group the four terms above,
\begin{equation}\label{eq:sym-asym-couplings-gap}
	\underbrace{
		\left(c_x c_x' + c_y c_y' \right) 
	}_\text{Symmetric}
	+ \underbrace{
		\left(s_x s_x' + s_y s_y' \right)
	}_\text{Anti-symmetric}
\end{equation}
The first two exhibit inversion symmetry for both arguments $\mathbf{k}$, $\mathbf{k}'$; the second two exhibit anti-symmetry. Decoupling the symmetric part,
\[
	c_x c_x' + c_y c_y' = \frac{1}{2} (c_x + c_y)(c_x' + c_y') + \frac{1}{2} (c_x - c_y)(c_x' - c_y')
\]
which finally gives:
\[
\begin{aligned}
	\cos \left( \delta k_x \right)	+ \cos \left( \delta k_y  \right) &= \frac{1}{2}  (c_x + c_y)(c_x' + c_y') &&\qquad\text{($s^*$-wave)} \\
	&\hphantom{=}+ s_x s_x' &&\qquad\text{($p_x$-wave)} \\
	&\hphantom{=}+ s_y s_y' &&\qquad\text{($p_y$-wave)} \\
	&\hphantom{=}+ \frac{1}{2} (c_x - c_y)(c_x' - c_y') &&\qquad\text{($d_{x^2-y^2}$-wave)}
\end{aligned}
\]
In other words, the two-body potential decomposes as
\[
\begin{aligned}
	V_{\mathbf{k}\mathbf{k}'} &= \sum_\gamma V^{(\gamma)} \varphi_\mathbf{k}^{(\gamma)} \varphi_{\mathbf{k}'}^{(\gamma)*}
	\hspace{5em}\qq{where}
	\gamma = s^*, p_x, p_y, d_{x^2-y^2} \\
	&= \frac{V}{L_xL_y} \sum_\gamma \varphi_\mathbf{k}^{(\gamma)} \varphi_{\mathbf{k}'}^{(\gamma)*}
\end{aligned}
\]
being $\varphi_\mathbf{k}^{(\gamma)}$ the reciprocal-space expressions for the form factors of Tab.~\ref{tab:x-wave-real-factors}, listed explicitly in Tab.~\ref{tab:x-wave-reciprocal-factors}, and $V_{\mathbf{k}\mathbf{k}'}^{(\gamma)}$ the symmetry-resolved components of the non-local attraction. Then the two-body potential has been decomposed in its planar symmetry components, each of which will naturally couple only to identically structured parameters in the full hamiltonian.

\setlength{\extrarowheight}{0.5em}
\begin{table}
	\centering
	\begin{tabular}{r E l l}
		\textbf{Structure} & \multicolumn{2}{c}{\textbf{Structure factor}} & \textbf{Graph} \\
		\midrule
		$s$-wave & $\varphi_\mathbf{k}^{(s)}$ & $1$ & Fig.~\ref{subfig:s-wave-correlator} \\
		Extended $s$-wave & $\varphi^{(s^*)}_{\mathbf{k}}$ & $\cos k_x + \cos k_y$ & Fig.~\ref{subfig:s*-wave-correlator} \\
		$p_x$-wave & $\varphi_\mathbf{k}^{(p_x)}$ & $i \sqrt{2} \sin k_x $ & Fig.~\ref{subfig:px-wave-correlator} \\
		$p_y$-wave & $\varphi_\mathbf{k}^{(p_y)}$ & $i \sqrt{2} \sin k_y$ & Fig.~\ref{subfig:py-wave-correlator} \\
		$d_{x^2-y^2}$-wave & $\varphi_\mathbf{k}^{(d)}$ & $\cos k_x - \cos k_y$ & Fig.~\ref{subfig:d-wave-correlator} 
	\end{tabular}
	\caption{Structure factors derived from the correlation structures of Tab.~\ref{tab:wave-correlators}. The functions hereby defined are orthonormal, and define the various components of the non-local topological effective potential.}
	\label{tab:x-wave-reciprocal-factors}
\end{table}
\setlength{\extrarowheight}{0em}

Define now the non-local gap function
\begin{equation}\label{eq:self-consistency-equation-intermediate}
	\mathcal{V}_\mathbf{k} \equiv \sum_{\mathbf{k}'}
	V_{\mathbf{k}\mathbf{k}'}
	\langle
	\hat \phi_{\mathbf{k}'}^\dagger
	\rangle
\end{equation}
one gets immediately
\begin{equation}\label{eq:extended-hubbard-nonlocal-interaction-mean-field-reciprocal}
	\hat H_V \simeq -\sum_\mathbf{k} \left[
	\mathcal{V}_\mathbf{k} \hat \phi_\mathbf{k} + \mathcal{V}_\mathbf{k}^* \hat \phi_\mathbf{k}^\dagger
	\right]	
\end{equation}
To assume symmetry is broken in a specific symmetry channel $\gamma$ means precisely to assume $g_{ij} \propto \varphi_{ij}^{(\gamma)}$, which in turn implies $\langle \hat \phi_\mathbf{k} \rangle \propto \varphi_\mathbf{k}^{(\gamma)}$. Of course, in Eq.~\eqref{eq:self-consistency-equation-intermediate} only the $\gamma$ component of the potential survives, implying the gap function acquires the same symmetry,
\[
\begin{aligned}
	\mathcal{V}_\mathbf{k} &\propto \sum_{\mathbf{k}'}
	\frac{V}{L_xL_y} \varphi_\mathbf{k}^{(\gamma)} \varphi_{\mathbf{k}'}^{(\gamma)*}
	\varphi_{\mathbf{k}'}^{(\gamma)} \\
	&\propto \varphi_\mathbf{k}^{(\gamma)}
\end{aligned}
\]
where I used orthonormality of the $\varphi_\mathbf{k}^{(\gamma)}$ functions.

\subsection{Local interaction and gap function}

A very similar argument can be carried out for the local $U$ term. Without delving in too many details, the local gap $\mathcal{U}_\mathbf{k}$ is given by
\begin{equation}\label{eq:sgap-function-definition}
	\mathcal{U}_\mathbf{k} \equiv \frac{U}{2 L_x L_y} \sum_{\mathbf{k}} \langle 
	\hat \phi_\mathbf{k}
	\rangle
\end{equation}
evidently independent of $\mathbf{k}$, correctly. Identical considerations as in the above section hold for the local gap. The local part of the hamiltonian then gets
\begin{equation}\label{eq:extended-hubbard-local-interaction-mean-field-reciprocal}
	\hat H_U \simeq \sum_\mathbf{k} \left[
	\mathcal{U}_\mathbf{k} \hat \phi_\mathbf{k} + \mathcal{U}_\mathbf{k}^* \hat \phi_\mathbf{k}^\dagger
	\right]	
\end{equation}
and the full gap function is simply
\begin{equation}\label{eq:gap-function-definition}
	\Delta_\mathbf{k} \equiv \mathcal{V}_\mathbf{k} - \mathcal{U}_\mathbf{k}
\end{equation}
Notice here that the only possible topology here is $s$-wave; define trivially the $s$-wave component of the total two-body interaction,
\[
V^{(s)} = - \frac{U}{2L_xL_y}
\]
Then the full effective interaction is collected in
\[
\begin{aligned}
	\hat H_U +\hat H_V &\simeq - \sum_\gamma \sum_{\mathbf{k}, \mathbf{k}'}
	V^{(\gamma)} \varphi_\mathbf{k}^{(\gamma)} \varphi_{\mathbf{k}'}^{(\gamma)*} \left[
		\langle 
			\hat \phi_\mathbf{k}^\dagger
		\rangle \hat \phi_{\mathbf{k}'} + 	\langle 
			\hat \phi_\mathbf{k}
		\rangle \hat \phi_{\mathbf{k}'}^\dagger
	\right]	\\
	&= - \sum_\mathbf{k} \left[
	\Delta_\mathbf{k} \hat \phi_\mathbf{k} + \Delta_\mathbf{k}^* \hat \phi_\mathbf{k}^\dagger
	\right]	
\end{aligned}
\]

The full self-consistency equation is given by
\begin{equation}\label{eq:self-consistency-equation}
	\Delta_\mathbf{k} \equiv \sum_{\mathbf{k}'}
	\left[
		V^{(s)} +
		V_{\mathbf{k}\mathbf{k}'}
	\right]
	\langle
		\hat \phi_{\mathbf{k}'}^\dagger
	\rangle
\end{equation}
The gap function decomposes in symmetry channels as well,
\[
\Delta_\mathbf{k} = \sum_\gamma \Delta^{(\gamma)} \varphi_\mathbf{k}^{(\gamma)}
\]
If SC arises in a specific symmetry channel, $\Delta_\mathbf{k}$ will show the same symmetry. It follows, due to orthonormality and using Eq.~\eqref{eq:self-consistency-equation},
\begin{align}
	\Delta^{(\gamma)} &= \frac{1}{L_xL_y} \sum_{\mathbf{k}} \varphi_\mathbf{k}^{(\gamma)*} \Delta_\mathbf{k} \nonumber \\
	&= \frac{1}{L_xL_y} \sum_{\mathbf{k}} \varphi_\mathbf{k}^{(\gamma)*} \sum_{\mathbf{k}'}
	\left[
		V^{(s)} +
	V_{\mathbf{k}\mathbf{k}'}
	\right]
	\langle
		\hat \phi_{\mathbf{k}'}^\dagger
	\rangle \nonumber \\
	&= \frac{1}{L_xL_y} \sum_{\mathbf{k}} \varphi_\mathbf{k}^{(\gamma)*} \sum_{\mathbf{k}'\gamma'}
	V^{(\gamma')} \varphi_\mathbf{k}^{(\gamma')} \varphi_{\mathbf{k}'}^{(\gamma')*}
	\langle
		\hat \phi_{\mathbf{k}'}^\dagger
	\rangle \nonumber \\
	&= V^{(\gamma)} \sum_{\mathbf{k}} \varphi_\mathbf{k}^{(\gamma)*} \langle
		\hat \phi_{\mathbf{k}}^\dagger
	\rangle \label{eq:self-consistency-equation-explicit}
\end{align}
This result provides a set of self-consistency equations for each symmetry channel, listed in Tab.~\ref{tab:x-wave-self-consistency-equation}. Notice that to reconstruct self-consistently the full $s$-wave phase transition, the actual gap function is given by
\[
	\Delta^{(s)} + \Delta^{(s^*)} (c_x + c_y)
\]
The $s$-wave transition is the only one equipped of both the local and the non-local parts. Within this structure, we are finally able to move to Nambu formalism.

\setlength{\extrarowheight}{1em}
\begin{table}
	\centering
	\begin{tabular}{r E l l}
		\textbf{Structure} & \multicolumn{2}{c}{\textbf{Self-consistency equation}} & \textbf{Graph} \\
		\midrule
		$s$-wave & $\Delta^{(s)}$ & $\displaystyle -\frac{U}{2L_xL_y} \sum_{\mathbf{k}} \langle
		\hat \phi_{\mathbf{k}}^\dagger
		\rangle $ & Fig.~\ref{subfig:s-wave-correlator} \\
		Extended $s$-wave & $\Delta^{(s^*)}$ & $\displaystyle \frac{V}{L_xL_y} \sum_{\mathbf{k}} (c_x + c_y) \langle
		\hat \phi_{\mathbf{k}}^\dagger
		\rangle$ & Fig.~\ref{subfig:s*-wave-correlator} \\
		$p_x$-wave & $\Delta^{(p_x)}$ & $\displaystyle - i\sqrt{2} \frac{V}{L_xL_y} \sum_{\mathbf{k}} s_x \langle
		\hat \phi_{\mathbf{k}}^\dagger
		\rangle$ & Fig.~\ref{subfig:px-wave-correlator} \\
		$p_y$-wave & $\Delta^{(p_y)}$ & $\displaystyle -i \sqrt{2} \frac{V}{L_xL_y} \sum_{\mathbf{k}} s_y \langle
		\hat \phi_{\mathbf{k}}^\dagger
		\rangle$ & Fig.~\ref{subfig:py-wave-correlator} \\
		$d_{x^2-y^2}$-wave & $\Delta^{(d)}$ & $\displaystyle \frac{V}{L_xL_y} \sum_{\mathbf{k}} (c_x - c_y) \langle
		\hat \phi_{\mathbf{k}}^\dagger
		\rangle$ & Fig.~\ref{subfig:d-wave-correlator} 
	\end{tabular}
	\caption{Symmetry resolved self-consistency equations for the MFT parameters $\Delta^{(\gamma)}$, based on Eq.~\eqref{eq:self-consistency-equation} and \eqref{eq:self-consistency-equation-explicit}. By computing $\langle \hat \phi_{\mathbf{k}}^\dagger \rangle$, it is possible to reconstruct the various components of the gap function.}
	\label{tab:x-wave-self-consistency-equation}
\end{table}
\setlength{\extrarowheight}{0em}

\subsection{Nambu formalism and Bogoliubov transform}\label{subsec:nambu-formalism-mean-field-extended-hubbard}

Define the Nambu spinor\footnote{
	Notice that the spinor is here differently defined with respect to App.~\ref{appendix:mean-field-hubbard}, where because of the HF prevalence in mean-field decoupling the spinor components were homogeneously fermions creations or destructions.
} as in BCS
\[
\hat \Psi_\mathbf{k} \equiv \begin{bmatrix}
	\hat c_{\mathbf{k}\uparrow} \\
	\hat c_{-\mathbf{k}\downarrow}^\dagger
\end{bmatrix}
\]
Evidently,
\begin{equation}\label{eq:extended-hubbard-phi-psi-expressions}
	\phi_\mathbf{k} = \hat \Psi_\mathbf{k}^\dagger \begin{bmatrix}
		0 & 1 \\ 0 & 0
	\end{bmatrix} \hat \Psi_\mathbf{k}
	\qquad
	\phi_\mathbf{k}^\dagger = \hat \Psi_\mathbf{k}^\dagger \begin{bmatrix}
		0 & 0 \\ 1 & 0
	\end{bmatrix} \hat \Psi_\mathbf{k}
\end{equation}
The full hamiltonian is then given by:
\begin{equation}\label{eq:extended-hubbard-hamiltonian-nambu-bogoliubov}
	\hat H = \sum_\mathbf{k} \hat \Psi_\mathbf{k} h_\mathbf{k} \hat \Psi_\mathbf{k}
	\qquad
	h_\mathbf{k} \equiv \begin{bmatrix}
		\epsilon_\mathbf{k} & - \Delta_\mathbf{k}^* \\
		- \Delta_\mathbf{k} & - \epsilon_\mathbf{k}
	\end{bmatrix}
\end{equation}
Let $\tau^\alpha$ for $\alpha = x,y,z$ be the Pauli matrices. Define:
\[
\hat s_\mathbf{k}^\alpha \equiv \hat \Psi_\mathbf{k}^\dagger \tau^\alpha \hat \Psi_\mathbf{k}
\qq{for}
\alpha = x,y,z
\]
As can be shown easily, these operators realize spin-$1/2$ algebra. $\hat H$ represents an ensemble of $L_x L_y$ independent spins subject to pseudo-magnetic fields. Note that, differently form App.~\ref{app:mean-field-hubbard} where the chemical potential is inserted later (because in Nambu formalism it accounts for a diagonal term) here the chemical potential is part of the $z$ component of the pseudo-magnetic field, since
\begin{align}
	\hat n_{\mathbf{k}\uparrow} + \hat n_{-\mathbf{k}\downarrow} &= \hat c_{\mathbf{k}\uparrow}^\dagger \hat c_{\mathbf{k}\uparrow} + \hat c_{-\mathbf{k}\downarrow}^\dagger \hat c_{-\mathbf{k}\downarrow} \nonumber \\
	&= \hat c_{\mathbf{k}\uparrow}^\dagger \hat c_{\mathbf{k}\uparrow} - \hat c_{-\mathbf{k}\downarrow} \hat c_{-\mathbf{k}\downarrow}^\dagger + \mathbb{I} \nonumber \\
	&=  \hat \Psi_\mathbf{k}^\dagger \tau^z \hat \Psi_\mathbf{k} + \mathbb{I} \label{eq:number-operator-z-spin-relation}
\end{align}
and then it follows
\[
\begin{aligned}
	-\mu\hat N &= -\mu \sum_{\mathbf{k} \in \mathrm{BZ}} \left[ 
		\hat n_{\mathbf{k}\uparrow} + \hat n_{-\mathbf{k}\downarrow} 
	\right] \\
	&= -\mu \sum_{\mathbf{k} \in \mathrm{BZ}} \hat \Psi_\mathbf{k}^\dagger \tau^z \hat \Psi_\mathbf{k} -\mu L_x L_y
\end{aligned}
\]
Then, adding a term $-\mu \hat N$ to $\hat H$, apart from an irrelevant total energy increase, changes the pseudo-field whose explicit form becomes
\begin{equation} \label{eq:extended-hubbard-pseudo-magnetic-field}
	\mathbf{b}_\mathbf{k} \equiv \begin{bmatrix}
		-\Re{\Delta_\mathbf{k}} \\
		-\Im{\Delta_\mathbf{k}} \\ \epsilon_\mathbf{k} - \mu
	\end{bmatrix}
\end{equation}
This hamiltonian behaves as an ensemble of spins in local magnetic fields precisely as in Eq.~\eqref{appeq:hubbard-bogoliubov-hamiltonian-pseudofields},
\begin{equation}\label{eq:extended-hubbard-bogoliubov-hamiltonian-pseudofields}
	\hat H -\mu \hat N = \sum_{\mathbf{k} \in \mathrm{BZ}} \mathbf{b}_\mathbf{k} \cdot \hat{\mathbf{s}}_\mathbf{k}
	\qq{where}
	\hat{\mathbf{s}}_{\mathbf{k}\sigma} = \begin{bmatrix}
		\hat s_\mathbf{k}^x \\
		\hat s_\mathbf{k}^y \\
		\hat s_\mathbf{k}^z
	\end{bmatrix}
\end{equation}
Proceed as in App.~\ref{app:mean-field-hubbard} and diagonalize via a rotation,
\[
	d_\mathbf{k} \equiv \begin{bmatrix}
		-E_\mathbf{k} & \\ & E_\mathbf{k}
	\end{bmatrix}
	\qq{being}
	E_\mathbf{k} \equiv \sqrt{\xi_\mathbf{k}^2 + \abs{\Delta_\mathbf{k}}^2}
\]
and $\xi_\mathbf{k} \equiv \epsilon_\mathbf{k} - \mu$. Given the pseudoangles
\[
	\tan(2\theta_\mathbf{k}) \equiv \frac{\abs{\Delta_\mathbf{k}}}{\epsilon_\mathbf{k}}
	\qquad
	\tan(2\zeta_\mathbf{k}) \equiv \frac{\Im{\Delta_\mathbf{k}}}{\Re{\Delta_\mathbf{k}}}
\]
the general diagonalizer will be an orthogonal rotation matrix
\begin{align}
	W_\mathbf{k} &= e^{i \left(\theta_\mathbf{k} - \frac{\pi}{2}\right) \tau^y} e^{i \zeta_\mathbf{k} \tau^z} \nonumber \\
	&= \begin{bmatrix}
		-\sin\theta_\mathbf{k}  & -\cos\theta_\mathbf{k}  \\ 
		\cos\theta_\mathbf{k}  & -\sin\theta_\mathbf{k} 
	\end{bmatrix} \begin{bmatrix}
		e^{i\zeta_\mathbf{k} } & \\ & e^{-i\zeta_\mathbf{k} }
	\end{bmatrix} \nonumber \\
	&= \begin{bmatrix}
		-\sin\theta_\mathbf{k}  e^{i\zeta_\mathbf{k} } & -\cos\theta_\mathbf{k}  e^{-i\zeta_\mathbf{k} }  \\ 
		\cos\theta_\mathbf{k}  e^{i\zeta_\mathbf{k} } & -\sin\theta_\mathbf{k}  e^{-i\zeta_\mathbf{k} } 
	\end{bmatrix} \label{eq:extended-hubbard-bogoliubov-W-diagonalizer-explicit}
\end{align}
given by a rotation of angle $\zeta_\mathbf{k}$ around the $z$ axis, to align the $x$ axis with the field projection onto the $xy$ plane, followed by a rotation around the $y$ axis to anti-align with the pseudo-field. The MFT-BCS solution is given by a degenerate Fermi gas at ground state, whose quasi-particles occupy two bands $\pm E_\mathbf{k}$ and their fermionic operators are given by
\[
	\hat \gamma_\mathbf{k}^{(-)} \equiv \left[
		W_\mathbf{k} \hat \Psi_\mathbf{k}
	\right]_1
	\qquad
	\hat \gamma_\mathbf{k}^{(+)} \equiv \left[
		W_\mathbf{k} \hat \Psi_\mathbf{k}
	\right]_2
\]
The diagonalization operators are given by
\[
	\hat\Gamma_\mathbf{k} \equiv W_\mathbf{k} \hat\Psi_\mathbf{k}
	\qq{where}
	\hat\Gamma_\mathbf{k} = \begin{bmatrix}
		\hat \gamma_\mathbf{k}^{(-)} \\ \hat \gamma_\mathbf{k}^{(+)}
	\end{bmatrix}
\]
then, using Eq.~\eqref{appeq:finite-temperature-order-parameter-derivation-intermediate}, 
\[
	\left\langle [\hat \Psi_\mathbf{k}^\dagger]_i [\hat \Psi_\mathbf{k}]_j \right\rangle = [W_\mathbf{k}]_{1 i} [W_\mathbf{k}^\dagger]_{j 1} f\left( -E_\mathbf{k}; \beta,0 \right) + [W_\mathbf{k}]_{2 i} [W_\mathbf{k}^\dagger]_{j 2} f\left( E_\mathbf{k}; \beta,0 \right)
\]
where in the Fermi-Dirac function chemical potential was set to zero, because it already was included in the diagonalized hamiltonian.
Recalling Eq.~\eqref{eq:extended-hubbard-phi-psi-expression}, it follows
\begin{align}
	\langle \phi_\mathbf{k}^\dagger \rangle &= [W_\mathbf{k}]_{11} [W_\mathbf{k}^\dagger]_{21} f\left( -E_\mathbf{k}; \beta,0 \right) + [W_\mathbf{k}]_{21} [W_\mathbf{k}^\dagger]_{22} f\left( E_\mathbf{k}; \beta,0 \right) \label{eq:pairing-dagger-expectation-algorithmic} \\
	&= \frac{1}{2} \sin \left(
	2 \theta_\mathbf{k}
	\right) e^{i 2 \zeta_\mathbf{k}} \tanh(\frac{\beta E_\mathbf{k}}{2}) \label{eq:pairing-dagger-expectation-theoretical}
\end{align}
The last passage has been obtained by computing the matrix element from the explicit form of $W_\mathbf{k}$ of Eq.~\eqref{eq:extended-hubbard-bogoliubov-W-diagonalizer-explicit} and by the simple relation
\[
\begin{aligned}
	\frac{1}{e^{-x}+1} - \frac{1}{e^x+1} &= \frac{e^x -1}{e^x +1} \\
	&= \tanh(\frac{x}{2})
\end{aligned}
\]
Eqns.~\eqref{eq:pairing-dagger-expectation-algorithmic}, \eqref{eq:pairing-dagger-expectation-theoretical} give us both the algorithmic formula (first row) and its theoretical counterpart (second row) to compute the order parameters in the HF approach at each point in $k$-space $(k_x,k_y)$. We can finally derive the BCS self-consistency equation
\begin{equation}\label{eq:self-consistency-equation-theoretical}
	\Delta_\mathbf{k} \equiv \frac{1}{2} \sum_{\mathbf{k}'}
	\left[
	V^{(s)} +
	V_{\mathbf{k}\mathbf{k}'}
	\right] \frac{\abs{\Delta_\mathbf{k}}}{\sqrt{\xi_\mathbf{k}^2 + \abs{\Delta_\mathbf{k}}^2}} e^{i \Im{\Delta_\mathbf{k}}/\Re{\Delta_\mathbf{k}}} \tanh(\frac{\beta}{2} \sqrt{\xi_\mathbf{k}^2 + \abs{\Delta_\mathbf{k}}^2})
\end{equation}
The whole point of the HF algorithm is to find an iterative solution for each symmetry channel, using the self-consistency equation projection of Tab.~\ref{tab:x-wave-self-consistency-equation}.

Notice that the $z$ component of the spin operators is related to density: using Eq.~\eqref{appeq:finite-temperature-order-parameter-derivation-intermediate},
\[
	\langle
		\hat\Psi_\mathbf{k}^\dagger \tau^z \hat\Psi_\mathbf{k}
	\rangle = \left\langle 
		[\hat \Psi_\mathbf{k}^\dagger]_1 [\hat \Psi_\mathbf{k}]_1 
	\right\rangle - \left\langle 
		[\hat \Psi_\mathbf{k}^\dagger]_2 [\hat \Psi_\mathbf{k}]_2
	\right\rangle
\]
I proceed in as done previously, and from Eq.~\eqref{eq:number-operator-z-spin-relation},
\begin{align}
	\langle \hat n_{\mathbf{k}\uparrow} \rangle + \langle \hat 	n_{-\mathbf{k}\downarrow} \rangle = 1 &+ \langle \hat\Psi_\mathbf{k}^\dagger \tau^z \hat\Psi_\mathbf{k} \rangle \nonumber \\
	= 1 &+ \left(
		\abs{[W_\mathbf{k}]_{11}}^2 - \abs{[W_\mathbf{k}]_{12}}^2
	\right) f\left( -E_\mathbf{k}; \beta,0 \right) \nonumber \\
	&+ \left(
		\abs{[W_\mathbf{k}]_{21}}^2 - \abs{[W_\mathbf{k}]_{22}}^2
	\right) f\left( E_\mathbf{k}; \beta,0 \right) \label{eq:density-expectation-algorithmic} \\
	= 1 &- \cos \left(
		2 \theta_\mathbf{k}
	\right) \tanh(\frac{\beta E_\mathbf{k}}{2}) \label{eq:density-expectation-theoretical} 
\end{align}
The expectation value for the density is needed in order to extract the optimal chemical potential $\mu$ for the target density we aim to simulate at the given parametrization. This is numerically obtained by using Eq.~\eqref{eq:density-expectation-algorithmic} directly on the diagonalization matrix of $h_\mathbf{k}$.

\subsection{A short comment on self-consistency}\label{subsec:mft-self-consitency-cooper-decomposition}

The Bogoliubov fermions in spinor representation satisfy obviously $\hat \Psi_\mathbf{k} = W_\mathbf{k}^\dagger \hat \Gamma_\mathbf{k}$. Consider e.g.
\[
	\langle
		\hat c_{\mathbf{k}\sigma}^\dagger
		\hat c_{-\mathbf{k}\sigma}^\dagger
	\rangle	
\]
which is a spin-symmetric anomalous Cooper pair. For simplicity, take $\sigma=\uparrow$. Expand:
\[
\begin{aligned}
	\langle
		\hat c_{\mathbf{k}\uparrow}^\dagger
		\hat c_{-\mathbf{k}\uparrow}^\dagger
	\rangle	&= \left\langle 
		[\hat \Psi_\mathbf{k}^\dagger]_1 [\hat \Psi_{-\mathbf{k}}^\dagger]_1 
	\right\rangle \\
	&= \left\langle 
		[W_\mathbf{k} \hat \Gamma_\mathbf{k}^\dagger]_1 [W_{-\mathbf{k}} \hat \Gamma_{-\mathbf{k}}^\dagger]_1 
	\right\rangle
\end{aligned}
\]
This expectation value is taken over the ground-state, the latter being the vacuum of $\Gamma$ fermions. Evidently the above expectation cannot assume non-zero values. Obviously the same holds for $\sigma=\downarrow$, and this argument explains why the Ferromagnetic terms of the hamiltonian decomposition do not contribute to Cooper instability. An identical argument, with the exchange
\[
	(\sigma,\sigma) \to (\uparrow,\downarrow)
	\qq{and}
	(\mathbf{k},-\mathbf{k}) \to (\mathbf{K}+\mathbf{k},\mathbf{K}-\mathbf{k})
	\qq{with}
	\mathbf{K}\neq\mathbf{0}
\]
justifies why in Sec.~\ref{subsec:nambu-formalism-mean-field-extended-hubbard} the only relevant contribution was given by $\mathbf{K}=\mathbf{0}$. In the next sections, the results of the self-consistent HF algorithm are exposed.

\section{Results of the HF algorihtm}\label{sec:mft-analysis-hf-results}

\todo