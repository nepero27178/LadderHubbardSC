\chapter{Theoretical introduction}\label{chapter:theoretical-introduction}

\todo

\section{Antiferromagnetic ordering in the Hubbard model}

Consider the ordinary Hubbard model:
\begin{equation}\label{eq:hubbard-model}
	\hat H = 
	-t \sum_{\langle ij \rangle} \sum_\sigma \hat c_{i\sigma}^\dagger \hat c_{j\sigma}
	+ U \sum_i \hat n_{i\uparrow} \hat n_{i\downarrow}
	\qquad
	t, U  > 0
\end{equation}
The two competing mechanisms are site-hopping of amplitude $t$ and local repulsion of amplitude $U$. For this model defined \textbf{on a bipartite lattice at half filling} and fixed electron number, it is well known \cite{hirsch1985hubbard} that, below a certain critical temperature $T_c$ and above some (small) critical repulsion $U_c/t$, the ground-state acquires antiferromagnetic ($\mathrm{AF}$) long-range ordering. schematically depicted in Fig.~\ref{subfig:antiferromagnet-phase}. The mechanism for the formation of the $\mathrm{AF}$ phase takes advantage of virtual hopping, as described in App.~\ref{appendix:superexchange-virtual-hopping}; the Mean-Field Theory treatment of ferromagnetic-antiferromagnetic orderings in $2\mathrm{D}$ Hubbard lattices is rapidly discussed in  App.~\ref{appendix:mean-field-hubbard}.

\begin{figure}
	\centering
	\subfloat[Antiferromagnetic phase.]{
		\input{chapters/theoretical-introduction/pictures/antiferromagnet-phase}
		\label{subfig:antiferromagnet-phase}	
	}
	\hfil
	\subfloat[Polarized sublattices.]{
		\newcount\xLength
\xLength=4	% Even!
\newcount\yLength
\yLength=2	% Even!

\newcount\xStop
\xStop=\xLength
\divide\xStop by 2 \advance\xStop by -1\relax

\newcount\yStop
\yStop=\yLength
\divide\yStop by 2 \advance\yStop by -1\relax

\begin{tikzpicture}
	\draw[
		color=lightgray, dashed
	] 
		(-0.25,-0.25) grid ({\xLength+0.25}, {\yLength+0.25});
		
	% Nested, no indentation
	\foreach \x in {
		0,...,\xStop
	}{
	\foreach \y in {
		0,...,\yStop
	}{
		% Up sites
		\fill[color=tabred] 
			({2*\x},{2*\y}) circle (1.5pt)
			({2*\x+1},{2*\y+1}) circle (1.5pt)
		;
		
		% Red sublattice
		\draw[color=tabred] 
			({2*\x},{2*\y}) -- ++ (2,2)
			({2*\x},{2-2*\y}) -- ++ (2,-2)
		;
		
		% Down sites
		\fill[color=tabblue] 
			({2*\x},{2*\y+1}) circle (1.5pt)
			({2*\x+1},{2*\y}) circle (1.5pt)
		;
		
		% Blue sublattice
		\draw[color=tabblue] 
			({2*\x},{2*\y+1}) -- ++ (1,1)
			-- ++ (1,-1)
			-- ++ (-1,-1)
			-- ++ (-1,1)
		;
	}}

	% Border
	\foreach \x in {
		0, ..., \xStop
	}{
	\foreach \y in {
		0, ..., \yStop
	}{
	
		% Up sites
		\fill[color=tabred] 
			({2*\x},\yLength) circle (1.5pt)
			(\xLength,{2*\y}) circle (1.5pt)
		;
		
		% Down sites
		\fill[color=tabblue] 
			({2*\x+1},\yLength) circle (1.5pt)
			(\xLength,{2*\y+1}) circle (1.5pt)
		;
		
	}}

	% Topright
	\fill[color=tabred] 
		(\xLength,\yLength) circle (1.5pt);
		
	\node[color=lightgray, anchor=east] 
		at (0,0.5)
			{$\mathcal{S}$};
			
	\node[color=tabred, anchor=west] 
		at (\xLength+0.25,2)
			{$\mathcal{S}_\uparrow$};
	
	\node[color=tabblue, anchor=west] 
		at (\xLength+0.25,1)
			{$\mathcal{S}_\downarrow$};
		
\end{tikzpicture}
		\label{subfig:antiferromagnet-sublattices}	
	}
	\caption{Schematic representation of the \AF phase. Fig.~\ref{subfig:antiferromagnet-phase} shows a portion of the square lattice with explicit representation of the spin for each site. Fig.~\ref{subfig:antiferromagnet-sublattices} divides the square lattice $\mathcal{S}$ in two polarized sublattices $\mathcal{S}_\uparrow$, $\mathcal{S}_\downarrow$. The AF phase results from the interaction of two inversely polarized ``ferromagnetic'' square lattices.}
	\label{fig:antiferromagnet-schemes}
\end{figure}

\section{The Extended Fermi-Hubbard model}

The Extended Fermi-Hubbard model is defined by:
\begin{equation}\label{eq:extended-hubbard-model}
	\hat H =
	-t \sum_{\langle ij \rangle} \sum_\sigma \hat c_{i\sigma}^\dagger \hat c_{j\sigma}
	+ U \sum_i \hat n_{i\uparrow} \hat n_{i\downarrow}
	- V \sum_{\langle ij \rangle} \sum_{\sigma \sigma'} \hat n_{i\sigma} \hat n_{j\sigma'}
\end{equation}
The last term represents an effective attraction between neighboring electrons, of amplitude $V$. Such an interaction is believed \cite{cao2025p-wave} necessary to describe the insurgence of high-$T_c$ superconductivity in cuprate SCs. \todo

\subsection{Mean-Field effective hamiltonian}

{\color{tabred}[Why has this subsection no number?]}

\begin{figure}
	\centering
	\begin{tikzpicture}
	\fill[color=lightgray] 
		(0,0) circle (1.5pt)
			node[anchor=south east, color=black]
				{$i$}
		(1,0) circle (1.5pt)
			node[anchor=west, color=black]
				{$i+\delta_x$}
		(-1,0) circle (1.5pt)
			node[anchor=east, color=black]
				{$i-\delta_x$}
		(0,1) circle (1.5pt)
			node[anchor=south, color=black]
				{$i+\delta_y$}
		(0,-1) circle (1.5pt)
			node[anchor=north, color=black]
				{$i-\delta_y$};
		
	\draw[color=lightgray] 
		(-1,0) -- (1,0)
		(0,-1) -- (0,1);
\end{tikzpicture}
	\caption{Schematic representation of the four NNs of a given site $i$ for a planar square lattice.}
	\label{fig:square-nearest-neighbors}
\end{figure}

Consider the non-local term,
\[
	\hat H_V \equiv - V \sum_{\langle ij \rangle} \sum_{\sigma \sigma'} \hat n_{i\sigma} \hat n_{j\sigma'}
\]
{\color{tabred}Since the relevant values for $V$ are $\mathcal{O}(t)$, in this model $V \ll U$.} The ground-state leading contribution will be antiferromagnetic, with the square lattice decomposed in two oppositely polarized square lattices with spacing increased by a factor $\sqrt{2}$. The non-local interaction can be written as a sum of local terms on one of the two sublattices, say, the one up-polarized:
\[
	\hat H_V = \sum_{i \in \mathcal{S}_\uparrow} \hat h_V^{(i)}
	\qquad
	\hat h_V^{(i)} = -V \sum_{\ell = x,y} \left(
		\hat n_{i\uparrow} \hat n_{i + \delta_\ell \downarrow} + \hat n_{i\uparrow} \hat n_{i - \delta_\ell \downarrow} 
	\right) 
\]
Here the notation of Fig.~\ref{subfig:antiferromagnet-sublattices} is used. The two-dimensional lattice is regular-square. For each site $i$, the nearest neighbors sites are four. The notation used is $i \pm \delta_x$, $i \pm  \delta_y$ as in Fig.~\ref{fig:square-nearest-neighbors}; all of these sites are part of $\mathcal{S}_\downarrow$. Note finally that, using $i \in \mathcal{S}_\uparrow$, the sum $\sum_{\sigma \sigma'}$ has been omitted: this is because operators resulting from $(\sigma, \sigma') \neq \uparrow\downarrow$ are suppressed in a ground-state with antiferromagnetic leading contribution.

Consider a single bond, say, the one connecting sites $i$ and $i+\delta_x$. Wick's Theorem states that, if the expectation value is performed onto a coherent state,
\[
\begin{aligned}
	\langle 
		\hat n_{i\uparrow} \hat n_{i + \delta_x \downarrow}
	\rangle &= \langle 
		\hat c_{i\uparrow}^\dagger \hat c_{i + \delta_x \downarrow}^\dagger \hat c_{i + \delta_x \downarrow} \hat c_{i\uparrow} 
	\rangle \\
	&= \langle 
		\hat c_{i\uparrow}^\dagger \hat c_{i + \delta_x \downarrow}^\dagger
	\rangle \langle	
		\hat c_{i + \delta_x \downarrow} \hat c_{i\uparrow} 
	\rangle - \langle 
		\hat c_{i\uparrow}^\dagger \hat c_{i + \delta_x \downarrow}
	\rangle \langle	
		\hat c_{i + \delta_x \downarrow}^\dagger \hat c_{i\uparrow} 
	\rangle + \langle 
		\hat c_{i\uparrow}^\dagger \hat c_{i\uparrow}
	\rangle \langle	
		\hat c_{i + \delta_x \downarrow}^\dagger \hat c_{i + \delta_x \downarrow} 
	\rangle
\end{aligned}
\]
\todo