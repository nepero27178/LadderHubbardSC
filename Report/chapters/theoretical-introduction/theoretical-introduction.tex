\chapter{Theoretical introduction}\label{chapter:theoretical-introduction}

\todo

\section{Antiferromagnetic ordering in the Hubbard model}

Consider the ordinary Hubbard model:
\begin{equation}\label{eq:hubbard-model}
	\hat H = 
	-t \sum_{\langle ij \rangle} \sum_\sigma \hat c_{i\sigma}^\dagger \hat c_{j\sigma}
	+ U \sum_i \hat n_{i\uparrow} \hat n_{i\downarrow}
	\qquad
	t, U  > 0
\end{equation}
The two competing mechanisms are site-hopping of amplitude $t$ and local repulsion of amplitude $U$. For this model defined \textbf{on a bipartite lattice at half filling} and fixed electron number, it is well known \cite{hirsch1985hubbard} that, below a certain critical temperature $T_c$ and above some (small) critical repulsion $U_c/t$, the ground-state acquires antiferromagnetic ($\mathrm{AF}$) long-range ordering. schematically depicted in Fig.~\ref{subfig:antiferromagnet-phase}. The mechanism for the formation of the $\mathrm{AF}$ phase takes advantage of virtual hopping, as described in App.~\ref{appendix:superexchange-virtual-hopping}; the Mean-Field Theory treatment of ferromagnetic-antiferromagnetic orderings in $2\mathrm{D}$ Hubbard lattices is rapidly discussed in  App.~\ref{appendix:mean-field-hubbard}.

\begin{figure}
	\centering
	\subfloat[Antiferromagnetic phase.]{
		\newcount\xLength
\xLength=4	% Even!
\newcount\yLength
\yLength=2	% Even!

\newcount\xStop
\xStop=\xLength
\divide\xStop by 2 \advance\xStop by -1\relax

\newcount\yStop
\yStop=\yLength
\divide\yStop by 2 \advance\yStop by -1\relax

\def\angle{60}
\def\arrowLength{0.5}

\begin{tikzpicture}
	\draw[
	color=lightgray, dashed
	] 
	(-0.25,-0.25) grid ({\xLength+0.25}, {\yLength+0.25});
	
	% Nested, no indentation
	\foreach \x in {
		0,...,\xStop
	}{
		\foreach \y in {
			0,...,\yStop
		}{
			% Up sites
			\fill[color=tabred] 
			({2*\x},{2*\y}) circle (1.5pt)
			({2*\x+1},{2*\y+1}) circle (1.5pt)
			;
			
			% Up arrows
			\draw[color=tabred, -stealth]
			(
			{2*\x - \arrowLength/2 * cos(\angle)},
			{2*\y - \arrowLength/2 * sin(\angle)}
			) --++ (
			{\arrowLength * cos(\angle)},
			{\arrowLength * sin(\angle)}
			);
			\draw[color=tabred, -stealth]
			(
			{2*\x+1 - \arrowLength/2 * cos(\angle)},
			{2*\y+1 - \arrowLength/2 * sin(\angle)}
			) --++ (
			{\arrowLength * cos(\angle)},
			{\arrowLength * sin(\angle)}
			);
			
			% Down sites
			\fill[color=tabblue] 
			({2*\x},{2*\y+1}) circle (1.5pt)
			({2*\x+1},{2*\y}) circle (1.5pt)
			;
			
			% Down arrows
			\draw[color=tabblue, stealth-]
			(
			{2*\x+1 - \arrowLength/2 * cos(\angle)},
			{2*\y - \arrowLength/2 * sin(\angle)}
			) --++ (
			{\arrowLength * cos(\angle)},
			{\arrowLength * sin(\angle)}
			);
			\draw[color=tabblue, stealth-]
			(
			{2*\x - \arrowLength/2 * cos(\angle)},
			{2*\y+1 - \arrowLength/2 * sin(\angle)}
			) --++ (
			{\arrowLength * cos(\angle)},
			{\arrowLength * sin(\angle)}
			);
	}}
	
	% Border
	\foreach \x in {
		0, ..., \xStop
	}{
		\foreach \y in {
			0, ..., \yStop
		}{
			
			% Up sites
			\fill[color=tabred] 
			({2*\x},\yLength) circle (1.5pt)
			(\xLength,{2*\y}) circle (1.5pt)
			;
			
			% Up arrows
			\draw[color=tabred, -stealth]
			(
			{2*\x - \arrowLength/2 * cos(\angle)},
			{\yLength - \arrowLength/2 * sin(\angle)}
			) --++ (
			{\arrowLength * cos(\angle)},
			{\arrowLength * sin(\angle)}
			);
			\draw[color=tabred, -stealth]
			(
			{\xLength - \arrowLength/2 * cos(\angle)},
			{2*\y - \arrowLength/2 * sin(\angle)}
			) --++ (
			{\arrowLength * cos(\angle)},
			{\arrowLength * sin(\angle)}
			);
			
			% Down sites
			\fill[color=tabblue] 
			({2*\x+1},\yLength) circle (1.5pt)
			(\xLength,{2*\y+1}) circle (1.5pt)
			;
			
			% Down arrows
			\draw[color=tabblue, stealth-]
			(
			{2*\x+1 - \arrowLength/2 * cos(\angle)},
			{\yLength - \arrowLength/2 * sin(\angle)}
			) --++ (
			{\arrowLength * cos(\angle)},
			{\arrowLength * sin(\angle)}
			);
			\draw[color=tabblue, stealth-]
			(
			{\xLength - \arrowLength/2 * cos(\angle)},
			{2*\y+1 - \arrowLength/2 * sin(\angle)}
			) --++ (
			{\arrowLength * cos(\angle)},
			{\arrowLength * sin(\angle)}
			);
	}}
	
	% Topright
	\fill[color=tabred] 
	(\xLength,\yLength) circle (1.5pt);
	\draw[color=tabred, -stealth]
	(
	{\xLength - \arrowLength/2 * cos(\angle)},
	{\yLength - \arrowLength/2 * sin(\angle)}
	) --++ (
	{\arrowLength * cos(\angle)},
	{\arrowLength * sin(\angle)}
	);
	
\end{tikzpicture}
		\label{subfig:antiferromagnet-phase}	
	}
	\hfil
	\subfloat[Polarized sublattices.]{
		\newcount\xLength
\xLength=4	% Even!
\newcount\yLength
\yLength=2	% Even!

\newcount\xStop
\xStop=\xLength
\divide\xStop by 2 \advance\xStop by -1\relax

\newcount\yStop
\yStop=\yLength
\divide\yStop by 2 \advance\yStop by -1\relax

\begin{tikzpicture}
	\draw[
		color=lightgray, dashed
	] 
		(-0.25,-0.25) grid ({\xLength+0.25}, {\yLength+0.25});
		
	% Nested, no indentation
	\foreach \x in {
		0,...,\xStop
	}{
	\foreach \y in {
		0,...,\yStop
	}{
		% Up sites
		\fill[color=tabred] 
			({2*\x},{2*\y}) circle (1.5pt)
			({2*\x+1},{2*\y+1}) circle (1.5pt)
		;
		
		% Red sublattice
		\draw[color=tabred] 
			({2*\x},{2*\y}) -- ++ (2,2)
			({2*\x},{2-2*\y}) -- ++ (2,-2)
		;
		
		% Down sites
		\fill[color=tabblue] 
			({2*\x},{2*\y+1}) circle (1.5pt)
			({2*\x+1},{2*\y}) circle (1.5pt)
		;
		
		% Blue sublattice
		\draw[color=tabblue] 
			({2*\x},{2*\y+1}) -- ++ (1,1)
			-- ++ (1,-1)
			-- ++ (-1,-1)
			-- ++ (-1,1)
		;
	}}

	% Border
	\foreach \x in {
		0, ..., \xStop
	}{
	\foreach \y in {
		0, ..., \yStop
	}{
	
		% Up sites
		\fill[color=tabred] 
			({2*\x},\yLength) circle (1.5pt)
			(\xLength,{2*\y}) circle (1.5pt)
		;
		
		% Down sites
		\fill[color=tabblue] 
			({2*\x+1},\yLength) circle (1.5pt)
			(\xLength,{2*\y+1}) circle (1.5pt)
		;
		
	}}

	% Topright
	\fill[color=tabred] 
		(\xLength,\yLength) circle (1.5pt);
		
	\node[color=lightgray, anchor=east] 
		at (0,0.5)
			{$\mathcal{S}$};
			
	\node[color=tabred, anchor=west] 
		at (\xLength+0.25,2)
			{$\mathcal{S}_\uparrow$};
	
	\node[color=tabblue, anchor=west] 
		at (\xLength+0.25,1)
			{$\mathcal{S}_\downarrow$};
		
\end{tikzpicture}
		\label{subfig:antiferromagnet-sublattices}	
	}
	\caption{Schematic representation of the \AF phase. Fig.~\ref{subfig:antiferromagnet-phase} shows a portion of the square lattice with explicit representation of the spin for each site. Fig.~\ref{subfig:antiferromagnet-sublattices} divides the square lattice $\mathcal{S}$ in two polarized sublattices $\mathcal{S}_\uparrow$, $\mathcal{S}_\downarrow$. The AF phase results from the interaction of two inversely polarized ``ferromagnetic'' square lattices.}
	\label{fig:antiferromagnet-schemes}
\end{figure}

In this chapter the discussion is limited to the two-dimensional square lattice Hubbard model. The lattice considered has $N$ sites per side, $N^2$ sites in total. All theoretical discussion neglects border effects, thus considering $N \to +\infty$.

\section{The Extended Fermi-Hubbard model}

The Extended Fermi-Hubbard model is defined by:
\begin{equation}\label{eq:extended-hubbard-model}
	\hat H =
	-t \sum_{\langle ij \rangle} \sum_\sigma \hat c_{i\sigma}^\dagger \hat c_{j\sigma}
	+ U \sum_i \hat n_{i\uparrow} \hat n_{i\downarrow}
	- V \sum_{\langle ij \rangle} \sum_{\sigma \sigma'} \hat n_{i\sigma} \hat n_{j\sigma'}
\end{equation}
The last term represents an effective attraction between neighboring electrons, of amplitude $V$. Such an interaction is believed \cite{cao2025p-wave} necessary to describe the insurgence of high-$T_c$ superconductivity in cuprate SCs. \todo

\subsection{Mean-Field effective hamiltonian}

\begin{figure}
	\centering
	\begin{tikzpicture}
	\fill[color=lightgray] 
		(0,0) circle (1.5pt)
			node[anchor=south east, color=black]
				{$i$}
		(1,0) circle (1.5pt)
			node[anchor=west, color=black]
				{$i+\delta_x$}
		(-1,0) circle (1.5pt)
			node[anchor=east, color=black]
				{$i-\delta_x$}
		(0,1) circle (1.5pt)
			node[anchor=south, color=black]
				{$i+\delta_y$}
		(0,-1) circle (1.5pt)
			node[anchor=north, color=black]
				{$i-\delta_y$};
		
	\draw[color=lightgray] 
		(-1,0) -- (1,0)
		(0,-1) -- (0,1);
\end{tikzpicture}
	\caption{Schematic representation of the four NNs of a given site $i$ for a planar square lattice.}
	\label{fig:square-nearest-neighbors}
\end{figure}

Consider the non-local term,
\begin{equation}\label{eq:extended-hubbard-nonlocal-interaction}
	\hat H_V \equiv - V \sum_{\langle ij \rangle} \sum_{\sigma \sigma'} \hat n_{i\sigma} \hat n_{j\sigma'}
\end{equation}
{\color{tabred}Since the relevant values for $V$ are $\mathcal{O}(t)$, in this model $V \ll U$.} The ground-state leading contribution will be the antiferromagnetic state, with the square lattice decomposed in two oppositely polarized square lattices with spacing increased by a factor $\sqrt{2}$. The non-local interaction can be written as a sum of local terms on one of the two sublattices, say, the one up-polarized:
\[
	\hat H_V = \sum_{i \in \mathcal{S}_\uparrow} \hat h_V^{(i)}
	\qquad
	\hat h_V^{(i)} = -V \sum_{\ell = x,y} \left(
		\hat n_{i\uparrow} \hat n_{i + \delta_\ell \downarrow} + \hat n_{i\uparrow} \hat n_{i - \delta_\ell \downarrow} 
	\right) 
\]
Here the notation of Fig.~\ref{subfig:antiferromagnet-sublattices} is used. The two-dimensional lattice is regular-square. For each site $i$, the nearest neighbors sites are four. The notation used is $i \pm \delta_x$, $i \pm  \delta_y$ as in Fig.~\ref{fig:square-nearest-neighbors}; all of these sites are part of $\mathcal{S}_\downarrow$. Note finally that, using $i \in \mathcal{S}_\uparrow$, the sum $\sum_{\sigma \sigma'}$ has been omitted: this is because operators resulting from $(\sigma, \sigma') \neq \uparrow\downarrow$ are suppressed in a ground-state with antiferromagnetic leading contribution.

The non-local interaction contribution to energy, as a function of the $T=0$ full hamiltonian ground-state $\ket{\Psi}$, is given by
\[
\begin{aligned}
	E_V [\Psi] &= \mel{\Psi}{\hat H_V}{\Psi} \\
	&= -V \sum_{i \in \mathcal{S}_\uparrow} \sum_{\ell = x,y} \langle
	\hat n_{i\uparrow} \hat n_{i + \delta_\ell \downarrow} + \hat n_{i\uparrow} \hat n_{i - \delta_\ell \downarrow}
	\rangle 
\end{aligned}
\]
Shorthand notation has been used: $\mel{\Psi}{\cdot}{\Psi} = \langle \cdot \rangle$. Consider one specific term, say, $\hat n_{i\uparrow} \hat n_{i + \delta_x \downarrow}$. Wick's Theorem states that, if the expectation value is performed onto a coherent state,
\[
\begin{aligned}
	\langle 
	\hat n_{i\uparrow} \hat n_{i + \delta_x \downarrow}
	\rangle &= \langle 
	\hat c_{i\uparrow}^\dagger \hat c_{i + \delta_x \downarrow}^\dagger \hat c_{i + \delta_x \downarrow} \hat c_{i\uparrow} 
	\rangle \\
	&= 
	\underbrace{
		\langle 
			\hat c_{i\uparrow}^\dagger \hat c_{i + \delta_x \downarrow}^\dagger
		\rangle \langle	
			\hat c_{i + \delta_x \downarrow} \hat c_{i\uparrow} 
		\rangle 
	}_{\text{Bogoliubov}}
	- 
	\underbrace{
		\langle 
			\hat c_{i\uparrow}^\dagger \hat c_{i + \delta_x \downarrow}
		\rangle \langle	
			\hat c_{i + \delta_x \downarrow}^\dagger \hat c_{i\uparrow} 
		\rangle 
	}_{\text{Fock}}
	+ 
	\underbrace{
		\langle 
			\hat c_{i\uparrow}^\dagger \hat c_{i\uparrow}
		\rangle \langle	
			\hat c_{i + \delta_x \downarrow}^\dagger \hat c_{i + \delta_x \downarrow} 
		\rangle
	}_{\text{Hartree}}
\end{aligned}
\]
As a first approximation, the theorem is assumed to hold (which, in a $\mathrm{BCS}$-like fashion, is equivalent to assuming for the ground-state to be a coherent state). The last two terms account for single-particle interactions with a background field; they are relevant in the Hartree-Fock scheme, being direct-exchange contributions to single particle energies. The first term accounts for non-local electrons pairing, mimicking the Bogoliubov term of $\mathrm{BCS}$ theory. {\color{tabred}I assume the ground-state to be realized such that the last two terms are suppressed, while the first survives.} Energy then is cast to the form
\[
	E_V[\Psi] = -V \sum_{i \in \mathcal{S}_\uparrow} \sum_{\ell = x,y}
	\left[
		\langle
			\hat c_{i\uparrow}^\dagger \hat c_{i + \delta_\ell \downarrow}^\dagger
		\rangle \langle	
			\hat c_{i + \delta_\ell \downarrow} \hat c_{i\uparrow} 
		\rangle + \langle 
			\hat c_{i\uparrow}^\dagger \hat c_{i - \delta_\ell \downarrow}^\dagger
		\rangle \langle	
			\hat c_{i - \delta_\ell \downarrow} \hat c_{i\uparrow} 
		\rangle
	\right]
\]

The ground-state must realize the condition
\[
	\fdv{}{\bra{\Psi}} E[\Psi] = 0
\]
being $E[\Psi]$ the total energy (made up of the three terms of couplings $t$, $U$ and $V$). {\color{tabred}[Expand derivation?]} The functional derivative must be carried out in a variational fashion including a Lagrange multiplier, the latter accounting for state-norm conservation, as is done normally in deriving the Hartree-Fock approximation for the eigenenergies of the electron liquid \cite{grosso2014solid, giuliani2005quantum}. This approach leads to the conclusion that the (coherent) ground-state of the system must be an eigenstate of the mean-field effective hamiltonian:
\begin{align}\label{eq:extended-hubbard-model-effective}
	\hat H^{(\mathrm{e})} =
	&-t \sum_{\langle ij \rangle} \sum_\sigma \hat c_{i\sigma}^\dagger \hat c_{j\sigma}
	+ U \sum_i \hat n_{i\uparrow} \hat n_{i\downarrow} \\
	&- V \sum_{i \in \mathcal{S}_\uparrow} \sum_{\ell = x,y} \sum_{\delta = \pm \delta_\ell} \left[
		\langle 
			\hat c_{i\uparrow}^\dagger \hat c_{i + \delta \downarrow}^\dagger
		\rangle
		\hat c_{i + \delta \downarrow} \hat c_{i\uparrow} 
		+ \mathrm{h}.~\mathrm{c}.
	\right]
\end{align}
\todo

\subsection{Fourier-transform of the non-local interaction}

Let me take a step back and perform explicitly the Fourier-transform of the non-local interaction of Eq.~\ref{eq:extended-hubbard-nonlocal-interaction}. Consider a generic bond, say, the one connecting sites $j$ and $j\pm\delta_\ell$ (variable $i$ is here referred to as the imaginary unit to avoid confusion). $\mathbf{x}_j$ is the $2$D notation for the position of site $j$, while $\bm{\delta}_\ell$ is the $2$D notation for the lattice spacing previously indicated as $\delta_\ell$. Fourier transform it according to the convention
\[
	\hat c_{j\sigma} = \frac{1}{N} \sum_{\mathbf{q} \in \mathrm{BZ}} e^{-i \mathbf{q} \cdot \mathbf{x}_j} \hat c_{\mathbf{q}\sigma}
\]
Then:
\[
\begin{aligned}
	\hat n_{j\uparrow} \hat n_{j \pm \delta_\ell \downarrow} &= \hat c_{j\uparrow}^\dagger \hat c_{j \pm \delta_\ell \downarrow}^\dagger \hat c_{j \pm \delta_\ell \downarrow} \hat c_{j\uparrow} \\
	&= \frac{1}{N^4} \sum_{\nu=1}^4 \sum_{\mathbf{q}_\nu \in \mathrm{BZ}} e^{i \left[ (\mathbf{q}_1 + \mathbf{q}_2) - (\mathbf{q}_3 + \mathbf{q}_4) \right] \cdot \mathbf{x}_j} e^{\pm i(\mathbf{q}_2-\mathbf{q}_3) \cdot \bm{\delta}_\ell}  \hat c_{\mathbf{q}_1 \uparrow}^\dagger \hat c_{\mathbf{q}_2 \downarrow}^\dagger \hat c_{\mathbf{q}_3 \downarrow} \hat c_{\mathbf{q}_4\uparrow}
\end{aligned}
\]
It follows,
\[
\begin{aligned}
	\hat h_V^{(j)} &= -\frac{V}{N^4} \sum_{\ell = x,y} \sum_{\nu=1}^4 \sum_{\mathbf{q}_\nu \in \mathrm{BZ}} e^{i \left[ (\mathbf{q}_1 + \mathbf{q}_2) - (\mathbf{q}_3 + \mathbf{q}_4) \right] \cdot \mathbf{x}_j} 
	\left(
		e^{ i(\mathbf{q}_2-\mathbf{q}_3) \cdot \bm{\delta}_\ell} + e^{ -i(\mathbf{q}_2-\mathbf{q}_3) \cdot \bm{\delta}_\ell} 
	\right)
	\hat c_{\mathbf{q}_1 \uparrow}^\dagger \hat c_{\mathbf{q}_2 \downarrow}^\dagger \hat c_{\mathbf{q}_3 \downarrow} \hat c_{\mathbf{q}_4\uparrow} \\
	&= -\frac{2V}{N^4} \sum_{\ell = x,y} \sum_{\nu=1}^4 \sum_{\mathbf{q}_\nu \in \mathrm{BZ}} e^{i \left[ (\mathbf{q}_1 + \mathbf{q}_2) - (\mathbf{q}_3 + \mathbf{q}_4) \right] \cdot \mathbf{x}_j} \cos\left[
		(\mathbf{q}_2-\mathbf{q}_3) \cdot \bm{\delta}_\ell
	\right]	\hat c_{\mathbf{q}_1 \uparrow}^\dagger \hat c_{\mathbf{q}_2 \downarrow}^\dagger \hat c_{\mathbf{q}_3 \downarrow} \hat c_{\mathbf{q}_4\uparrow}
\end{aligned}
\]
The full non-local interaction is given by summing over all sites of one sublattice. This gives back momentum conservation,
\[
	\frac{1}{N^2} \sum_{j \in \mathcal{S}_\uparrow} e^{i \left[ (\mathbf{q}_1 + \mathbf{q}_2) - (\mathbf{q}_3 + \mathbf{q}_4) \right] \cdot \mathbf{x}_j} = \delta_{\mathbf{q}_1 + \mathbf{q}_2 = \mathbf{q}_3 + \mathbf{q}_4}
\]
Let $\mathbf{q}_1 + \mathbf{q}_2 = \mathbf{q}_3 + \mathbf{q}_4 = \mathbf{Q}$, and define $\mathbf{q}$, $\mathbf{q}'$ such that
\[
	\mathbf{q}_1 \equiv \mathbf{Q} + \mathbf{q} 
	\qquad
	\mathbf{q}_2 \equiv \mathbf{Q} - \mathbf{q} 
	\qquad
	\mathbf{q}_3 \equiv \mathbf{Q} - \mathbf{q}' 
	\qquad
	\mathbf{q}_4 \equiv \mathbf{Q} + \mathbf{q}'
	\qquad
	\Delta \mathbf{q} \equiv \mathbf{q}-\mathbf{q}'
\]
Sums over these variable must be intended as over the Brillouin Zone ($\mathrm{BZ}$). Then, finally
\[
\begin{aligned}
	\hat H_V &= \sum_{j \in \mathcal{S}_\uparrow} \hat h_V^{(j)} \\
	&= - \frac{2V}{N^2} \sum_{\ell = x,y} \sum_{\mathbf{Q}, \mathbf{q}, \mathbf{q}'} \cos\left(
		\Delta \mathbf{q} \cdot \bm{\delta}_\ell
	\right)	\hat c_{\mathbf{Q}+\mathbf{q} \uparrow}^\dagger \hat c_{\mathbf{Q}-\mathbf{q} \downarrow}^\dagger \hat c_{\mathbf{Q}-\mathbf{q}' \downarrow} \hat c_{\mathbf{Q}+\mathbf{q}'\uparrow} \\
	&= -2V \sum_{\ell = x,y} \sum_{\mathbf{q}, \mathbf{q}'} \cos\left(
	\Delta \mathbf{q} \cdot \bm{\delta}_\ell
	\right)	\hat c_{\mathbf{q} \uparrow}^\dagger \hat c_{-\mathbf{q} \downarrow}^\dagger \hat c_{-\mathbf{q}' \downarrow} \hat c_{\mathbf{q}'\uparrow} \\
	&= -2V \sum_{\mathbf{q}, \mathbf{q}'} 
	\left[
		\cos \left(
			\Delta q_x \delta_x
		\right)	+ \cos \left(
			\Delta q_y \delta_y
		\right)	
	\right]
	\hat c_{\mathbf{q} \uparrow}^\dagger \hat c_{-\mathbf{q} \downarrow}^\dagger \hat c_{-\mathbf{q}' \downarrow} \hat c_{\mathbf{q}'\uparrow}
\end{aligned}
\]
In the second passage, a sum over $\mathbf{Q}$ has been absorbed recognizing that it generates $N^2$ identical terms. \todo