\chapter{Theoretical introduction}\label{chapter:theoretical-introduction}

\todo

\section{Antiferromagnetic ordering in the Hubbard model}

Consider the ordinary Hubbard model:
\begin{equation}\label{eq:hubbard-model}
	\hat H = 
	-t \sum_{\langle ij \rangle} \sum_\sigma \hat c_{i\sigma}^\dagger \hat c_{j\sigma}
	+ U \sum_i \hat n_{i\uparrow} \hat n_{i\downarrow}
	\qquad
	t, U  > 0
\end{equation}
The two competing mechanisms are site-hopping of amplitude $t$ and local repulsion of amplitude $U$. For this model defined \textbf{on a bipartite lattice at half filling} and fixed electron number, it is well known \mref{} that below a certain critical temperature $T_c$ the ground-state acquires antiferromagnetic ($\mathrm{AF}$) long-range ordering. schematically depicted in Fig.~\ref{fig:antiferromagnet}. The mechanism for the formation of the $\mathrm{AF}$ phase takes advantage of virtual hopping, as described in App.~\ref{appendix:superexchange-virtual-hopping}.

\begin{figure}
	\centering
	\newcount\xLength
\xLength=4	% Even!
\newcount\yLength
\yLength=2	% Even!

\newcount\xStop
\xStop=\xLength
\divide\xStop by 2 \advance\xStop by -1\relax

\newcount\yStop
\yStop=\yLength
\divide\yStop by 2 \advance\yStop by -1\relax

\def\angle{60}
\def\arrowLength{0.5}

\begin{tikzpicture}
	\draw[
		color=lightgray, dashed
	] 
		(-0.25,-0.25) grid ({\xLength+0.25}, {\yLength+0.25});
		
	% Nested, no indentation
	\foreach \x in {
		0,...,\xStop
	}{
	\foreach \y in {
		0,...,\yStop
	}{
		% Up sites
		\fill[color=tabred] 
			({2*\x},{2*\y}) circle (1.5pt)
			({2*\x+1},{2*\y+1}) circle (1.5pt)
		;
		
		% Up arrows
		\draw[color=tabred, -stealth]
			(
				{2*\x - \arrowLength/2 * cos(\angle)},
				{2*\y - \arrowLength/2 * sin(\angle)}
			) --++ (
				{\arrowLength * cos(\angle)},
				{\arrowLength * sin(\angle)}
			);
		\draw[color=tabred, -stealth]
			(
				{2*\x+1 - \arrowLength/2 * cos(\angle)},
				{2*\y+1 - \arrowLength/2 * sin(\angle)}
			) --++ (
				{\arrowLength * cos(\angle)},
				{\arrowLength * sin(\angle)}
			);
		
		% Down sites
		\fill[color=tabblue] 
			({2*\x},{2*\y+1}) circle (1.5pt)
			({2*\x+1},{2*\y}) circle (1.5pt)
		;
		
		% Down arrows
		\draw[color=tabblue, stealth-]
			(
				{2*\x+1 - \arrowLength/2 * cos(\angle)},
				{2*\y - \arrowLength/2 * sin(\angle)}
			) --++ (
				{\arrowLength * cos(\angle)},
				{\arrowLength * sin(\angle)}
			);
		\draw[color=tabblue, stealth-]
			(
				{2*\x - \arrowLength/2 * cos(\angle)},
				{2*\y+1 - \arrowLength/2 * sin(\angle)}
			) --++ (
				{\arrowLength * cos(\angle)},
				{\arrowLength * sin(\angle)}
			);
	}}

	% Border
	\foreach \x in {
		0, ..., \xStop
	}{
	\foreach \y in {
		0, ..., \yStop
	}{
	
		% Up sites
		\fill[color=tabred] 
			({2*\x},\yLength) circle (1.5pt)
			(\xLength,{2*\y}) circle (1.5pt)
		;
		
		% Up arrows
		\draw[color=tabred, -stealth]
			(
				{2*\x - \arrowLength/2 * cos(\angle)},
				{\yLength - \arrowLength/2 * sin(\angle)}
			) --++ (
				{\arrowLength * cos(\angle)},
				{\arrowLength * sin(\angle)}
			);
		\draw[color=tabred, -stealth]
			(
				{\xLength - \arrowLength/2 * cos(\angle)},
				{2*\y - \arrowLength/2 * sin(\angle)}
			) --++ (
				{\arrowLength * cos(\angle)},
				{\arrowLength * sin(\angle)}
			);
		
		% Down sites
		\fill[color=tabblue] 
			({2*\x+1},\yLength) circle (1.5pt)
			(\xLength,{2*\y+1}) circle (1.5pt)
		;
		
		% Down arrows
		\draw[color=tabblue, stealth-]
			(
				{2*\x+1 - \arrowLength/2 * cos(\angle)},
				{\yLength - \arrowLength/2 * sin(\angle)}
			) --++ (
				{\arrowLength * cos(\angle)},
				{\arrowLength * sin(\angle)}
			);
		\draw[color=tabblue, stealth-]
			(
				{\xLength - \arrowLength/2 * cos(\angle)},
				{2*\y+1 - \arrowLength/2 * sin(\angle)}
			) --++ (
				{\arrowLength * cos(\angle)},
				{\arrowLength * sin(\angle)}
			);
	}}

	% Topright
	\fill[color=tabred] 
		(\xLength,\yLength) circle (1.5pt);
	\draw[color=tabred, -stealth]
		(
			{\xLength - \arrowLength/2 * cos(\angle)},
			{\yLength - \arrowLength/2 * sin(\angle)}
		) --++ (
			{\arrowLength * cos(\angle)},
			{\arrowLength * sin(\angle)}
		);
		
\end{tikzpicture}
	\caption{Schematic representation of the \AF phase.}
	\label{fig:antiferromagnet}
\end{figure}

App.~\ref{appendix:mean-field-hubbard} describes the Mean-Field Theory description of ferromagnetic-antiferromagnetic orderings in $2\mathrm{D}$ Hubbard lattices.

\section{The Extended Fermi-Hubbard model}

The Extended Fermi-Hubbard model is defined by:
\begin{equation}\label{eq:extended-hubbard-model}
	\hat H = 
	-t \sum_{\langle ij \rangle} \sum_\sigma \hat c_{i\sigma}^\dagger \hat c_{j\sigma}
	+ U \sum_i \hat n_{i\uparrow} \hat n_{i\downarrow}
	- V \sum_{\langle ij \rangle} \sum_{\sigma \sigma'} \hat n_{i\sigma} \hat n_{j\sigma'}
\end{equation}
The last term represents an effective attraction between neighboring electrons, of amplitude $V$. Such an interaction is believed \cite{cao2025p-wave} to be a necessary ingredient to describe the insurgence of high-$T_c$ superconductivity in cuprate SCs. \todo

{\color{tabred}
\begin{enumerate}
	\item Fourier transform and Brillouin zone;
	\item Pairing operator;
\end{enumerate}}