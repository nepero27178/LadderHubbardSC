\def\GraphicsFolder{parts/mft-analysis/chapters/antiferromagnetic-instability/pictures}
\chapter{Anti-Ferromagnetic instability}\label{chap:mft-af-instability}

This chapter is devoted to an in-depth discussion of the antiferromagnetic (AF) long-range ordering in the EHM by the means of MFT. In the context of the $2$D Hubbard model (HM), discussed in App.~\ref{appendix:mean-field-hubbard}, AF ordering establishes as an unconventional metallic phase which is only stable at half-filling (as is discussed in Sec.~\mref). The situation for the EHM is similar: the present analysis is carried out based on the argument that, given the symmetry structure of the non-local NN interaction, the fundamental features of the commensurate AF phase of the HM are effectively preserved, while the parameters are redefined by the interaction itself. The two most relevant effects we discuss in this chapter are the ``magnetization boost'' (Sec.~\mref) and the ``hopping renormalization`` (Sec.~\mref), the latter being a general feature of the EHM common to all phases preserving (or, at most, weakening without destroying) the translational invariance, discussed in Sec.~\ref{subsec:fock-hopping-renormalization-normal}.

\section{Sketch and general features of the MFT solution}
\todo
\subsection{Symmetry considerations for the AF phase}
\todo
\subsection{Antiferromagnetism in the conventional Hubbard model}

Within the Hubbard model, the antiferromagnetic phase (AF) is specified by the Ansatz \eqref{appeq:af-mean-field-ansatz} which is explicitly breaking translational invariance in each spin sector, while preserving $\mathrm{U}^z(1)$ and $\mathrm{U}^c(1)$ symmetries and reduces the hamiltonian to the form of Eq.~\eqref{appeq:hubbard-mean-field-hamiltonian} (ignoring the double-counting terms)
\[
	\hat H_t + \hat H_U \stackrel{\mathrm{MFT}}{\simeq} -t \sum_{\langle \mathbf{r}\mathbf{r}' \rangle} \sum_\sigma \hat c_{\mathbf{r}\sigma}^\dagger \hat c_{\mathbf{r}'\sigma}
	+ nU \sum_\mathbf{r} \left[
		\hat n_{\mathbf{r}\uparrow} + \hat n_{\mathbf{r}\downarrow}
	\right] - mU \sum_\mathbf{r} (-1)^{x+y} \left[
		\hat n_{\mathbf{r}\uparrow} - \hat n_{\mathbf{r}\downarrow}
	\right]
\]
In reciprocal space, the hamiltonian decomposes as in Eq.~\eqref{appeq:hubbard-bogoliubov-hamiltonian},
\[
	\hat H_t + \hat H_U \stackrel{\mathrm{MFT}}{\simeq} \sum_{\mathbf{k} \in \mathrm{MBZ}} \sum_\sigma \hat \Psi_{\mathbf{k}\sigma}^\dagger h_{\mathbf{k}\sigma} \hat \Psi_{\mathbf{k}\sigma}
	\qq{being}
	h_{\mathbf{k}\sigma} \equiv \begin{bmatrix}
		\epsilon_\mathbf{k} & -\Delta_\sigma \\
		-\Delta_\sigma & - \epsilon_\mathbf{k}
	\end{bmatrix}
\]
and $\Delta_\uparrow = mU$, $\Delta_\downarrow = -mU$. Nambu spinorial formulation is used,
\[
\hat \Psi_{\mathbf{k}\sigma} \equiv \begin{bmatrix}
	\hat c_{\mathbf{k}\sigma} \\
	\hat c_{\mathbf{k}+\bm{\pi}\sigma} 
\end{bmatrix}
\]
and the free electrons energy is simply the tight binding energy
\[
	\epsilon_\mathbf{k} = -2t \left(
		\cos k_x + \cos k_y
	\right)
\]
which is spin-invariant. The MFT description of the model reduces to a gas of free ``$\gamma$-fermions'', described by the Nambu spinor of Eq.~\eqref{appeq:af-diagonalized-nambu-spinor},
\[
	\hat\Gamma_{\mathbf{k}\sigma} = W_{\mathbf{k}\sigma} \hat \Psi_{\mathbf{k}\sigma} = \begin{bmatrix}
		\hat \gamma_{\mathbf{k}\sigma}^{(-)} \\ \hat \gamma_{\mathbf{k}\sigma}^{(+)}
	\end{bmatrix}
\]
where
\[
	W_{\mathbf{k}\sigma} = \begin{bmatrix}
		- \sin \theta_{\mathbf{k}\sigma} & - \cos \theta_{\mathbf{k}\sigma} \\ 
		\cos \theta_{\mathbf{k}\sigma} & - \sin \theta_{\mathbf{k}\sigma}
	\end{bmatrix}
	\qq{and}
	\sin 2\theta_{\mathbf{k}\sigma} \equiv \frac{\Delta_\sigma}{E_\mathbf{k}}
\]
These fermions populate the two bands $\pm E_\mathbf{k} = \sqrt{\epsilon_\mathbf{k}^2 + \Delta^2}$. The entire system is mapped onto an ensemble of pseudo-spins, each subject to a pseudo-field, as in Fig.~\ref{fig:pseudo-magnetic-field}. To diagonalize the system essentially means to align each pseudo-spin with the $z$ axis. Within on the notation of Fig.~\ref{appfig:pseudo-magnetic-field}, the following expectations values hold:
\begin{align}
	\langle \hat \Psi_{\mathbf{k}\sigma}^\dagger \tau^x \hat \Psi_{\mathbf{k}\sigma} \rangle &= \sin \normalfont(2\theta_\mathbf{k})  \langle \hat \Gamma_{\mathbf{k}\sigma}^\dagger \tau^z \hat \Gamma_{\mathbf{k}\sigma} \rangle \label{eq:af-x-expectation}\\
	\langle \hat \Psi_{\mathbf{k}\sigma}^\dagger \tau^y \hat \Psi_{\mathbf{k}\sigma} \rangle &= 0\\
	\langle \hat \Psi_{\mathbf{k}\sigma}^\dagger \tau^z \hat \Psi_{\mathbf{k}\sigma} \rangle &= -\cos \normalfont(2\theta_\mathbf{k}) \langle \hat \Gamma_{\mathbf{k}\sigma}^\dagger \tau^z \hat \Gamma_{\mathbf{k}\sigma} \rangle \label{eq:af-z-expectation}
\end{align}
and since the $\gamma$-fermions are free and in the rotated frame the pseudo-field points ``up'',
\begin{equation}\label{eq:af-rot-z-expectation}
	\langle \hat \Gamma_{\mathbf{k}\sigma}^\dagger \tau^z \hat \Gamma_{\mathbf{k}\sigma} \rangle = \frac{1}{2} \left[
		f\left(
			-E_\mathbf{k};\beta,\mu
		\right) - f\left(
			E_\mathbf{k};\beta,\mu
		\right)
	\right]
\end{equation}
Of these properties, to pseudospin picture is a simple and powerful tool that our MFT discussion should preserve.

\subsection{Extension to the EHM}

Consider now the non-local interaction $\hat H_V$: since only translational invariance is partially broken in the AF phase, the only relevant contributions coming from Wick's decomposition are Hartree terms and the same-spin Fock term. The discussion is thus essentially the same as for the normal phase of Chap.~\ref{chap:normal-phase}, with the only relevant difference being the Ansatz we use in Eq.~\eqref{eq:normal-density-ansatz}, substituted by the commensurate oscillatory Ansatz of Eq.~\eqref{appeq:af-mean-field-ansatz}. As for the normal phase, by design the MFT solution to the model for the given phase is described by a redefinition of the various physical quantities and mathematical objects,
\[
	t \to \tilde{t}
	\qquad
	\epsilon_\mathbf{k} \to \tilde{\epsilon}_{\mathbf{k}\sigma}
	\qquad
	E_\mathbf{k} \to \tilde{E}_{\mathbf{k}\sigma}
	\qquad
	\Delta_\sigma \to \tilde{\Delta}_{\mathbf{k}\sigma}
	\qquad
	\mu \to \tilde{\mu}
\]
The band energies renormalization is simply
\[
	\tilde{E}_{\mathbf{k}\sigma} \equiv \sqrt{\tilde{\epsilon}_{\mathbf{k}\sigma}^2 + |\tilde{\Delta}_{\mathbf{k}\sigma}|^2}
\]
and the following relations hold (the \textit{tilde} sign indicates the renormalized quantity):
\begin{align}
	\langle \hat \Psi_{\mathbf{k}\sigma}^\dagger \tau^x \hat \Psi_{\mathbf{k}\sigma} \rangle &= \sin \normalfont(2\tilde{\theta}_\mathbf{k}) \sin \normalfont(2\tilde{\zeta}_\mathbf{k}) \langle \hat \Gamma_{\mathbf{k}\sigma}^\dagger \tau^z \hat \Gamma_{\mathbf{k}\sigma} \rangle \label{eq:af-x-expectation-non-local}\\
	\langle \hat \Psi_{\mathbf{k}\sigma}^\dagger \tau^y \hat \Psi_{\mathbf{k}\sigma} \rangle &= \sin \normalfont(2\tilde{\theta}_\mathbf{k}) \cos \normalfont(2\tilde{\zeta}_\mathbf{k}) \langle \hat \Gamma_{\mathbf{k}\sigma}^\dagger \tau^z \hat \Gamma_{\mathbf{k}\sigma} \rangle \label{eq:af-y-expectation-non-local}\\
	\langle \hat \Psi_{\mathbf{k}\sigma}^\dagger \tau^z \hat \Psi_{\mathbf{k}\sigma} \rangle &= -\cos \normalfont(2\tilde{\theta}_\mathbf{k}) \langle \hat \Gamma_{\mathbf{k}\sigma}^\dagger \tau^z \hat \Gamma_{\mathbf{k}\sigma} \rangle \label{eq:af-z-expectation-non-local}
\end{align}
with:
\begin{equation}\label{eq:af-bogoliubov-angle-relations}
	\sin \normalfont(2\tilde{\xi}_\mathbf{k}) = \frac{\Re\{\tilde{\Delta}_\mathbf{k}\}}{|\tilde{\Delta}_\mathbf{k}|}
	\qquad
	\cos \normalfont(2\tilde{\xi}_\mathbf{k}) = \frac{\Im\{\tilde{\Delta}_\mathbf{k}\}}{|\tilde{\Delta}_\mathbf{k}|}
	\qquad
	\sin \normalfont(2\tilde{\theta}_\mathbf{k}) = \frac{|\tilde{\Delta}_\mathbf{k}|}{\tilde{E}_\mathbf{k}}
	\qquad
	\cos \normalfont(2\tilde{\theta}_\mathbf{k}) = \frac{\tilde{\epsilon}_\mathbf{k}}{\tilde{E}_\mathbf{k}}
\end{equation}
The physical behavior is the same as for the pure Hubbard model. In terms of MFT discussion, the AF phase is similar enough to the normal phase of Chap.~\ref{chap:normal-phase}: the RWCs are the same of Sec.~\ref{sec:normal-phase-introduction}, listed in Tabs.~\ref{tab:U-wick-terms-phases} and \ref{tab:V-wick-terms-phases}. All three Hartree contractions, responsible in the normal phase of the chemical potential shift, and the s.s. Fock contraction, that originated the hopping renormalization. Let us start by the Hartree terms: due to the ``staggered'' nature of the commensurate AF phase, their role is here more complex. Then we will move to the Fock term, discussing the MFT approximations effect on bands. Next sections are basically an extension of the calculations of Chap.~\ref{chap:normal-phase}.

\begin{figure}
	\centering
	\def\reDeltaParameter{0.8}
\def\imDeltaParameter{0.55}
\def\EpsilonParameter{0.3}
\def\r{1.5}
\pgfmathsetmacro{\DeltaParameter}{sqrt(\reDeltaParameter2 + \imDeltaParameter^2)}
\pgfmathsetmacro{\XiParameter}{atan(\imDeltaParameter/\reDeltaParameter)}

\begin{tikzpicture}
	\begin{axis}[
			axis lines=center,
			axis on top,
			%
			xlabel={$x$},
			ylabel={$y$},
			zlabel={$z$},
			xlabel style={below right},
			ylabel style={right},
			zlabel style={above},
			%
			xtick={-\reDeltaParameter},
			xticklabel={$-\Re\{\tilde{\Delta}_\mathbf{k}\}$},
			xticklabel style={above right},
			extra x ticks={\reDeltaParameter},
			extra x tick labels={$\Re\{\tilde{\Delta}_\mathbf{k}\}$},
			extra x tick style={xticklabel style={below left,yshift=-0.25em}},		
			%
			ytick={-\imDeltaParameter},
			yticklabel={$-\Im\{\tilde{\Delta}_\mathbf{k}\}$},
			yticklabel style={below,yshift=-0.25em},
			extra y ticks={\imDeltaParameter},
			extra y tick labels={$\Im\{\tilde{\Delta}_\mathbf{k}\}$},
			extra y tick style={yticklabel style={above,yshift=0.25em}},	
			%
			ztick={-\EpsilonParameter},
			extra z ticks={\EpsilonParameter},
			zticklabels={$-\tilde{\epsilon}_\mathbf{k}$,$\tilde{\epsilon}_\mathbf{k}$},
			zticklabel style={left},
			extra z ticks={\EpsilonParameter},
			extra z tick labels={$\tilde{\epsilon}_\mathbf{k}$},
			extra z tick style={zticklabel style={right,xshift=0.2em}},
			%
			xmin=-1, xmax=1,
			ymin=-1, ymax=1,
			zmin=-0.6, zmax=0.6,
			view/h=62,
			view/v=30,
			scale=2.25,
		]
		
		% Dashed lines
%		\draw[color=lightgray,dashed]
%			(axis cs:-\reDeltaParameter,0,0) -- (axis cs:-\reDeltaParameter,0,\EpsilonParameter);
%		\draw[color=lightgray,dashed]
%			(axis cs:0,0,\EpsilonParameter) -- (axis cs:-\reDeltaParameter,0,\EpsilonParameter);
%		\draw[color=lightgray,dashed]
%			(axis cs:\reDeltaParameter,0,0) -- (axis cs:\reDeltaParameter,0,-\EpsilonParameter);
%		\draw[color=lightgray,dashed]
%			(axis cs:0,0,-\EpsilonParameter) -- (axis cs:\reDeltaParameter,0,-\EpsilonParameter);

		\draw[color=lightgray,dashed]
			(axis cs:-\reDeltaParameter,-\imDeltaParameter,0) -- (axis cs:-\reDeltaParameter,-\imDeltaParameter,\EpsilonParameter);
		\draw[color=lightgray,dashed]
			(axis cs:0,0,\EpsilonParameter) -- (axis cs:-\reDeltaParameter,-\imDeltaParameter,\EpsilonParameter);
		\draw[color=lightgray,dashed]
			(axis cs:-\reDeltaParameter,-\imDeltaParameter,0) -- (axis cs:\reDeltaParameter,\imDeltaParameter,0);
		\draw[color=lightgray,dashed]
			(axis cs:\reDeltaParameter,\imDeltaParameter,0) -- (axis cs:\reDeltaParameter,\imDeltaParameter,-\EpsilonParameter);
		\draw[color=lightgray,dashed]
			(axis cs:0,0,-\EpsilonParameter) -- (axis cs:\reDeltaParameter,\imDeltaParameter,-\EpsilonParameter);
		\draw[color=lightgray,dashed]
			(axis cs:-\reDeltaParameter,-\imDeltaParameter,0) -- (axis cs:-\reDeltaParameter,0,0);
		\draw[color=lightgray,dashed]
			(axis cs:-\reDeltaParameter,-\imDeltaParameter,0) -- (axis cs:0,-\imDeltaParameter,0);
		\draw[color=lightgray,dashed]
			(axis cs:\reDeltaParameter,\imDeltaParameter,0) -- (axis cs:\reDeltaParameter,0,0);
		\draw[color=lightgray,dashed]
			(axis cs:\reDeltaParameter,\imDeltaParameter,0) -- (axis cs:0,\imDeltaParameter,0);
			
		% Field line
		\draw[color=lightgray]
			(axis cs:{1.2*\reDeltaParameter},{1.2*\imDeltaParameter},{-1.2*\EpsilonParameter}) -- (axis cs:{-1.2*\reDeltaParameter},{-1.2*\imDeltaParameter},{1.2*\EpsilonParameter});
		
		% Eigenstates
		\filldraw[color=tabblue] 
			(axis cs:-\reDeltaParameter,-\imDeltaParameter,\EpsilonParameter) circle (1.2pt)
				node[anchor=south west] (E)
					{$\tilde{E}_\mathbf{k}$};
		\filldraw[color=tabblue] 
			(axis cs:\reDeltaParameter,\imDeltaParameter,-\EpsilonParameter) circle (1.2pt)
				node[anchor=north east] (mE)
					{$-\tilde{E}_\mathbf{k}$};
		
		% Pseudo field
		\draw[color=tabgreen,-stealth]
			(axis cs:0,0,0) node (O) {} -- (axis cs:-\reDeltaParameter,-\imDeltaParameter,\EpsilonParameter) 
				node[anchor=north east]
					{$\tilde{\mathbf{b}}_{\mathbf{k}\uparrow}$};
				
		\begin{scope}[canvas is xy plane at z=0]
			\draw[color=tabgreen] 
				(0,\EpsilonParameter/2) arc [start angle=90, end angle=\XiParameter,radius=\EpsilonParameter/2]
					node[anchor=west,midway]
						{$2\tilde{\zeta}_\mathbf{k}$};
		\end{scope}		
			
		\begin{scope}[canvas is plane={O(0,0,0)x(\reDeltaParameter,\imDeltaParameter,0)y(0,0,1)}]		
			\draw[color=tabgreen] 
				(0,\EpsilonParameter/2) arc [start angle=90,end angle=90+atan(\DeltaParameter/\EpsilonParameter),radius=\EpsilonParameter/2]
					node[anchor=south, midway, xshift=-0.25em]
						{$2\tilde{\theta}_\mathbf{k}$};
						
			\draw[color=tabred,dashed,-stealth] 
				(0,\r*\EpsilonParameter) arc [start angle=90,end angle=-90+atan(\DeltaParameter/\EpsilonParameter),radius=\r*\EpsilonParameter]
					node[anchor=south west,yshift=0.25em,midway]
						{\small Diagonalization};
			\draw[color=tabred,dashed,-stealth] 
				(0,-\r*\EpsilonParameter) arc [start angle=270,end angle=90+atan(\DeltaParameter/\EpsilonParameter),radius=\r*\EpsilonParameter];
		\end{scope}
	\end{axis}
\end{tikzpicture}
	\caption{Sketch of the diagonalization of the pseudo-spin problem. The dashed line represents the diagonalization given by the $z$-axis alignment with the field direction. In the top-right side of the picture are listed the expectation values of the original Nambu spinor.}
	\label{fig:pseudo-magnetic-field}
\end{figure}

\section{Hartree effects: renormalization of chemical potential and gap}

As in Eq.~\eqref{eq:normal-V-wick-decomposition}, the same-spin and opposite-spin non-local Hartree terms are
\begin{multline*}
	\hat H_V \simeq
	\overbrace{
		-V \sum_{\ev{ij}} \sum_\sigma \left[
			\langle
			\hat n_{i\sigma}
			\rangle \hat n_{j\sigma} + \hat n_{i\sigma} \langle
			\hat n_{j\sigma}
			\rangle - \langle
			\hat n_{i\sigma}
			\rangle \langle
			\hat n_{j\sigma}
			\rangle
		\right]
	}^{\mathrm{s.s.}} \\
	\underbrace{
		-V \sum_{\ev{ij}} \sum_\sigma \left[
			\langle
			\hat n_{i\sigma}
			\rangle \hat n_{j\overline{\sigma}} + \hat n_{i\sigma} \langle
			\hat n_{j\overline{\sigma}}
			\rangle
			- \langle
			\hat n_{i\sigma}
			\rangle \langle
			\hat n_{j\overline{\sigma}}
			\rangle
		\right]
	}_{\mathrm{o.s.}} \, +\,
	(\text{all the rest})
\end{multline*}
Let $i \to \mathbf{r} = (x,y)$ and $j \to \mathbf{r}' = (x',y')$. The key difference in the present derivation, with respect to the normal phase, is the single-site density Ansatz: instead of Eq.~\eqref{eq:normal-density-ansatz}, we use Eq.~\eqref{appeq:af-mean-field-ansatz}, which describes an oscillatory density for each spin sector,
\[
	\langle \hat n_{\mathbf{r}\sigma} \rangle = n - (-1)^{x+y+\delta_{\sigma=\uparrow}} m
\]
This Ansatz is composed of two parts: the offset $n$, and the oscillatory part $m$. The algebraic derivation relative to the $n$ part is, of course, the same as for the normal phase. We now treat separately the ``operatorial'' part of the braced terms of Eq.~\eqref{eq:normal-V-wick-decomposition},
\begin{equation}\label{eq:af-V-wick-decomposition-operatorial}
	-V \sum_{\ev{ij}} \sum_\sigma \left[
		\langle
		\hat n_{i\sigma}
		\rangle \hat n_{j\sigma} + \hat n_{i\sigma} \langle
		\hat n_{j\sigma}
		\rangle
		\right] -V \sum_{\ev{ij}} \sum_\sigma \left[
		\langle
		\hat n_{i\sigma}
		\rangle \hat n_{j\overline{\sigma}} + \hat n_{i\sigma} \langle
		\hat n_{j\overline{\sigma}}
		\rangle
	\right]
\end{equation}
as well as the its scalar counterpart
\begin{equation}\label{eq:af-V-wick-decomposition-double-counting}
	V \sum_{\ev{ij}} \sum_\sigma
	\langle \hat n_{i\sigma} \rangle \langle \hat n_{j\sigma} \rangle + V \sum_{\ev{ij}} \sum_\sigma
	\langle \hat n_{i\sigma} \rangle \langle \hat n_{j\overline{\sigma}} \rangle
\end{equation}
which accounts for the energy shift to be accounted when performing Wick's contractions.

\subsection{Operatorial part and mapping on the gapped AF hamiltonian}

Let us start from the operator of Expr.~\eqref{eq:af-V-wick-decomposition-operatorial}. By inserting the Ansatz of Eq.~\eqref{appeq:af-mean-field-ansatz}, we get
\begin{multline*}
	\overbrace{
		-nV \sum_{\ev{\mathbf{r}\mathbf{r}'}} \sum_\sigma \left[
		\hat n_{\mathbf{r}'\sigma} + \hat n_{\mathbf{r}\sigma}
		\right] + mV \sum_{\ev{\mathbf{r}\mathbf{r}'}} \sum_\sigma (-1)^{\delta_{\sigma=\uparrow}} \left[
		(-1)^{x'+y'} \hat n_{\mathbf{r}'\sigma} + (-1)^{x+y} \hat n_{\mathbf{r}\sigma}
		\right]
	}^{\mathrm{s.s.}} \\
	\underbrace{
		-nV \sum_{\ev{\mathbf{r}\mathbf{r}'}} \sum_\sigma \left[
		\hat n_{\mathbf{r}'\overline{\sigma}} + \hat n_{\mathbf{r}\sigma}
		\right] + mV \sum_{\ev{\mathbf{r}\mathbf{r}'}} \sum_\sigma \left[
		(-1)^{x'+y'+\delta_{\overline{\sigma}=\uparrow}} \hat n_{\mathbf{r}'\overline{\sigma}} + (-1)^{x+y+\delta_{\sigma=\uparrow}} \hat n_{\mathbf{r}\sigma}
		\right]
	}_{\mathrm{o.s.}}
\end{multline*}
For a square lattice, if $\mathbf{r} = (x,y)$ and $\mathbf{r}' = (x',y')$ are NNs, evidently
\[
	(-1)^{x'+y'} = (-1)^{x+y+1}
\]
Moreover,
\[
	(-1)^{\delta_{\overline{\sigma}=\uparrow}} = (-1)^{\delta_{\sigma=\uparrow}+1}
\]
We obtain
\begin{multline}
	\overbrace{
		-nV \sum_{\ev{\mathbf{r}\mathbf{r}'}} \sum_\sigma \left[
		\hat n_{\mathbf{r}\sigma} + \hat n_{\mathbf{r}'\sigma}
		\right]
	}^{\text{s.s. (density)}} + \overbrace{
		mV \sum_{\ev{\mathbf{r}\mathbf{r}'}} \sum_\sigma (-1)^{x+y+\delta_{\sigma=\uparrow}} \left[
		\hat n_{\mathbf{r}\sigma} - \hat n_{\mathbf{r}'\sigma}
		\right]
	}^{\text{s.s. (magnetization)}} \\
	\underbrace{
		-nV \sum_{\ev{\mathbf{r}\mathbf{r}'}} \sum_\sigma \left[
		\hat n_{\mathbf{r}\sigma} + \hat n_{\mathbf{r}'\overline{\sigma}}
		\right]
	}_{\text{o.s. (density)}} + \underbrace{
		mV \sum_{\ev{\mathbf{r}\mathbf{r}'}} \sum_\sigma (-1)^{x+y+\delta_{\sigma=\uparrow}} \left[
		\hat n_{\mathbf{r}\sigma} + \hat n_{\mathbf{r}'\overline{\sigma}}
		\right]
	}_{\text{o.s. (magnetization)}}
	\label{eq:af-hartree-renormalization-intermediate}
\end{multline}
In the above expressions the various contribution have been separated in ``density'' contributions and ``magnetization'' contributions. Let us deal with these separately.

\paragraph{Density terms.}

Consider the s.s. and o.s. (density) terms of Expr.~\eqref{eq:af-hartree-renormalization-intermediate}, reintronducing the double counting energy shift
\begin{equation}\label{eq:af-hartree-renormalization-density}
	-nV \sum_{\ev{\mathbf{r}\mathbf{r}'}} \sum_\sigma \left[
	\hat n_{\mathbf{r}\sigma} + \hat n_{\mathbf{r}'\sigma}
	\right] -nV \sum_{\ev{\mathbf{r}\mathbf{r}'}} \sum_\sigma \left[
	\hat n_{\mathbf{r}\sigma} + \hat n_{\mathbf{r}'\overline{\sigma}}
	\right]
\end{equation}
The discussion is identical to the one of Sec.~\ref{subsec:hartree-shifts-mu} for the normal phase, with the renormalization of $\mu$ being the same of Eq.~\eqref{eq:total-hartree-mu-shift}.
\begin{equation}\label{eq:af-hartree-chemical-potential-renormalization}
	\tilde{\mu} \equiv \mu + n(2zV-U)
\end{equation}
Within this equation we are also already implicitly considering the $\mu$ shift effect due to $\hat H_U$.

\paragraph{Magnetization terms.}

The $\mathrm{s.s.}$ and $\mathrm{o.s.}$ (magnetization) terms of Expr.~\eqref{eq:af-hartree-renormalization-intermediate} are to be reduced to a renormalization of the gap function. Explicitly,
\begin{multline}
	mV \sum_{\ev{\mathbf{r}\mathbf{r}'}} \sum_\sigma (-1)^{x+y+\delta_{\sigma=\uparrow}} \left[
	\hat n_{\mathbf{r}\sigma} - \hat n_{\mathbf{r}'\sigma}
	\right] + mV \sum_{\ev{\mathbf{r}\mathbf{r}'}} \sum_\sigma (-1)^{x+y+\delta_{\sigma=\uparrow}} \left[
	\hat n_{\mathbf{r}\sigma} + \hat n_{\mathbf{r}'\overline{\sigma}}
	\right] \\
	= -2zmV \sum_\mathbf{r} (-1)^{x+y} \left[
	\hat n_{\mathbf{r}\uparrow} - \hat n_{\mathbf{r}\downarrow}
	\right] \label{eq:af-hartree-renormalization-intermediate-2}
\end{multline}
Consider now the last term of the pure Hubbard model under MFT approximations of Eq.~\eqref{appeq:hubbard-mean-field-hamiltonian},
\[
	- mU \sum_\mathbf{r} (-1)^{x+y} \left[
	\hat n_{\mathbf{r}\uparrow} - \hat n_{\mathbf{r}\downarrow}
	\right]
	\qq{(Local gap)}
\]
Expr.~\eqref{eq:af-hartree-renormalization-intermediate-2} is formally identical, thus we obtain a contribution to the renormalization of the AF gap,
\begin{equation}\label{eq:af-hartree-gap-os-renormalization}
	\Delta \to \Delta + 2zmV + (\text{s.s. contribution})
\end{equation}
Let us now move to the last part of the Hartree effects on the AF phase originated by $\hat H_V$, which is, the energy shift due to double counting terms originated by MFT decomposition of the quartic interactions $\hat H_U$ and $\hat H_V$.

\subsection{Double counting terms: energy shift due to contractions}

Let us now focus on the scalar Expr.~\eqref{eq:af-V-wick-decomposition-double-counting}: it's the sum of two parts, the first coming from s.s. sector, the second from o.s. sector.

\paragraph{s.s. sector double counting energy.}

For the s.s. part, explicitating the two spin sectors, we get
\begin{multline*}
	V \sum_{\ev{ij}} \sum_\sigma \langle
		\hat n_{i\sigma}
	\rangle \langle
		\hat n_{j\sigma}
	\rangle = \overbrace{
		V \sum_{\ev{\mathbf{r}\mathbf{r}'}} \left[
			(
				n - (-1)^{x+y} m
			) (
				n - (-1)^{x'+y'} m
			)
		\right]
	}^{\sigma=\uparrow} \\
	+ \underbrace{
		V \sum_{\ev{\mathbf{r}\mathbf{r}'}} \left[
			(
				n + (-1)^{x+y} m
			) (
				n + (-1)^{x'+y'} m
			)
		\right]
	}_{\sigma=\downarrow}
\end{multline*}
Evidently the mixed terms (those of order $\mathcal{O}(n)\times\mathcal{O}(m)$) cancel out, and considering that the sign factor is necessarily alternating on neighboiring site, the remainder is just
\[
	V \sum_{\ev{ij}} \sum_\sigma \langle
		\hat n_{i\sigma}
	\rangle \langle
		\hat n_{j\sigma}
	\rangle = V \sum_{\ev{\mathbf{r}\mathbf{r}'}} 2(
		n^2-m^2
	)
\]

\paragraph{o.s. sector double counting energy.}

Proceeding identically on the o.s. sector, we get
\begin{multline*}
	V \sum_{\ev{ij}} \sum_\sigma \langle
		\hat n_{i\sigma}
	\rangle \langle
		\hat n_{j\overline{\sigma}}
	\rangle = \overbrace{
		V \sum_{\ev{\mathbf{r}\mathbf{r}'}} \left[
			(
				n - (-1)^{x+y} m
			) (
				n + (-1)^{x'+y'} m
			)
		\right]
	}^{\sigma=\uparrow} \\
	+ \underbrace{
		V \sum_{\ev{\mathbf{r}\mathbf{r}'}} \left[
			(
				n + (-1)^{x+y} m
			) (
				n - (-1)^{x'+y'} m
			)
		\right]
	}_{\sigma=\downarrow}
\end{multline*}
which gives immediately
\[
	V \sum_{\ev{ij}} \sum_\sigma \langle
		\hat n_{i\sigma}
	\rangle \langle
		\hat n_{j\overline{\sigma}}
	\rangle = V \sum_{\ev{\mathbf{r}\mathbf{r}'}} 2(
		n^2+m^2
	)
\]
Hence, the total ``contraction energy'', given by the sum of both contributions, is identical to the normal one of Eq.~\eqref{eq:normal-V-contraction-energy} indeed:
\begin{equation}\label{eq:af-hartree-V-contraction-energy}
	E_{\mathrm{H}/V}^{(\mathrm{AF})} = 2V \sum_{\ev{ij}} \sum_\sigma n^2 = 4n^2 V \times \frac{z}{2} L_x L_y
\end{equation}
This implies that, when comparing normal and AF phase, we can rule out this energy contribution, being equal for both free energy densities.

\section{Fock effects: renormalization of the hopping parameter}\label{subsec:fock-hopping-renormalization-af}

Let us now move to the effects originated by Fock contractions. We can use the derivation of Sec.~\ref{subsec:fock-hopping-renormalization-normal} up to Eq.~\eqref{eq:normal-V-fock-int2},
\begin{multline}
	V \sum_{\ev{ij}} \sum_\sigma \left[
		\langle
		\hat c_{i\sigma}^\dagger \hat c_{j\sigma}
		\rangle \hat c_{j\sigma}^\dagger  \hat c_{i\sigma} +
		\hat c_{i\sigma}^\dagger \hat c_{j\sigma}
		\langle \hat c_{j\sigma}^\dagger  \hat c_{i\sigma} \rangle - \langle \hat c_{i\sigma}^\dagger \hat c_{j\sigma}
		\rangle \langle \hat c_{j\sigma}^\dagger  \hat c_{i\sigma} \rangle
	\right] \\
	\stackrel{\mathcal{F}}{=} \frac{2V}{L_x L_y} \sum_{\mathbf{K}, \mathbf{k}} \sum_\sigma \left[
		\cos \left(
			2 k_x
		\right)	+ \cos \left(
			2 k_y
		\right)
	\right]
	\langle
		\hat c_{\mathbf{K}+\mathbf{k} \sigma}^\dagger
		\hat c_{\mathbf{K}+\mathbf{k} \sigma}
	\rangle
	\hat c_{\mathbf{K}-\mathbf{k} \sigma}^\dagger  \hat c_{\mathbf{K}-\mathbf{k}\sigma} - E_{\mathrm{F}/V}^{(\mathrm{AF})} \label{eq:af-V-fock-int1}
\end{multline}
where $E_{\mathrm{F}/V}^{(\mathrm{AF})}$ is defined as in Eq.~\eqref{eq:reciprocal-space-non-local-interaction-fock-double-couting}, with the expecatation value being computed on the AF (thermal) ground-state. As before, we discuss the operatorial part and the scalar part separately.

\subsection{Operatorial part and mapping on the gapped AF hamiltonian}

In order to proceed from Eq.~\eqref{eq:af-V-fock-int1}, it is necessary to understand how the AF phase is realized in reciprocal space. As is exposed in App.~\ref{appendix:mean-field-hubbard}, to impose an AF Ansatz of the form
\[
	\langle \hat n_{\mathbf{r}\sigma} \rangle = n - (-1)^{x+y+\delta_{\sigma=\uparrow}} m
\]
leads to an AF ground-state of free fermions at temperature $\beta$ described by the Nambu spinor of Eq.~\eqref{appeq:af-diagonalized-nambu-spinor}. All parameters are renormalized, thus we must account for renormalized band energies $\pm \tilde{E}_{\mathbf{k}\sigma}$ as well. The ground-state is realized by simply populating the two bands $\pm \tilde{E}_{\mathbf{k}\sigma}$ as
\[
	\bigotimes_{\mathbf{k}\in\mathrm{MBZ}} \bigotimes_\sigma \left[
	\left(\hat \gamma_{\mathbf{k}\sigma}^{(-)}\right)^\dagger f(-\tilde{E}_\mathbf{k};\beta,\mu) + \left(\hat \gamma_{\mathbf{k}\sigma}^{(+)}\right)^\dagger f(\tilde{E}_\mathbf{k};\beta,\mu)
	\right] \ket{\Omega}
\]
The $\hat\gamma$ operators are normalized superpositions of two $\hat c$ operators at points in reciprocal space separated by a $\bm{\pi}$ shift. It follows that the above state is ultimately a superposition of many-body pure states, each of which has either the $\mathbf{k}\sigma$ state occupied \textit{or} the $\mathbf{k}+\bm{\pi}\sigma$ state for each $\mathbf{k}\in\mathrm{MBZ}$, $\sigma\in\lbrace\uparrow,\downarrow\rbrace$. It follows that, when computing generically $\langle \hat c_{\mathbf{k}_1\sigma}^\dagger \hat c_{\mathbf{k}_2\sigma} \rangle$, such expectation value can be non-zero if and only if $\mathbf{k}_1 = \mathbf{k}_2 + n\bm{\pi}$, being $n \in \mathbb{Z}$. Going back to Eq.~\eqref{eq:reciprocal-space-non-local-interaction-fock-intermediate}, this implies only two contributions are non-zero:
\[
	\mathbf{k} = -\mathbf{k}'
	\qq{or}
	\mathbf{k}+\bm{\pi} = -\mathbf{k}'
\]
Then Eq.~\eqref{eq:af-V-fock-int1} is reduced to:
\begin{multline}
	(\cdots) \stackrel{\mathcal{F}}{=} \frac{2V}{L_x L_y} \sum_{\mathbf{K}, \mathbf{k}} \sum_\sigma \left[
	\cos \left(
	2k_x
	\right)	+ \cos \left(
	2k_y
	\right)	
	\right]	\\
	\Big[
	\underbrace{
		\langle
		\hat c_{\mathbf{K}+\mathbf{k} \sigma}^\dagger 
		\hat 	c_{\mathbf{K}+\mathbf{k} \sigma}
		\rangle
		\hat c_{\mathbf{K}-\mathbf{k} \sigma}^\dagger  \hat c_{\mathbf{K}-\mathbf{k}\sigma}
	}_\text{Diagonal terms}
	-
	\underbrace{
		\langle
		\hat c_{\mathbf{K}+\mathbf{k} \sigma}^\dagger 
		\hat 	c_{\mathbf{K}+\mathbf{k}+\bm{\pi} \sigma}
		\rangle
		\hat c_{\mathbf{K}-\mathbf{k} \sigma}^\dagger  \hat c_{\mathbf{K}-\mathbf{k}-\bm{\pi}\sigma}
	}_\text{Off-diagonal terms}
	\Big] \label{eq:af-V-fock-int2}
\end{multline}
Now, the above equation presents \textit{diagonal} and \textit{off-diagonal} terms.

\paragraph{Diagonal terms.}

\begin{figure}
	\def\Kx{0.5}
	\def\Ky{0.2}
	\def\kx{0.1}
	\def\ky{0.7}
	\centering
	\subfloat[
	Generic vectors.
	]{
		\label{subfig:af-bz-fock-terms-generic}
		\begin{tikzpicture}
	\begin{axis}[
			axis lines=center,
			grid=both,
			grid style={color=lightgray,dashed},
			%
			xlabel={$k_x$},
			xlabel style={right},
			ylabel={$k_y$},
			ylabel style={above},
			xtick={-1},
			xticklabel={$-\pi$},
			xticklabel style={below left},
			extra x ticks={1},
			extra x tick label={$\pi$},
			extra x tick style={xticklabel style={below right}},
			ytick={-1},
			yticklabel={$-\pi$},
			yticklabel style={below left},
			extra y ticks={1},
			extra y tick label={$\pi$},
			extra y tick style={yticklabel style={above left}},
			xmin=-1.32, xmax=1.32,
			ymin=-1.32, ymax=1.32,
			%
			width=0.5\textwidth,
			height=0.5\textwidth
		]
		
		\fill[color=lightgray,opacity=0.5]
			(axis cs:-1,-1) rectangle (axis cs:1,1) 
				node[anchor=north west,color=gray,opacity=1]
					{\small BZ};
					
		\filldraw[color=tabblue]
			(axis cs:\Kx,\Ky) circle (0.75pt);
		\draw[color=tabblue,-stealth]
			(axis cs:0,0) -- (axis cs:\Kx,\Ky)
				node[anchor=south,midway]
					{\small $\mathbf{K}$};
		\draw[color=tabblue,-stealth]
			(axis cs:\Kx,\Ky) -- (axis cs:{\Kx+\kx},{\Ky+\ky})
				node[anchor=west,midway]
					{\small $\mathbf{k}$};
		\draw[color=tabblue,-stealth,dashed]
			(axis cs:\Kx,\Ky) -- (axis cs:{\Kx-\kx},{\Ky-\ky})
				node[anchor=west,midway]
					{\small $-\mathbf{k}$};

					
		\filldraw[color=tabred]
			(axis cs:0,0) circle (1.5pt)
				node[anchor=south east]
					{\small $\mathbf{0}$};
		\filldraw[color=tabred]
			(axis cs:1,1) circle (1.5pt)
				node[anchor=south east]
					{\small $\bm{\pi}$};
		
	\end{axis}
\end{tikzpicture}
	}
	\subfloat[
	Change of variables.
	]{
		\label{subfig:af-bz-fock-terms-newvariables}
		\input{\GraphicsFolder/af-bz-fock-terms-newvariables}
	}
	\caption{Representation of the vectors involved in the diagonal terms of Eq.~\eqref{eq:reciprocal-space-non-local-interaction-fock-intermediate-2}. In Fig.~\ref{subfig:af-bz-fock-terms-generic} generic vectors are considered, cycling over all values of $\mathbf{K},\mathbf{k}\in\mathrm{BZ}$. In Fig.~\ref{subfig:af-bz-fock-terms-newvariables} is depicted the variables change to the new vectors $\mathbf{q},\mathbf{q}'$.}
	\label{fig:af-bz-fock-terms}
\end{figure}

The diagonal terms of Eq.~\eqref{eq:af-V-fock-int2} are simple density interactions with the mean density field. Consider Fig.~\ref{subfig:af-bz-fock-terms-generic}: density at vector $\mathbf{q}\equiv\mathbf{K}+\mathbf{k}$ interacts with the mean density at vector $\mathbf{q}'\equiv\mathbf{K}-\mathbf{k}$. These variables are depicted in Fig.~\ref{subfig:af-bz-fock-terms-newvariables}. Apply in the diagonal part of Eq.~\eqref{eq:af-V-fock-int2} this variables change (which is the same of Eq.~\eqref{eq:normal-V-fock-int3})
\begin{align}
	\frac{2V}{L_x L_y} \sum_{\mathbf{q}, \mathbf{q}'} \sum_\sigma \left[
	\cos \left(
	2k_x
	\right)	+ &\cos \left(
	2k_y
	\right)	
	\right] \langle
	\hat c_{\mathbf{K}+\mathbf{k} \sigma}^\dagger 
	\hat 	c_{\mathbf{K}+\mathbf{k} \sigma}
	\rangle
	\hat c_{\mathbf{K}-\mathbf{k} \sigma}^\dagger  \hat c_{\mathbf{K}-\mathbf{k}\sigma} \nonumber \\
	&= \frac{2V}{L_x L_y} \sum_{\mathbf{q}, \mathbf{q}'} \sum_\sigma \left[
	\cos \left( \delta q_x \right) + \cos \left( \delta q_y \right)	
	\right] \langle
	\hat c_{\mathbf{q}\sigma}^\dagger  \hat c_{\mathbf{q}\sigma}
	\rangle
	\hat c_{\mathbf{q}'\sigma}^\dagger  \hat c_{\mathbf{q}'\sigma}
	\nonumber \\
	&\stackrel{\mathcal{*}}{=} \sum_{\gamma,\sigma} \left[
		\frac{1}{L_x L_y} \sum_\mathbf{q} \varphi_\mathbf{q}^{(\gamma)}
		\langle
			\hat c_{\mathbf{q}\sigma}^\dagger
			\hat c_{\mathbf{q}\sigma}
		\rangle
	\right] \times V
	\sum_{\mathbf{q}'}
	\varphi_{\mathbf{q}'}^{(\gamma)*}
	\hat c_{\mathbf{q}'\sigma}^\dagger
	\hat c_{\mathbf{q}'\sigma} \label{eq:af-V-fock-int3}
\end{align}
where the sign $\stackrel{*}{=}$ indicates that we have substituted the decomposition of Eq.~\eqref{eq:coscos-symmetries-decomposition}. Define the expectation value
\[
	g_{\mathbf{q}\sigma} \equiv \langle
		\hat c_{\mathbf{q}\sigma}^\dagger
		\hat c_{\mathbf{q}\sigma}
	\rangle
\]
The expression in square brackets of Eq.~\eqref{eq:af-V-fock-int3} is just its $\gamma$ component, and is expanded in
\[
\begin{aligned}
	g^{(\gamma)} &= \frac{1}{L_x L_y} \sum_{\mathbf{q}\in\mathrm{BZ}} \varphi_\mathbf{q}^{(\gamma)}
	\langle
		\hat c_{\mathbf{q}\sigma}^\dagger
		\hat c_{\mathbf{q}\sigma}
	\rangle \\
	&= \frac{1}{L_x L_y} \sum_{\mathbf{q}\in\mathrm{MBZ}} \left[
		\varphi_\mathbf{q}^{(\gamma)}
		\langle
			\hat c_{\mathbf{q}\sigma}^\dagger
			\hat c_{\mathbf{q}\sigma}
		\rangle
		+
		\varphi_{\mathbf{q}+\bm{\pi}}^{(\gamma)}
		\langle
			\hat c_{\mathbf{q}+\bm{\pi}\sigma}^\dagger
			\hat c_{\mathbf{q}+\bm{\pi}\sigma}
		\rangle
	\right] \\
	&= \frac{1}{L_x L_y} \sum_{\mathbf{q}\in\mathrm{MBZ}} \varphi_\mathbf{q}^{(\gamma)} \left[
		\langle
			\hat c_{\mathbf{q}\sigma}^\dagger
			\hat c_{\mathbf{q}\sigma}
		\rangle
		-
		\langle
			\hat c_{\mathbf{q}+\bm{\pi}\sigma}^\dagger
			\hat c_{\mathbf{q}+\bm{\pi}\sigma}
		\rangle
	\right] \\
	&= \frac{1}{L_x L_y} \sum_{\mathbf{q}\in\mathrm{MBZ}} \varphi_\mathbf{q}^{(\gamma)}
	\langle
		\hat \Psi_{\mathbf{q}\sigma}^\dagger \tau^z \hat \Psi_{\mathbf{q}\sigma}
	\rangle \\
	&= -\frac{1}{2L_x L_y} \sum_{\mathbf{q}\in\mathrm{MBZ}} \varphi_\mathbf{q}^{(\gamma)}
	\frac{\tilde{\epsilon}_\mathbf{q}}{\tilde{E}_\mathbf{q}} \left[
		f\left(
			-\tilde{E}_\mathbf{q};\beta,\tilde{\mu}
		\right) - f\left(
			\tilde{E}_\mathbf{q};\beta,\tilde{\mu}
		\right)
	\right]
\end{aligned}
\]
where, in the last passage, Eqns.~\eqref{eq:af-z-expectation} and \eqref{eq:af-rot-z-expectation} have been used. Now, the AF phase does not break inversion symmetry: this implies that, whatever the shape of the renormalized bands, it must be
\[
	\tilde{\epsilon}_\mathbf{q} = \tilde{\epsilon}_{-\mathbf{q}}
	\quad\implies\quad
	g^{(p_\ell)} = 0
	\qq{with}
	\ell\in\lbrace x,y \rbrace
\]
The renormalized bands need to be a symmetric function of the moment. Hence for $\gamma\in\lbrace p_x, p_y \rbrace$ the above sum vanishes. Thus, Eq.~\eqref{eq:af-V-fock-int3} becomes
\begin{multline}
		\frac{2V}{L_x L_y} \sum_{\mathbf{q}, \mathbf{q}'} \sum_\sigma \left[
			\cos \left(
				2k_x
		\right)	+ \cos \left(
				2k_y
		\right)
	\right] \langle
	\hat c_{\mathbf{K}+\mathbf{k} \sigma}^\dagger
	\hat c_{\mathbf{K}+\mathbf{k} \sigma}
	\rangle
	\hat c_{\mathbf{K}-\mathbf{k} \sigma}^\dagger  \hat c_{\mathbf{K}-\mathbf{k}\sigma} \nonumber \\
	= V \sum_{\mathbf{q},\sigma} \left[
		g^{(s^*)} (\cos q_x + \cos q_y) + g^{(d)} (\cos q_x - \cos q_y)
	\right] \hat c_{\mathbf{q} \sigma}^\dagger
	\hat c_{\mathbf{q}\sigma}
\end{multline}
These two factors, together, define the full bands renormalization
\[
	\tilde{\epsilon}_\mathbf{k} \equiv \epsilon_\mathbf{k} + g^{(s^*)} V (\cos q_x + \cos q_y) + g^{(d)} V (\cos q_x - \cos q_y)
\]
The $s^*$-wave component is particularly handy, because if we let:
\begin{align}
	w^{(\mathrm{AF})} \equiv \frac{g^{(s^*)}}{2}
\end{align}
we can interpret part of the bands renormalization as a rigid hopping shift,
\[
	t \to \tilde{t} = t - w^{(\mathrm{AF})} V
\]
\todo

\color{tabred}
In the AF phase, $\langle \hat c_{\mathbf{q}\sigma}^\dagger  \hat c_{\mathbf{q}\sigma} \rangle$ must be $s^*$-wave symmetric, as anticipated in the starting discussion of Sec.~\ref{subsec:fock-hopping-renormalization-af}. This is due to the fact that of the four symmetries listed above, only the first one exhibits both $x$, $y$ reflections symmetry and $\pi/2$ rotational invariance. It follows, only the $s^*$-wave component when coupled to $\langle \hat c_{\mathbf{q}\sigma}^\dagger  \hat c_{\mathbf{q}\sigma} \rangle$ in Eq.~\eqref{eq:reciprocal-space-non-local-interaction-fock-dd-intermediate} gives a non-null contribution, reducing the latter to
\begin{multline}
	\frac{2V}{L_x L_y} \sum_{\mathbf{q}, \mathbf{q}'} \sum_\sigma \left[
	\cos \left( \delta q_x \right) + \cos \left( \delta q_y \right)	
	\right] \langle
	\hat c_{\mathbf{q}\sigma}^\dagger  \hat c_{\mathbf{q}\sigma}
	\rangle
	\hat c_{\mathbf{q}'\sigma}^\dagger  \hat c_{\mathbf{q}'\sigma} \\
	= \frac{V}{L_x L_y} \sum_{\mathbf{q}'\sigma} \left(
	\cos q_x' + \cos q_y'
	\right) \hat c_{\mathbf{q}'\sigma}^\dagger  \hat c_{\mathbf{q}'\sigma} \sum_\mathbf{q} \left(
	\cos q_x + \cos q_y
	\right) \langle
	\hat c_{\mathbf{q}\sigma}^\dagger  \hat c_{\mathbf{q}\sigma}
	\rangle
	\label{eq:reciprocal-space-non-local-interaction-fock-dd-intermediate-2}
\end{multline}
Note that, for two vectors separated by a $\bm{\pi}$ shift,
\[
\cos q_x + \cos q_y = - \cos \left( q_x+\pi \right) - \cos \left( q_y+\pi \right)
\]
Because of this feature, changing the variables names $\mathbf{q}'\to\mathbf{k}$, $\mathbf{q}\to\mathbf{k}'$ for the sake of general aesthetic coherence, it becomes evident that the above equation gives the bands renormalization:
\[
\begin{aligned}
	\epsilon_\mathbf{k} &\equiv -2t \left(
	\cos k_x + \cos k_y
	\right) \\
	\tilde{\epsilon}_\mathbf{k} &\equiv \epsilon_\mathbf{k} + 
	\left[
	\frac{1}{2L_xL_y} \sum_{\mathbf{k}'} \left(
	\cos k_x' + \cos k_y'
	\right) \langle
	\hat c_{\mathbf{k}'\sigma}^\dagger  \hat c_{\mathbf{k}'\sigma}
	\rangle
	\right] \times 2V \left(
	\cos k_x + \cos k_y
	\right)
\end{aligned}
\]
Note that on the left-hand side $\tilde{\epsilon}_\mathbf{k}$ is independent of $\sigma$. To explain this, let:
\[
w_\sigma^{(\mathbf{0})} \equiv \frac{1}{2L_xL_y} \sum_{\mathbf{k}\in\mathrm{BZ}} \left(
\cos k_x + \cos k_y
\right) \langle
\hat c_{\mathbf{k}\sigma}^\dagger  \hat c_{\mathbf{k}\sigma}
\rangle
\]
By simple symmetry considerations, it must be $\langle \hat c_{\mathbf{k}\uparrow}^\dagger  \hat c_{\mathbf{k}\uparrow} \rangle = \langle \hat c_{\mathbf{k}\downarrow}^\dagger  \hat c_{\mathbf{k}\downarrow} \rangle$ (as is later seen explicitly). Then,
\[
w_\uparrow^{(\mathbf{0})} = w_\downarrow^{(\mathbf{0})} \equiv w^{(\mathbf{0})}
\]
The computation can be simplified:
\begin{align}
	w^{(\mathbf{0})} &= \frac{1}{2L_xL_y} \sum_{\mathbf{k}\in\mathrm{BZ}} \left(
	\cos k_x + \cos k_y
	\right) \langle
	\hat c_{\mathbf{k}\uparrow}^\dagger \hat c_{\mathbf{k}\uparrow}
	\rangle \nonumber \\
	&= \frac{1}{2L_xL_y} \sum_{\mathbf{k}\in\mathrm{MBZ}} \left(
	\cos k_x + \cos k_y
	\right) \langle
	\hat c_{\mathbf{k}\uparrow}^\dagger  \hat c_{\mathbf{k}\uparrow} - \hat c_{\mathbf{k}+\bm{\pi}\uparrow}^\dagger  \hat c_{\mathbf{k}+\bm{\pi}\uparrow}
	\rangle \nonumber \\
	&= \frac{1}{2L_xL_y} \sum_{\mathbf{k}\in\mathrm{MBZ}} \left(
	\cos k_x + \cos k_y
	\right) \langle
	\hat \Psi_{\mathbf{k}\uparrow}^\dagger \tau^z \hat \Psi_{\mathbf{k}\uparrow}
	\rangle \nonumber \\
	&= -\frac{1}{4L_xL_y} \sum_{\mathbf{k}\in\mathrm{MBZ}} \left(
	\cos k_x + \cos k_y
	\right) \frac{\tilde{\epsilon}_\mathbf{k}}{\tilde{E}_\mathbf{k}} \left[
	f\left(
	-\tilde{E}_\mathbf{k};\beta,\tilde{\mu}
	\right) - f\left(
	\tilde{E}_\mathbf{k};\beta,\tilde{\mu}
	\right)
	\right] \label{eq:af-renormalized-self-consistent-equation-w0}
\end{align}
where in the second passage the sign change is due to the presence of the structure factor, and in the fourth passage Eq.~\eqref{eq:af-z-expectation-non-local} and the relations \eqref{eq:af-bogoliubov-angle-relations} have been used. It follows, finally, that the hopping parameter gets effectively renormalized:
\begin{equation}\label{eq:af-fock-hop-ss-renormalization}
	\tilde{t} \equiv t - w^{(\mathbf{0})}V
\end{equation}
The full effective MFT hamiltonian is spin-independent, then similarly the renormalized parameters cannot exhibit spin dependency. This justifies the fact that $\tilde{t}$ is spin-independent, and so is $\tilde{\epsilon}_\mathbf{k}$.

\paragraph{Off-diagonal terms.}

Consider the off-diagonal terms of Eq.~\eqref{eq:reciprocal-space-non-local-interaction-fock-intermediate-2}. These contribute instead to the gap renormalization, being out of diagonal in the $2\times2$ hamiltonian matrix. Define $\mathbf{q}$, $\mathbf{q}'$ as in Fig.~\ref{subfig:af-bz-fock-terms-newvariables}, and rewrite
\begin{multline*}
	-\frac{2V}{L_x L_y} \sum_{\mathbf{K}, \mathbf{k}} \sum_\sigma \left[
	\cos \left(
	2k_x
	\right)	+ \cos \left(
	2k_y
	\right)	
	\right]
	\langle
	\hat c_{\mathbf{K}+\mathbf{k} \sigma}^\dagger 
	\hat 	c_{\mathbf{K}+\mathbf{k}+\bm{\pi} \sigma}
	\rangle
	\hat c_{\mathbf{K}-\mathbf{k} \sigma}^\dagger  \hat c_{\mathbf{K}-\mathbf{k}-\bm{\pi}\sigma} \\
	= -\frac{2V}{L_xL_y} \sum_{\mathbf{q}} \langle
	\hat c_{\mathbf{q}\sigma}^\dagger 
	\hat c_{\mathbf{q}+\bm{\pi}\sigma}
	\rangle \sum_{\mathbf{q}'\sigma} \left[
	\cos \left(
	\delta q_x
	\right)	+ \cos \left(
	\delta q_y
	\right)	
	\right] \hat c_{\mathbf{q}'\sigma}^\dagger  \hat c_{\mathbf{q}'+\bm{\pi}\sigma}
\end{multline*}
Identical considerations about the $s^*$-wave symmetry structure of the expectation value $\langle \hat c_{\mathbf{q}\sigma}^\dagger \hat c_{\mathbf{q}+\bm{\pi}\sigma} \rangle$ as in the above paragraph hold. Once again renaming the variables $\mathbf{q}'\to\mathbf{k}$, $\mathbf{q}\to\mathbf{k}'$ for the sake of general aesthetic coherence, this gives
\begin{multline}
	-\frac{2V}{L_x L_y} \sum_{\mathbf{K}, \mathbf{k}} \sum_\sigma \left[
	\cos \left(
	2k_x
	\right)	+ \cos \left(
	2k_y
	\right)	
	\right]
	\langle
	\hat c_{\mathbf{K}+\mathbf{k} \sigma}^\dagger 
	\hat 	c_{\mathbf{K}+\mathbf{k}+\bm{\pi} \sigma}
	\rangle
	\hat c_{\mathbf{K}-\mathbf{k} \sigma}^\dagger  \hat c_{\mathbf{K}-\mathbf{k}-\bm{\pi}\sigma} \\
	= -2V \left[
	\frac{1}{2L_xL_y} \sum_{\mathbf{k}'}
	\left(
	\cos k_x' + \cos k_y'
	\right) \langle
	\hat c_{\mathbf{k}'\sigma}^\dagger  \hat c_{\mathbf{k}'+\bm{\pi}\sigma}
	\rangle
	\right]
	\sum_{\mathbf{k}\sigma} \left(
	\cos k_x + \cos	k_y	
	\right)
	\hat c_{\mathbf{k}\sigma}^\dagger  \hat c_{\mathbf{k}+\bm{\pi}\sigma}
\end{multline}
Because of this, the $x$ component of the pseudo-magnetic field -- the gap already renormalized by Eq.~\eqref{eq:af-hartree-gap-os-renormalization} when analyzing o.s. terms -- takes up another renormalization contribution, finally giving
\begin{equation}\label{eq:af-fock-gap-ss-renormalization-intermediate}
	\tilde{\Delta}_{\mathbf{k}\sigma} \equiv m(U + 2zV) \times (-1)^{\delta_{\sigma=\uparrow}} + i2V w_\sigma^{(\bm{\pi})} \left(
	\cos k_x + \cos	k_y	
	\right)
\end{equation}
where
\[
w_\sigma^{(\bm{\pi})} \equiv -\frac{i}{2L_xL_y} \sum_{\mathbf{k}\in\mathrm{BZ}}
\left(
\cos k_x + \cos k_y
\right) \langle
\hat c_{\mathbf{k}\sigma}^\dagger  \hat c_{\mathbf{k}+\bm{\pi}\sigma}
\rangle
\]
As will be clear in few lines, $w_\sigma^{(\bm{\pi})}$ as is defined here is purely real (due to the presence of a $-i$ prefactor). This makes $\tilde{\Delta}_{\mathbf{k}\sigma}$ made of two contributions,
\[
\Re\{\tilde{\Delta}_{\mathbf{k}\sigma}\} = m(U + 2zV) \times (-1)^{\delta_{\sigma=\uparrow}}
\qquad
\Im\{\tilde{\Delta}_{\mathbf{k}\sigma}\} = 2V w_\sigma^{(\bm{\pi})} \left(
\cos k_x + \cos	k_y	
\right)
\]
Now, since the gapped band value cannot depend on the spin index for symmetry reasons,
\[
\tilde{E}_\mathbf{k} = \sqrt{
	\tilde{\epsilon}_\mathbf{k}^2 + |\tilde{\Delta}_{\mathbf{k}\sigma}|^2
}
\]
this implies necessarily $|\tilde{\Delta}_{\mathbf{k}\uparrow}| = |\tilde{\Delta}_{\mathbf{k}\downarrow}|$. This is possible either if $w_\uparrow^{(\bm{\pi})} = \pm w_\downarrow^{(\bm{\pi})}$. Actually, in the end the exact sign does not matter: all that matters is the gap amplitude $|\tilde{\Delta}_{\mathbf{k}\sigma}|$, thus we may restrict to $\sigma=\uparrow$ and omit from now on the spin index. {\color{tabred}[
	Not so sure about this.	
	]}. This then gives us the final result for the renormalized gap function,
\begin{equation}\label{eq:af-fock-gap-ss-renormalization}
	\tilde{\Delta}_\mathbf{k} \equiv
	m(U + 2zV) + 2iw^{(\bm{\pi})}V
	\left(
	\cos k_x + \cos	k_y	
	\right)
\end{equation}
This result, together with Eqns.~\eqref{eq:af-hartree-chemical-potential-renormalization} and \eqref{eq:af-fock-hop-ss-renormalization}, concludes the renormalization of all parameters due to the non-local interaction. To calculate $w^{(\bm{\pi})}$ self consistently, we may use:
\begin{align}
	w^{(\bm{\pi})} &= -\frac{i}{2L_xL_y} \sum_{\mathbf{k}\in\mathrm{BZ}}
	\left(
	\cos k_x + \cos k_y
	\right) \langle
	\hat c_{\mathbf{k}\uparrow}^\dagger  \hat c_{\mathbf{k}+\bm{\pi}\uparrow}
	\rangle \nonumber \\
	&= -\frac{i}{2L_xL_y} \sum_{\mathbf{k}\in\mathrm{MBZ}}
	\left(
	\cos k_x + \cos k_y
	\right) \langle
	\hat c_{\mathbf{k}\uparrow}^\dagger  \hat c_{\mathbf{k}+\bm{\pi}\uparrow} - \hat c_{\mathbf{k}+\bm{\pi}\uparrow}^\dagger  \hat c_{\mathbf{k}\uparrow}
	\rangle \nonumber  \\
	&= \frac{1}{2L_xL_y} \sum_{\mathbf{k}\in\mathrm{MBZ}}
	\left(
	\cos k_x + \cos k_y
	\right) \langle
	\hat \Psi_{\mathbf{k}\uparrow}^\dagger \tau^y \hat \Psi_{\mathbf{k}\uparrow}
	\rangle \nonumber \\
	&= \frac{1}{4L_xL_y} \sum_{\mathbf{k}\in\mathrm{MBZ}}
	\left(
	\cos k_x + \cos k_y
	\right) \frac{\Im\{\tilde{\Delta}_\mathbf{k}\}}{\tilde{E}_\mathbf{k}} \left[
	f\left(
	-\tilde{E}_\mathbf{k};\beta,\tilde{\mu}
	\right) - f\left(
	\tilde{E}_\mathbf{k};\beta,\tilde{\mu}
	\right)
	\right] \label{eq:af-renormalized-self-consistent-equation-wpi}
\end{align}
Notice that this expression is purely real, as promised, and contributes to the $y$ component of the pseudo-field of Fig.~\ref{fig:pseudo-magnetic-field}.

\subsection{Double counting terms: energy shift due to contractions}
\todo

\subsection{Renormalized hamiltonian behavior}

Summing up, the non-local interaction $\hat H_V$ when discussed within MFT affects the EHM hamiltonian by renormalizing the various parameters as:
\[
\begin{aligned}
	\tilde{\mu} &\equiv \mu + 2znV \\
	\tilde{t} &\equiv t - w^{(\mathbf{0})} V \\
	\tilde{\Delta}_\mathbf{k} &\equiv
	m(U + 2zV) + 2iw^{(\bm{\pi})}V
	\left[
	\cos \left(
	k_x
	\right)	+ \cos \left(
	k_y
	\right)	
	\right]
\end{aligned}
\]
Various details are to be noted. First, the non-local interaction both contributes by enlarging the real part of the gap {\color{tabred}[
	To be understood: why does a non-local attraction increase the gap?
	]} as well as introducing a $s^*$-wave shaped imaginary gap. Interestingly, if
\[
\left(w^{(\mathbf{0})}\right)^{-1} = V/t
\]
the diffusive part of the hamiltonian drops to zero. For even larger values, diffusion becomes energetically expensive and $V$-induced localization appears.

The new set of Hartree-Fock parameters to be determined is given by the vector
\[
\mathbf{v} \equiv \begin{bmatrix}
	m \\ w^{(\mathbf{0})} \\ w^{(\bm{\pi})}
\end{bmatrix}
\]
Its three components are self-consistently determined by Eqns.~\eqref{eq:af-renormalized-self-consistent-equation-w0} and \eqref{eq:af-renormalized-self-consistent-equation-wpi}. The self-consistent equation for $m$ comes from Eq.~\eqref{appeq:antiferromagnet-magnetization-self-consistence-finite-temperature}, and reads
\begin{align}
	m &= \frac{1}{2L_xL_y} \sum_{\mathbf{k} \in \mathrm{BZ}} \langle
	\hat c_{\mathbf{k}\uparrow}^\dagger  \hat c_{\mathbf{k}+\bm{\pi}\uparrow} - \hat c_{\mathbf{k}\downarrow}^\dagger  \hat c_{\mathbf{k}+\bm{\pi}\downarrow}
	\rangle \nonumber \\
	&= \frac{1}{2L_xL_y} \sum_{\mathbf{k} \in \mathrm{MBZ}} \langle
	\hat \Psi_{\mathbf{k}\uparrow}^\dagger \tau^x \hat \Psi_{\mathbf{k}\uparrow} - \hat \Psi_{\mathbf{k}\downarrow}^\dagger \tau^x \hat \Psi_{\mathbf{k}\downarrow}
	\rangle \nonumber \\
	&= \frac{1}{L_xL_y} \sum_{\mathbf{k} \in \mathrm{MBZ}} \langle
	\hat \Psi_{\mathbf{k}\uparrow}^\dagger \tau^x \hat \Psi_{\mathbf{k}\uparrow}
	\rangle \nonumber \\
	&= \frac{1}{2L_xL_y} \sum_{\mathbf{k} \in \mathrm{MBZ}} \frac{\Re\{\tilde{\Delta}_\mathbf{k}\}}{\tilde{E}_\mathbf{k}} \left[
	f\left(
	-\tilde{E}_\mathbf{k};\beta,\tilde{\mu}
	\right) - f\left(
	\tilde{E}_\mathbf{k};\beta,\tilde{\mu}
	\right)
	\right]
	\label{eq:af-renormalized-self-consistent-equation-m}
\end{align}
In the second passage $\langle \hat \Psi_{\mathbf{k}\uparrow}^\dagger \tau^x \hat \Psi_{\mathbf{k}\uparrow} \rangle = - \langle \hat \Psi_{\mathbf{k}\downarrow}^\dagger \tau^x \hat \Psi_{\mathbf{k}\downarrow} \rangle$ has been used. In the third passage, relations \eqref{eq:af-bogoliubov-angle-relations} were inserted. The algorithm sketched in Sec.~\ref{appsubsec:hartree-fock-algorithm} remains essentially identical, with the \textit{caveat} of defining three HF parameters, running for each a convergence analysis.

\section{Stability of the AF phase}

The discussion above is formally correct and self-consistent, thus a HF algorithm can be sketched to extract the HFPs. However we cannot proceed without considering the instability effects of Sec.~\ref{appsubsec:instability-af-commensurate}. As is known (and we would have liked to notice earlier) the AF-SSB at wavevector $\bm{\pi}$ we are studying here, responsible of halving the BZ down to the MBZ, can produce a stable phase on pure Hubbard lattices only at half filling \cite{singh1990collective}. For the EHM the situation is even worse than that: repeating the simple Linear Reponse Theory (LRT) computations of Sec.~\ref{appsubsec:instability-af-commensurate} and mimicking Eq.~\eqref{appeq:af-commensurate-proper-response}, we get a proper response function
\begin{equation}\label{eq:af-commensurate-proper-response}
	\tilde{\chi}_{\hat{\mathbf{S}}^- \hat{\mathbf{S}}^+}(\mathbf{k},\omega) = \frac{\chi_{\hat{\mathbf{S}}^- \hat{\mathbf{S}}^+}(\mathbf{k},\omega)}{1 - \left[ 
		U {\color{tabred} - 2V (\cos k_x + \cos k_y)}
	\right] \chi_{\hat{\mathbf{S}}^- \hat{\mathbf{S}}^+}(\mathbf{k},\omega)}
\end{equation}
{\color{tabred}[Double check the red part and conclude this paragraph...]}.

As is well documented in literature \cite{wietek2021stripes}, AF can establish in a wide range of spatial structures with particular shapes, for example stripes. In order to extract the precise AF phase, we should
\begin{enumerate}
	\item Choose an initializer wavevector, say $\mathbf{k}=\bm{\pi}$;
	\item Solve self-consistently for the HFPs at $\mathbf{k}$;
	\item Compute the free energy $F$ as a functional of the wavevector around $\mathbf{k}$;
	\item Look for gradient descent of the free energy, $\grad F < 0$;
	\item If the gradient descent is found, update $\mathbf{k} \to \mathbf{k} + \delta \mathbf{k}$ and repeat from step 2, otherwise halt.
\end{enumerate}
Of course the sketched procedure can lead to a lot of complications, first the fact that at incommensurate wavevector $\mathbf{k} \neq \bm{\pi}$ the unit cell is not as simple as a pair of sites, but can become much larger requiring more refined computational solutions.

In this text we will show the self consistent result for AF-SSB at wavevector $\bm{\pi}$ even at finite doping as a first step in the above sketched procedure. The presented results are self-consistent, but unfortunately deeply unstable and unphysical at finite doping, a situation where an incommensurate and insulating AF phase is preferred to this commensurate metallic counterpart. Nevertheless we present them anyway.

\section{HF algorithm, computational strategy and results}

This section is devoted to the discussion of the self-consistent results we obtained running an iterative HF algorithm in the parameters space. As said earlier, not all presented results are in fact stable; however we decided to show them as sketch of the first step in a general self-consistent algorithm in search of the stablest AF structure.

\subsection{Preliminary symmetry considerations}

\begin{figure}
	\centering
	\begin{tikzpicture}
	\begin{axis}[
			axis lines=center,
			grid=both,
			grid style={color=lightgray,dashed},
			%
			xlabel={$k_x$},
			xlabel style={right},
			ylabel={$k_y$},
			ylabel style={above},
			xtick={-1},
			xticklabel={$-\pi$},
			xticklabel style={below left},
			extra x ticks={1},
			extra x tick label={$\pi$},
			extra x tick style={xticklabel style={below right}},
			ytick={-1},
			yticklabel={$-\pi$},
			yticklabel style={below left},
			extra y ticks={1},
			extra y tick label={$\pi$},
			extra y tick style={yticklabel style={above left}},
			xmin=-1.32, xmax=1.32,
			ymin=-1.32, ymax=1.32,
			%
			width=0.5\textwidth,
			height=0.5\textwidth
		]
							
        % Down paths
	    \addplot [color=lightgray,name path=D1,domain=-1:0]
	        {abs(x)-1};
	    \addplot [color=lightgray,name path=D2,domain=0:1]
	        {abs(x)-1};
	        
	    % Up paths
	    \addplot [color=lightgray,name path=U1,domain=-1:0] 
	        {-abs(x)+1};
	    \addplot [color=tabred,name path=U2,domain=0:1,line width=1] 
	        {-abs(x)+1};
	     
	    % Zero path
	    \addplot[name path=Z,domain=0:1,opacity=0]
	        {0};
	        
	    % Fill between, nodes and points
	    \addplot[color=tabgreen,opacity=0.25] 
	        fill between [of=U2 and Z];
	    \addplot[color=lightgray,opacity=0.25] 
	        fill between [of=U1 and D1];
	    \addplot[color=lightgray,opacity=0.25] 
	        fill between [of=Z and D2];
	    \draw[color=tabgreen,line width=1]
	        (axis cs:0,0) -- (axis cs:0,1);
	    \fill[color=tabblue]
			(axis cs:0,0) circle (1.5pt)
			(axis cs:0,1) circle (1.5pt);
			
		\node[anchor=north west]
		    at (-1,1) {BZ};
		\node[color=gray,anchor=north west]
		    at (-0.5,0.5) {MBZ};
	\end{axis}
	
	\node
		at (9,3) {
			\begin{tabular}{c c}
				\textbf{Color code} & $w(\mathbf{k})$ \\
					\midrule
				White & 0 \\
				{\color{gray}Gray} & 0 \\
				{\color{tabblue}Blue} & 1 \\
				{\color{tabred}Red} & 2 \\
				{\color{tabgreen}Green} & 4 \\
				\bottomrule
			\end{tabular}
		};
\end{tikzpicture}

	\caption{Graphic rendition of $s^*$-wave symmetry over the MBZ employed to optimize calculations. The right-side table indicates, for any point in BZ, the relative weight assigned to calculation.}
	\label{fig:af-mbz-s*-optimization}
\end{figure}

A very important feature when dealing with optimization is recognizing the complete $s^*$ symmetry of the model, and the HFP parameters with relative self-consistency equations. Consider Eqns.~\eqref{eq:af-renormalized-self-consistent-equation-m}, \eqref{eq:af-renormalized-self-consistent-equation-w0} and \eqref{eq:af-renormalized-self-consistent-equation-wpi}: in all of three the term inside the sum exhibits $s^*$-wave symmetry and is summed over the MBZ. Then there is no need of sweeping the entirety of the MBZ, it suffices to sweep just one fourth and just multiply coherently the result.

A little care is necessary when dealing with borders, due to MBZ nesting. Consider in fact the MBZ rhombus, sketched in gray in Fig.~\ref{fig:af-mbz-s*-optimization}: due to periodicity, two of its four boundaries must be excluded from computation in order to avoid redundancy. Let those be the lower boundaries,
\[
	k_y = \abs{k_x} - \pi
\]
Due to periodicity, the remaining upper boundaries lead to identical results, thus we may compute just one of them and multiply the result by $2$. In doing this, we need to avoid the edges: due to nesting of the MBZ, the three points
\[
	(0,\pi) \qquad (-\pi,0) \qquad (\pi,0)
\]
are the same point. Thus these need to be considered once. Looking to the bulk of the rhombus, due to $s^*$ periodicity it suffices to integrate over one quarter (considering just one internal border) and multiply by $4$. In doing this, the origin $(0,0)$ must be counted only once. Fig.~\ref{fig:af-mbz-s*-optimization} summarizes this argument: if we assign to any given point $\mathbf{k}\in\mathrm{BZ}$ a weight function $w(\mathbf{k})$ defined as in figure, in particular avoiding computations over white and gray points, we get an optimized integral over the entire MBZ saving \textit{circa} $75\%$ of runtime.

\subsection{Results of the HF algorithm at generic doping}

Within this section we will ignore stability considerations. The first set of simulations was run keeping the local repulsion fixed at a \textit{not-so-strong} coupling value, $U/t=4$, and letting $V$ vary up to a comparable value, $0 \le V/t \le 3$, for various fillings. The temperature is kept to a finite large value $\beta=100$ to avoid Fermi surface discontinuities, while lattice size is kept to a reasonably high value $L_x=L_y=256$ to suppress finite-size effects while keeping runtime low enough. The HF has been set with the following parameters:
\begin{lstlisting}[language=julia]
p::Int64 = 100						# Maximum number of iterations
dv::Dict{String,Float64} = Dict([	# Relative tolerance on each HFP
	"m" => 1e-4,
	"w0" => 1e-4,
	"wp" => 1e-4
])
dn::Float64 = 1e-2					# Relative tolerance on density
g::Float64 = 0.5					# Mixing parameter
\end{lstlisting}

\paragraph{First HFP: $m$ (magnetization).}

\begin{figure}
	\centering
	\subfloat[Non-local attraction $V/t$ depencence.]{
		\includegraphics[height=0.3\textwidth]{../Project/EHM-HartreeFock/analysis/Phase=AF/scan/Setup=B[256]/PlotOrderParameter/xVar=V/pVar=δ/m_t=1.0_U=4.0_Lx=256.0_β=100.0.pdf}
		\label{subfig:mV-scan-beta=100}
	}
	\subfloat[Doping $\delta = n-0.5$ dependence.]{
		\includegraphics[height=0.3\textwidth]{../Project/EHM-HartreeFock/analysis/Phase=AF/scan/Setup=B[256]/PlotOrderParameter/xVar=δ/pVar=V/m_t=1.0_U=4.0_Lx=256.0_β=100.0.pdf}
		\label{subfig:md-scan-beta=100}
	}	
	\caption{Plots of the magnetization $m$ in the antiferromagnetic phase as a function of both the non-local attraction $V/t$ (Fig.~\ref{subfig:mV-scan-beta=100}) and the doping $\delta = n-1/2$ (Fig.~\ref{subfig:md-scan-beta=100}), at fixed local repulsion $U/t=4$. As discussed in text, actually only $\delta=0$ simulations are stable.}
	\label{fig:mdV-scan-beta=100}
\end{figure}

Consider first Fig.~\ref{subfig:mV-scan-beta=100}. As is to be expected from Eq.~\eqref{eq:af-renormalized-self-consistent-equation-m}, being it dependent on
\[
	\Re\{\tilde{\Delta}_\mathbf{k}\} =
	m(U + 2zV)
	\qq{with $z=4$}
\]
in this renormalized antiferromagnetic phase the non-local attraction acts as a magnetization boost, essentially reproducing the same behavior of $m$ with $U$ for the conventional Hubbard model plotted in Fig.~\ref{appfig:mdU-beta=100}. The non-local attraction also enlarges the AF phase when considering various dopings, as is seen in Fig.~\ref{fig:mdV-scan-beta=100}. The magnetized region is largely extended with respect to the low-doped segment that magnetizes at $V=0$. These effects are not particularly interesting or surprising. However, something more interesting arises when looking to Fig.~\ref{fig:mUVd-heatmaps-beta=100}, a couple of heatmaps in $UV$ and $V\delta$ planes obtained respectively at $\delta=0.2$ and $U=4.0$, with $L_x$ halved down to $128$ for computational reasons. Fig.~\ref{fig:mUVd-heatmaps-beta=100} expresses again the boost to magnetization given by $V$, with a magnetized region separated by a linear boundary approximately tilted as $-1/2z$, coherently with the renormalization of $\Re\{\tilde{\Delta}_\mathbf{k}\}$. Fig.~\ref{subfig:mVd-heatmap-beta=100} in particular is interesting: a sort of shallow ``phase boundary'' appears to follow a regular sub-linear curve.

\begin{figure}
	\centering
	\subfloat[Magnetization in the $UV$ plane.]{
		\includegraphics[height=0.3\textwidth]{../Project/EHM-HartreeFock/analysis/Phase=AF/heatmap/Setup=A[128]/Heatmaps/xVar=U_yVar=V/m_t=1.0_Lx=128.0_δ=0.2_β=100.0.pdf}
		\label{subfig:mUV-heatmap-beta=100}
	}
	\subfloat[Magnetization in the $V\delta$ plane.]{
		\includegraphics[height=0.3\textwidth]{../Project/EHM-HartreeFock/analysis/Phase=AF/heatmap/Setup=B[128]/Heatmaps/xVar=V_yVar=δ/m_t=1.0_U=4.0_Lx=128.0_β=100.0.pdf}
		\label{subfig:mVd-heatmap-beta=100}
	}	
	\caption{Heatmaps of the magnetization $m$ in the antiferromagnetic phase in the $UV$ plane (Fig.~\ref{subfig:mUV-heatmap-beta=100}) and the $V\delta$ plane (Fig.~\ref{subfig:mVd-heatmap-beta=100}).}
	\label{fig:mUVd-heatmaps-beta=100}
\end{figure}

\paragraph{Second HFP: $w^{(\mathbf{0})}$ (hopping renormalization coefficient).}

\begin{figure}
	\centering
	\subfloat[Non-local attraction $V/t$ depencence.]{
		\includegraphics[height=0.3\textwidth]{../Project/EHM-HartreeFock/analysis/Phase=AF/scan/Setup=B[256]/PlotOrderParameter/xVar=V/pVar=δ/w0_t=1.0_U=4.0_Lx=256.0_β=100.0.pdf}
		\label{subfig:w0V-scan-beta=100}
	}
	\subfloat[Doping $\delta = n-0.5$ dependence.]{
		\includegraphics[height=0.3\textwidth]{../Project/EHM-HartreeFock/analysis/Phase=AF/scan/Setup=B[256]/PlotOrderParameter/xVar=δ/pVar=V/w0_t=1.0_U=4.0_Lx=256.0_β=100.0.pdf}
		\label{subfig:w0d-scan-beta=100}
	}	
	\caption{Plots of the parameter $w^{(\mathbf{0})}$ in the antiferromagnetic phase as a function of both the non-local attraction $V/t$ (Fig.~\ref{subfig:w0V-scan-beta=100}) and the doping $\delta = n-1/2$ (Fig.~\ref{subfig:w0d-scan-beta=100}), at fixed local repulsion $U/t=4$.}
	\label{fig:w0dV-scan-beta=100}
\end{figure}

Moving to the second HFP, things get interesting. Fig.~\ref{fig:w0dV-scan-beta=100} is set up as Fig.~\ref{fig:mdV-scan-beta=100}, while Fig.~\ref{fig:w0UVd-heatmaps-beta=100} as Fig.~\ref{fig:mUVd-heatmaps-beta=100}. When considering the behavior of the parameter at fixed $U$, thus looking to Fig.~\ref{subfig:w0V-scan-beta=100}, a series of \textit{plateaus} later interrupted by a continuous change in derivative are present at each doping. This is clearly evident also in Fig.~\ref{subfig:mVd-heatmap-beta=100}, with the parameter staying constant when approaching from left the maximum situated at the position of the ``phase boundary'' of Fig.~\ref{subfig:mVd-heatmap-beta=100}. The fact that there exists a geometric region of the $V\delta$ plane where are located both the phase transition and the maximum of this parameter, which controls hopping renormalization, is particularly interesting, suggesting the possibility of finding a maximally localized system by following the path extrapolated by maximizing $w^{(\mathbf{0})}$ around the phase transition.

\begin{figure}
	\centering
	\subfloat[Magnetization in the $UV$ plane.]{
		\includegraphics[height=0.3\textwidth]{../Project/EHM-HartreeFock/analysis/Phase=AF/heatmap/Setup=A[128]/Heatmaps/xVar=U_yVar=V/w0_t=1.0_Lx=128.0_δ=0.2_β=100.0.pdf}
		\label{subfig:w0UV-heatmap-beta=100}
	}
	\subfloat[Magnetization in the $V\delta$ plane.]{
		\includegraphics[height=0.3\textwidth]{../Project/EHM-HartreeFock/analysis/Phase=AF/heatmap/Setup=B[128]/Heatmaps/xVar=V_yVar=δ/w0_t=1.0_U=4.0_Lx=128.0_β=100.0.pdf}
		\label{subfig:w0Vd-heatmap-beta=100}
	}	
	\caption{Heatmaps of the parameter $w^{(\mathbf{0})}$ in the antiferromagnetic phase in the $UV$ plane (Fig.~\ref{subfig:w0UV-heatmap-beta=100}) and the $V\delta$ plane (Fig.~\ref{subfig:w0Vd-heatmap-beta=100}).}
	\label{fig:w0UVd-heatmaps-beta=100}
\end{figure}

Interestingly enough, the highly doped region, $\delta>0.4$, presents seemingly no dependency on $V$ in Fig.~\ref{subfig:w0Vd-heatmap-beta=100} (at least for thus value of $U$), as is also evident from the asymptotic behavior at increasing $V$ visible in Fig.~\ref{subfig:w0Vd-heatmap-beta=100}. Moreover, Fig.~\ref{subfig:mUV-heatmap-beta=100} and \ref{subfig:w0UV-heatmap-beta=100} appear almost reciprocal: where one parameter grows, the other is suppressed. In terms of hopping renormalization, this tells us that antiferromagnetic ordering tends to maintain $t$ non-renormalized.

\begin{figure}
	\centering
	\subfloat[Effective hopping in the $UV$ plane.]{
		\includegraphics[height=0.3\textwidth]{../Project/EHM-HartreeFock/analysis/Phase=AF/RMPs/Setup=A[128]/RMPs/xVar=U_yVar=V/t_tilde_t=1.0_Lx=128.0_δ=0.2_β=100.0.pdf}
		\label{subfig:tUV-RMP-beta=100}
	}
	\subfloat[Effective hopping in the $V\delta$ plane.]{
		\includegraphics[height=0.3\textwidth]{../Project/EHM-HartreeFock/analysis/Phase=AF/RMPs/Setup=B[128]/RMPs/xVar=V_yVar=δ/t_tilde_t=1.0_U=4.0_Lx=128.0_β=100.0.pdf}
		\label{subfig:tVd-RMP-beta=100}
	}	
	\caption{Plots of the effective hopping $\tilde{t}$ in the antiferromagnetic phase in the $UV$ plane (Fig.~\ref{subfig:tUV-RMP-beta=100}) and the $V\delta$ plane (Fig.~\ref{subfig:tVd-RMP-beta=100}).}
	\label{fig:tUVd-RMPs-beta=100}
\end{figure}

Fig.~\ref{fig:tUVd-RMPs-beta=100} reports in its subplots the renormalized hopping as a function of the model parameters, in the same setups as in the aforementioned heatmaps. As is depicted, the NN interaction $V$ tends to reduce effective hopping amplitude down to a significant fraction of the starting value: $t$ is damped of a $30\%$ factor or so. As is evident from Fig.~\ref{subfig:tVd-RMP-beta=100}, for a small local repulsion $U=4$ the value of $\tilde{t}$ lowers as the doping grows, reaching a maximum approximately on the position of the maxima of $w^{(\mathbf{0})}$ in Fig.~\ref{subfig:w0Vd-heatmap-beta=100}, which was expected considering Eq.~\eqref{eq:af-fock-hop-ss-renormalization}.

\paragraph{Third HFP: $w^{(\bm{\pi})}$ (imaginary gap coefficient).}

Unsurprisingly, the third HFP turns out to be essentially zero in all parameters space, considering also numerical fluctuations. This was expected: when deriving the MF solution to the model within AF phase, we essentially reduced it to the already known solution of Sec.~\ref{appsec:antiferromagnetic-phase}. The gap there was a real parameter, thus by purely renormalizing the parameters obtained within such solution there is no need for the present gap to acquire a non-zero imaginary part.

\subsection{Role of hopping renormalization: switching off $w^{(\mathbf{0})}$}
\todo
