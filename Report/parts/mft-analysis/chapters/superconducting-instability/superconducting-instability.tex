\def\GraphicsFolder{parts/mft-analysis/chapters/superconducting-instability/pictures}
\chapter{Superconducting instability}\label{chap:mft-su-instability}

This chapter is devoted to studying the superconducting phase of the system. The only symmetry we assume to break is the $\mathrm{U}^\mathrm{c}(1)$ charge symmetry, thus allowing for superconducting fluctuations. As is described thoroughly in Sec.~\ref{subsec:fock-renormalization-hopping-amplitude}, the hopping amplitude is renormalized because of the non-local attraction. The symmetry structure of the pairing mechanism determines the contributing Cooper fluctuations: for $s$-wave and $d$-wave superconductivity, only the o.s. Cooper term contributes; for $p_\ell$-wave superconductivity, the s.s. term contributes as well. In the following sections, a derivation containing both Cooper terms is proposed.

{\color{tabred}[
	To be continued: separate singlet and triplet pairing channels, and describe them separately by the means of four-components Nambu spinors. Use selection rules to set $\Delta^{(p_\ell)}=0$ in the singlet channel, in order to justify results obtained by a pure space-even simulation containing just the o.s. terms.
	]}

\section{Cooper fluctuations in the EHM}

Let us start once again from the general EHM of Eq.~\eqref{eq:extended-hubbard-model},
\[
	\hat H =
	\underbrace{
		-t \sum_{\langle ij \rangle} \sum_\sigma \hat c_{i\sigma}^\dagger \hat c_{j\sigma}
	}_{\hat H_t} \underbrace{
		+U \sum_i \hat c_{i\uparrow}^\dagger \hat c_{i\downarrow}^\dagger \hat c_{i\downarrow} \hat c_{i\uparrow}
		\vphantom{
			\sum_{\langle ij \rangle}
		}
	}_{\hat H_U}
	\underbrace{
		- V \sum_{\langle ij \rangle} \sum_{\sigma \sigma'} \hat c_{i\sigma}^\dagger \hat c_{j\sigma'}^\dagger \hat c_{j\sigma'} \hat c_{i\sigma}
	}_{\hat H_V}
\]
As is discussed in Sec.~\ref{sec:mft-discussion-non-local-source}, when applying Wick's theorem the resulting terms break the natural symmetries of the model. The superconducting symmetry we study breaks just the $\mathrm{U}^\mathrm{c}(1)$ charge conservation. Now, when dealing with superconducting Cooper pairing we need to account also for the spatial structure of the Cooper pair itself. Consider the generic Cooper fluctuation
\[
	\big\langle
		\hat c_{i\sigma}^\dagger \hat c_{j\sigma'}^\dagger
	\big\rangle
	\qq{with $i,j$ NN}
\]
From basic Quantum Mechanics we know the summation rules of the spin algebra $\mathfrak{su}(2)$,
\[
	\frac 1 2 \otimes \frac 1 2 = 0 \oplus 1
\]
The two pairing channels are, at this level, the singlet channel associated to total spin $0$ and the triplet channel associated to total spin $1$. If we impose a specific spatial symmetry on the hamiltonian the ground state wavefunction will follow naturally, and the pairing channel will be the one providing a total anti-symmetry to the full wavefunction. This gives a selection rule over the relevant pairings: if we work with space symmetric structures -- say, $s^*$-wave or $d$-wave -- the pairing will happen in singlet channel, allowing us to eliminate the triplet pairing. This concept is summarized in Tab.~\ref{tab:su-pairings-symmetry-structures}.

\begin{table}
	\centering
	\begin{tabular}{c c c}
		\textbf{Spatial structure} & \textbf{Pairing channel} & \textbf{Relevant pairing} \\
		\midrule
		Symmetric wave function & Singlet pairing & Just $\displaystyle \big\langle
		\hat c_{i\sigma}^\dagger \hat c_{j\overline{\sigma}}^\dagger
		\big\rangle$ \\
		Anti-symmetric wave function & Triplet pairing & Both $\displaystyle \big\langle
		\hat c_{i\sigma}^\dagger \hat c_{j\overline{\sigma}}^\dagger
		\big\rangle$ and $\displaystyle \big\langle
		\hat c_{i\sigma}^\dagger \hat c_{j\sigma}^\dagger
		\big\rangle$
	\end{tabular}
	\caption{Relation of the Cooper pairing channell with the wavefunction spatial symmetry (intended as the inversion $(x,y) \to (-x,-y)$).}
	\label{tab:su-pairings-symmetry-structures}
\end{table}

{\color{tabred}[ Add: Cooper pairing considerations in the pure Hubbard model. ]}

\section{Cooper fluctuations in the opposite-spin sector}

This section deals with Cooper fluctuations induced by the o.s. part of the non-local hamiltonian --referring to the notation of Eq.~\eqref{eq:ehm-non-local-ss-os-terms}-- somewhat the simplest form of Cooper pairing. This sector contributes both to singlet pairing and triplet pairing. Considering Cooper fluctuations in the singlet channel, we need to break $\mathrm{U}^\mathrm{c}(1)$ symmetry imposing space inversion symmetry, while for the triplet anti-symmetry is required. Let us now break down the MFT discussion for the local and non-local interactions.

\paragraph{Local interaction $U$.}

Consider first the local part,
\[
	\hat H_U = U \sum_{i \in \mathcal{S}} \hat n_{i\uparrow} \hat n_{i\downarrow} \simeq U \sum_{i \in \mathcal{S}} \left[
		\big\langle
			\hat c_{i\uparrow}^\dagger \hat c_{i\downarrow}^\dagger
		\big\rangle
		\hat c_{i\downarrow} \hat c_{i\uparrow} + \hat c_{i\uparrow}^\dagger \hat c_{i\downarrow}^\dagger \big\langle
			\hat c_{i\downarrow} \hat c_{i\uparrow}
		\big\rangle
	\right]
\]
and use the result of Eq.~\eqref{eq:reciprocal-space-local-interaction-explicit},
\[
	\hat H_U \simeq \frac{U}{L_x L_y}
	\sum_{\mathbf{K}, \mathbf{k}, \mathbf{k}'}
	\left[
		\big\langle
			\hat c_{\mathbf{K}+\mathbf{k} \uparrow}^\dagger \hat c_{\mathbf{K}-\mathbf{k} \downarrow}^\dagger
		\big\rangle
		\hat c_{\mathbf{K}-\mathbf{k}' \downarrow} \hat c_{\mathbf{K}+\mathbf{k}' \uparrow} +
		\hat c_{\mathbf{K}+\mathbf{k} \uparrow}^\dagger \hat c_{\mathbf{K}-\mathbf{k} \downarrow}^\dagger
		\big\langle
			\hat c_{\mathbf{K}-\mathbf{k}' \downarrow} \hat c_{\mathbf{K}+\mathbf{k}' \uparrow}
		\big\rangle
	\right]
\]
We are not breaking translational invariance, thus only Cooper fluctuations with net zero total momentum are allowed. This means only $\mathbf{K}=\mathbf{0}$ contributes. Define the pairing operator 
\[
	\hat \phi_\mathbf{k} \equiv \hat c_{-\mathbf{k}\downarrow} \hat c_{\mathbf{k} \uparrow}
	\qquad
	\hat \phi_\mathbf{k}^\dagger \equiv \hat c_{\mathbf{k} \uparrow}^\dagger \hat c_{-\mathbf{k}\downarrow}^\dagger
\]
Then the non local repulsion reduces to the ordinary BCS-like interaction,
\begin{equation}\label{eq:su-ugap-mft}
	\hat H_U \simeq \sum_\mathbf{k} \left[
		\mathcal{U}_\mathbf{k} \hat \phi_\mathbf{k} + \mathcal{U}_\mathbf{k}^* \hat \phi_\mathbf{k}^\dagger
	\right]
\end{equation}
where the MFT parameter $\mathcal{U}_\mathbf{k}$ must satisfy the self-consistency equation
\begin{equation}\label{eq:su-ugap-self-consistency}
	\mathcal{U}_\mathbf{k} \equiv \frac{U}{L_x L_y} \sum_\mathbf{k} \big\langle 
		\hat \phi_\mathbf{k}^\dagger
	\big\rangle
\end{equation}
Note that $\mathcal{U}_\mathbf{k}$ is actually momentum independent. This is due to the fact that the repulsion is completely localized.

\paragraph{Non-local interaction $V$.}

The non-local attraction in the opposite-spin sector of $\hat H_V$ is given by
\[
	\hat H_V^{(\mathrm{o.s.})} = -V \sum_{\ev{ij}} \sum_\sigma \hat n_{i\sigma} \hat n_{j\overline{\sigma}}
\]
which, using Eq.~\eqref{eq:ehm-mft-nonlocal-opposite-spin} and performing Wick's contractions, reduces to:
\[
\begin{aligned}
	\hat H_V^{(\mathrm{o.s.})} &= -V \sum_{i \in \mathcal{S}} \sum_{\ell = x,y} \sum_{\delta = \pm \delta_\ell}
	\hat n_{i\uparrow} \hat n_{i+\delta \downarrow} \\
	&\simeq - V \sum_{i \in \mathcal{S}} \sum_{\ell = x,y} \sum_{\delta = \pm \delta_\ell} \left[
		\langle 
			\hat c_{i\uparrow}^\dagger \hat c_{i + \delta \downarrow}^\dagger
		\rangle
		\hat c_{i + \delta \downarrow} \hat c_{i\uparrow} 
		+
		\hat c_{i\uparrow}^\dagger \hat c_{i + \delta \downarrow}^\dagger
		\langle
			\hat c_{i + \delta \downarrow} \hat c_{i\uparrow} 
		\rangle
	\right]
\end{aligned}
\]
Using Eq.~\eqref{eq:reciprocal-space-non-local-interaction-explicit} we can move to reciprocal space,
\begin{multline*}
	\hat H_V^{(\mathrm{o.s.})} \simeq -\frac{V}{L_x L_y}
	\sum_{\mathbf{K}, \mathbf{k}, \mathbf{k}'} \left[
		\cos \left(
			\delta k_x
		\right)	+ \cos \left(
			\delta k_y
		\right)	
	\right]	\\
	\times \left[ 
		\big\langle
			\hat c_{\mathbf{K}+\mathbf{k} \uparrow}^\dagger \hat c_{\mathbf{K}-\mathbf{k} \downarrow}^\dagger
		\big\rangle 
		\hat c_{\mathbf{K}-\mathbf{k}' \downarrow} \hat c_{\mathbf{K}+\mathbf{k}'\uparrow} 
		+
		\hat c_{\mathbf{K}+\mathbf{k} \uparrow}^\dagger \hat c_{\mathbf{K}-\mathbf{k} \downarrow}^\dagger
		\big\langle 
			\hat c_{\mathbf{K}-\mathbf{k}' \downarrow} \hat c_{\mathbf{K}+\mathbf{k}'\uparrow}
		\big\rangle
	\right]
\end{multline*}
Identical considerations as above hold, and just the $\mathbf{K}=\mathbf{0}$ term contributes. We have finally
\[
\begin{aligned}
	\hat H_V^{(\mathrm{o.s.})} \simeq - \sum_{\mathbf{k}, \mathbf{k}'}
	V_{\mathbf{k}\mathbf{k}'} \left[
	\langle 
	\hat \phi_\mathbf{k}^\dagger
	\rangle \hat \phi_{\mathbf{k}'} + \langle 
	\hat \phi_\mathbf{k}
	\rangle \hat \phi_{\mathbf{k}'}^\dagger
	\right]	
\end{aligned}
\]
where the two-body potential was defined
\[
	V_{\mathbf{k}\mathbf{k}'} = \frac{V}{L_x L_y} \left[
		\cos \left(
			\delta k_x
		\right)	+ \cos \left(
			\delta k_y
		\right)	
	\right]
\]
Making use of the decomposition of Eq.~\eqref{eq:coscos-symmetries-decomposition}, the two-body potential becomes
\[
\begin{aligned}
	V_{\mathbf{k}\mathbf{k}'} &= \frac{V}{2L_xL_y} \sum_\gamma \varphi_\mathbf{k}^{(\gamma)} \varphi_{\mathbf{k}'}^{(\gamma)*} \\
	&= \sum_\gamma V^{(\gamma)} \varphi_\mathbf{k}^{(\gamma)} \varphi_{\mathbf{k}'}^{(\gamma)*} \\
\end{aligned}
\]
being $\gamma \in \lbrace s^*, p_x, p_y, d_{x^2-y^2} \rbrace$ and $\varphi_\mathbf{k}^{(\gamma)}$ the reciprocal-space expressions for the form factors of Tab.~\ref{tab:x-wave-real-factors}, listed explicitly in Tab.~\ref{tab:x-wave-reciprocal-factors}, and $V_{\mathbf{k}\mathbf{k}'}^{(\gamma)}$ the symmetry-resolved components of the non-local attraction. Then the two-body potential has been decomposed in its planar symmetry components, each of which will naturally couple only to identically structured parameters in the full hamiltonian.

Define now the non-local gap function
\begin{equation}\label{eq:su-ugap-self-consistency-intermediate}
	\mathcal{V}_\mathbf{k} \equiv \sum_{\mathbf{k}'}
	V_{\mathbf{k}\mathbf{k}'}
	\langle
	\hat \phi_{\mathbf{k}'}^\dagger
	\rangle
\end{equation}
one gets immediately
\begin{equation}\label{eq:su-vgap-mft}
	\hat H_V \simeq -\sum_\mathbf{k} \left[
	\mathcal{V}_\mathbf{k} \hat \phi_\mathbf{k} + \mathcal{V}_\mathbf{k}^* \hat \phi_\mathbf{k}^\dagger
	\right]	
\end{equation}
To assume symmetry is broken in a specific symmetry channel $\gamma$ means precisely to assume $\langle \hat \phi_\mathbf{k} \rangle \propto \varphi_\mathbf{k}^{(\gamma)}$. Of course, in Eq.~\eqref{eq:su-ugap-self-consistency-intermediate} only the $\gamma$ component of the potential survives, implying the gap function acquires the same symmetry,
\[
	\mathcal{V}_\mathbf{k} \propto \sum_{\mathbf{k}'} \varphi_\mathbf{k}^{(\gamma)} \varphi_{\mathbf{k}'}^{(\gamma)*}
	\varphi_{\mathbf{k}'}^{(\gamma)} \propto \varphi_\mathbf{k}^{(\gamma)}
\]
where orthonormality of the $\varphi_\mathbf{k}^{(\gamma)}$ functions of Tab.~\ref{tab:x-wave-reciprocal-factors} was used. Thus, assuming to have superconductivity in a given sector $\gamma$, the self consistency equation reads
\begin{equation}\label{eq:su-vgap-self-consistency}
	\forall \gamma \in \lbrace s^*, p_x, p_y, d_{x^2-y^2} \rbrace
	\qquad
	\mathcal{V}_\mathbf{k}\big|_\gamma \equiv \frac{V}{2L_x L_y} \sum_\mathbf{k}
	\varphi_\mathbf{k}^{(\gamma)}
	\big\langle 
		\hat \phi_\mathbf{k}^\dagger
	\big\rangle
\end{equation}
Now we merge the two interaction in a single gap function.

\subsection{Full gap function and self-consistency equations}

Define the full gap function as the sum of both contributions,
\begin{equation}\label{eq:su-os-gap-function-definition}
	\Delta_\mathbf{k} \equiv \mathcal{V}_\mathbf{k} - \mathcal{U}_\mathbf{k}
\end{equation}
The full self-consistency equation is given by the simple combination of Eqns.~\eqref{eq:su-ugap-self-consistency} and \eqref{eq:su-vgap-self-consistency},
\begin{equation}\label{eq:self-consistency-equation}
	\Delta_\mathbf{k} \equiv \sum_{\mathbf{k}'}
	\left[
		V^{(s)} +
		V_{\mathbf{k}\mathbf{k}'}
	\right]
	\langle
		\hat \phi_{\mathbf{k}'}^\dagger
	\rangle
	\qq{with}
	V^{(s)} = - \frac{U}{2L_xL_y}
\end{equation}
The gap function decomposes in symmetry channels as well,
\[
\Delta_\mathbf{k} = \sum_\gamma \Delta^{(\gamma)} \varphi_\mathbf{k}^{(\gamma)}
\]
If SC arises in a specific symmetry channel, $\Delta_\mathbf{k}$ will show the same symmetry. It follows, due to orthonormality and using Eq.~\eqref{eq:self-consistency-equation},
\begin{align}
	\Delta^{(\gamma)} &= \frac{1}{L_xL_y} \sum_{\mathbf{k}} \varphi_\mathbf{k}^{(\gamma)*} \Delta_\mathbf{k} \nonumber \\
	&= \frac{1}{L_xL_y} \sum_{\mathbf{k}} \varphi_\mathbf{k}^{(\gamma)*} \sum_{\mathbf{k}'}
	\left[
	V^{(s)} +
	V_{\mathbf{k}\mathbf{k}'}
	\right]
	\langle
	\hat \phi_{\mathbf{k}'}^\dagger
	\rangle \nonumber \\
	&= \frac{1}{L_xL_y} \sum_{\mathbf{k}} \varphi_\mathbf{k}^{(\gamma)*} \sum_{\mathbf{k}'\gamma'}
	V^{(\gamma')} \varphi_\mathbf{k}^{(\gamma')} \varphi_{\mathbf{k}'}^{(\gamma')*}
	\langle
	\hat \phi_{\mathbf{k}'}^\dagger
	\rangle \nonumber \\
	&= V^{(\gamma)} \sum_{\mathbf{k}} \varphi_\mathbf{k}^{(\gamma)*} \langle
	\hat \phi_{\mathbf{k}}^\dagger
	\rangle \label{eq:self-consistency-equation-explicit}
\end{align}
This result provides a set of self-consistency equations for each symmetry channel, listed in Tab.~\ref{tab:x-wave-self-consistency-equation}. Notice that to reconstruct self-consistently the full $s$-wave phase transition, the actual gap function is given by
\[
\Delta^{(s)} + \Delta^{(s^*)} (c_x + c_y)
\]
The $s$-wave transition is the only one equipped of both the local and the non-local parts. Within this structure, we are finally able to move to Nambu formalism.

\setlength{\extrarowheight}{1em}
\begin{table}
	\centering
	\begin{tabular}{r E l l}
		\textbf{Structure} & \multicolumn{2}{c}{\textbf{Self-consistency equation}} & \textbf{Graph} \\
		\midrule
		$s$-wave & $\Delta^{(s)}$ & $\displaystyle -\frac{U}{2L_xL_y} \sum_{\mathbf{k}} \langle
		\hat \phi_{\mathbf{k}}^\dagger
		\rangle $ & Fig.~\ref{subfig:s-wave-correlator} \\
		Extended $s$-wave & $\Delta^{(s^*)}$ & $\displaystyle \frac{V}{L_xL_y} \sum_{\mathbf{k}} (c_x + c_y) \langle
		\hat \phi_{\mathbf{k}}^\dagger
		\rangle$ & Fig.~\ref{subfig:s*-wave-correlator} \\
		$p_x$-wave & $\Delta^{(p_x)}$ & $\displaystyle - i\sqrt{2} \frac{V}{L_xL_y} \sum_{\mathbf{k}} s_x \langle
		\hat \phi_{\mathbf{k}}^\dagger
		\rangle$ & Fig.~\ref{subfig:px-wave-correlator} \\
		$p_y$-wave & $\Delta^{(p_y)}$ & $\displaystyle -i \sqrt{2} \frac{V}{L_xL_y} \sum_{\mathbf{k}} s_y \langle
		\hat \phi_{\mathbf{k}}^\dagger
		\rangle$ & Fig.~\ref{subfig:py-wave-correlator} \\
		$d_{x^2-y^2}$-wave & $\Delta^{(d)}$ & $\displaystyle \frac{V}{L_xL_y} \sum_{\mathbf{k}} (c_x - c_y) \langle
		\hat \phi_{\mathbf{k}}^\dagger
		\rangle$ & Fig.~\ref{subfig:d-wave-correlator} 
	\end{tabular}
	\caption{Symmetry resolved self-consistency equations for the MFT parameters $\Delta^{(\gamma)}$, based on Eq.~\eqref{eq:self-consistency-equation} and \eqref{eq:self-consistency-equation-explicit}. By computing $\langle \hat \phi_{\mathbf{k}}^\dagger \rangle$, it is possible to reconstruct the various components of the gap function.}
	\label{tab:x-wave-self-consistency-equation}
\end{table}
\setlength{\extrarowheight}{0em}

\section{Superconducting solution in the translationally invariant sector}

As a first approach to anisotropic SC in the Hubbard model, let us discuss the superconducting solutions in the simple scenario where crystal translational invariance is preserved and thus the system can be treated by the means of common BCS. As will be discussed, the non-local attraction acts as a source of SC in all symmetry sectors.

\subsection{Nambu formalism and Bogoliubov transform}\label{subsec:nambu-formalism-mean-field-extended-hubbard}

Define the Nambu spinor\footnote{
	Notice that the spinor is here differently defined with respect to App.~\ref{appendix:mean-field-hubbard}, where because of the HF prevalence in mean-field decoupling the spinor components were homogeneously fermions creations or destructions.
} as in BCS
\[
\hat \Psi_\mathbf{k} \equiv \begin{bmatrix}
	\hat c_{\mathbf{k}\uparrow} \\
	\hat c_{-\mathbf{k}\downarrow}^\dagger
\end{bmatrix}
\]
Evidently,
\begin{equation}\label{eq:extended-hubbard-phi-psi-expressions}
	\phi_\mathbf{k} = \hat \Psi_\mathbf{k}^\dagger \begin{bmatrix}
		0 & 1 \\ 0 & 0
	\end{bmatrix} \hat \Psi_\mathbf{k}
	\qquad
	\phi_\mathbf{k}^\dagger = \hat \Psi_\mathbf{k}^\dagger \begin{bmatrix}
		0 & 0 \\ 1 & 0
	\end{bmatrix} \hat \Psi_\mathbf{k}
\end{equation}
The full hamiltonian is then given by:
\begin{equation}\label{eq:extended-hubbard-hamiltonian-nambu-bogoliubov}
	\hat H = \sum_\mathbf{k} \hat \Psi_\mathbf{k} h_\mathbf{k} \hat \Psi_\mathbf{k}
	\qquad
	h_\mathbf{k} \equiv \begin{bmatrix}
		\epsilon_\mathbf{k} & - \Delta_\mathbf{k}^* \\
		- \Delta_\mathbf{k} & - \epsilon_\mathbf{k}
	\end{bmatrix}
\end{equation}
Let $\tau^\alpha$ for $\alpha = x,y,z$ be the Pauli matrices. Define:
\[
\hat s_\mathbf{k}^\alpha \equiv \hat \Psi_\mathbf{k}^\dagger \tau^\alpha \hat \Psi_\mathbf{k}
\qq{for}
\alpha = x,y,z
\]
As can be shown easily, these operators realize spin-$1/2$ algebra. $\hat H$ represents an ensemble of $L_x L_y$ independent spins subject to pseudo-magnetic fields. Note that, differently form App.~\ref{app:mean-field-hubbard} where the chemical potential is inserted later (because in Nambu formalism it accounts for a diagonal term) here the chemical potential is part of the $z$ component of the pseudo-magnetic field, since
\begin{align}
	\hat n_{\mathbf{k}\uparrow} + \hat n_{-\mathbf{k}\downarrow} &= \hat c_{\mathbf{k}\uparrow}^\dagger \hat c_{\mathbf{k}\uparrow} + \hat c_{-\mathbf{k}\downarrow}^\dagger \hat c_{-\mathbf{k}\downarrow} \nonumber \\
	&= \hat c_{\mathbf{k}\uparrow}^\dagger \hat c_{\mathbf{k}\uparrow} - \hat c_{-\mathbf{k}\downarrow} \hat c_{-\mathbf{k}\downarrow}^\dagger + \mathbb{I} \nonumber \\
	&=  \hat \Psi_\mathbf{k}^\dagger \tau^z \hat \Psi_\mathbf{k} + \mathbb{I} \label{eq:number-operator-z-spin-relation}
\end{align}
and then it follows
\[
\begin{aligned}
	-\mu\hat N &= -\mu \sum_{\mathbf{k} \in \mathrm{BZ}} \left[ 
	\hat n_{\mathbf{k}\uparrow} + \hat n_{-\mathbf{k}\downarrow} 
	\right] \\
	&= -\mu \sum_{\mathbf{k} \in \mathrm{BZ}} \hat \Psi_\mathbf{k}^\dagger \tau^z \hat \Psi_\mathbf{k} -\mu L_x L_y
\end{aligned}
\]
Then, adding a term $-\mu \hat N$ to $\hat H$, apart from an irrelevant total energy increase, changes the pseudo-field whose explicit form becomes
\begin{equation} \label{eq:extended-hubbard-pseudo-magnetic-field}
	\mathbf{b}_\mathbf{k} \equiv \begin{bmatrix}
		-\Re{\Delta_\mathbf{k}} \\
		-\Im{\Delta_\mathbf{k}} \\ \epsilon_\mathbf{k} - \mu
	\end{bmatrix}
\end{equation}
This hamiltonian behaves as an ensemble of spins in local magnetic fields precisely as in Eq.~\eqref{appeq:hubbard-bogoliubov-hamiltonian-pseudofields},
\begin{equation}\label{eq:extended-hubbard-bogoliubov-hamiltonian-pseudofields}
	\hat H -\mu \hat N = \sum_{\mathbf{k} \in \mathrm{BZ}} \mathbf{b}_\mathbf{k} \cdot \hat{\mathbf{s}}_\mathbf{k}
	\qq{where}
	\hat{\mathbf{s}}_{\mathbf{k}\sigma} = \begin{bmatrix}
		\hat s_\mathbf{k}^x \\
		\hat s_\mathbf{k}^y \\
		\hat s_\mathbf{k}^z
	\end{bmatrix}
\end{equation}
Proceed as in App.~\ref{app:mean-field-hubbard} and diagonalize via a rotation,
\[
d_\mathbf{k} \equiv \begin{bmatrix}
	-E_\mathbf{k} & \\ & E_\mathbf{k}
\end{bmatrix}
\qq{being}
E_\mathbf{k} \equiv \sqrt{\xi_\mathbf{k}^2 + \abs{\Delta_\mathbf{k}}^2}
\]
and $\xi_\mathbf{k} \equiv \epsilon_\mathbf{k} - \mu$. Given the pseudoangles
\[
\tan(2\theta_\mathbf{k}) \equiv \frac{\abs{\Delta_\mathbf{k}}}{\epsilon_\mathbf{k}}
\qquad
\tan(2\zeta_\mathbf{k}) \equiv \frac{\Im{\Delta_\mathbf{k}}}{\Re{\Delta_\mathbf{k}}}
\]
the general diagonalizer will be an orthogonal rotation matrix
\begin{align}
	W_\mathbf{k} &= e^{i \left(\theta_\mathbf{k} - \frac{\pi}{2}\right) \tau^y} e^{i \zeta_\mathbf{k} \tau^z} \nonumber \\
	&= \begin{bmatrix}
		-\sin\theta_\mathbf{k}  & -\cos\theta_\mathbf{k}  \\ 
		\cos\theta_\mathbf{k}  & -\sin\theta_\mathbf{k} 
	\end{bmatrix} \begin{bmatrix}
		e^{i\zeta_\mathbf{k} } & \\ & e^{-i\zeta_\mathbf{k} }
	\end{bmatrix} \nonumber \\
	&= \begin{bmatrix}
		-\sin\theta_\mathbf{k}  e^{i\zeta_\mathbf{k} } & -\cos\theta_\mathbf{k}  e^{-i\zeta_\mathbf{k} }  \\ 
		\cos\theta_\mathbf{k}  e^{i\zeta_\mathbf{k} } & -\sin\theta_\mathbf{k}  e^{-i\zeta_\mathbf{k} } 
	\end{bmatrix} \label{eq:extended-hubbard-bogoliubov-W-diagonalizer-explicit}
\end{align}
given by a rotation of angle $\zeta_\mathbf{k}$ around the $z$ axis, to align the $x$ axis with the field projection onto the $xy$ plane, followed by a rotation around the $y$ axis to anti-align with the pseudo-field. The MFT-BCS solution is given by a degenerate Fermi gas at ground state, whose quasi-particles occupy two bands $\pm E_\mathbf{k}$ and their fermionic operators are given by
\[
\hat \gamma_\mathbf{k}^{(-)} \equiv \left[
W_\mathbf{k} \hat \Psi_\mathbf{k}
\right]_1
\qquad
\hat \gamma_\mathbf{k}^{(+)} \equiv \left[
W_\mathbf{k} \hat \Psi_\mathbf{k}
\right]_2
\]
The diagonalization operators are given by
\[
\hat\Gamma_\mathbf{k} \equiv W_\mathbf{k} \hat\Psi_\mathbf{k}
\qq{where}
\hat\Gamma_\mathbf{k} = \begin{bmatrix}
	\hat \gamma_\mathbf{k}^{(-)} \\ \hat \gamma_\mathbf{k}^{(+)}
\end{bmatrix}
\]
then, using Eq.~\eqref{appeq:finite-temperature-order-parameter-derivation-intermediate}, 
\[
\left\langle [\hat \Psi_\mathbf{k}^\dagger]_i [\hat \Psi_\mathbf{k}]_j \right\rangle = [W_\mathbf{k}]_{1 i} [W_\mathbf{k}^\dagger]_{j 1} f\left( -E_\mathbf{k}; \beta,0 \right) + [W_\mathbf{k}]_{2 i} [W_\mathbf{k}^\dagger]_{j 2} f\left( E_\mathbf{k}; \beta,0 \right)
\]
where in the Fermi-Dirac function chemical potential was set to zero, because it already was included in the diagonalized hamiltonian.
Recalling Eq.~\eqref{eq:extended-hubbard-phi-psi-expression}, it follows
\begin{align}
	\langle \phi_\mathbf{k}^\dagger \rangle &= [W_\mathbf{k}]_{11} [W_\mathbf{k}^\dagger]_{21} f\left( -E_\mathbf{k}; \beta,0 \right) + [W_\mathbf{k}]_{21} [W_\mathbf{k}^\dagger]_{22} f\left( E_\mathbf{k}; \beta,0 \right) \label{eq:pairing-dagger-expectation-algorithmic} \\
	&= \frac{1}{2} \sin \left(
	2 \theta_\mathbf{k}
	\right) e^{i 2 \zeta_\mathbf{k}} \tanh(\frac{\beta E_\mathbf{k}}{2}) \label{eq:pairing-dagger-expectation-theoretical}
\end{align}
The last passage has been obtained by computing the matrix element from the explicit form of $W_\mathbf{k}$ of Eq.~\eqref{eq:extended-hubbard-bogoliubov-W-diagonalizer-explicit} and by the simple relation
\[
\begin{aligned}
	\frac{1}{e^{-x}+1} - \frac{1}{e^x+1} &= \frac{e^x -1}{e^x +1} \\
	&= \tanh(\frac{x}{2})
\end{aligned}
\]
Eqns.~\eqref{eq:pairing-dagger-expectation-algorithmic}, \eqref{eq:pairing-dagger-expectation-theoretical} give us both the algorithmic formula (first row) and its theoretical counterpart (second row) to compute the order parameters in the HF approach at each point in $k$-space $(k_x,k_y)$. We can finally derive the BCS self-consistency equation
\begin{align}
	\Delta_\mathbf{k} &\equiv \frac{1}{2} \sum_{\mathbf{k}'}
	\left[
	V^{(s)} +
	V_{\mathbf{k}\mathbf{k}'}
	\right] \frac{\abs{\Delta_{\mathbf{k}'}}}{\sqrt{\xi_{\mathbf{k}'}^2 + \abs{\Delta_{\mathbf{k}'}}^2}} \exp{i \arctan(\frac{\Im{\Delta_{\mathbf{k}'}}}{\Re{\Delta_{\mathbf{k}'}}})} \tanh(\frac{\beta}{2} \sqrt{\xi_{\mathbf{k}'}^2 + \abs{\Delta_{\mathbf{k}'}}^2}) \nonumber \\
	&= \frac{1}{2} \sum_{\mathbf{k}'}
	\left[
	V^{(s)} +
	V_{\mathbf{k}\mathbf{k}'}
	\right] \frac{\Delta_{\mathbf{k}'}}{\sqrt{\xi_{\mathbf{k}'}^2 + \abs{\Delta_{\mathbf{k}'}}^2}} \tanh(\frac{\beta}{2} \sqrt{\xi_{\mathbf{k}'}^2 + \abs{\Delta_{\mathbf{k}'}}^2}) 
	\label{eq:self-consistency-equation-theoretical}
\end{align}
The whole point of the HF algorithm is to find an iterative solution for each symmetry channel, using the self-consistency equation projection of Tab.~\ref{tab:x-wave-self-consistency-equation}.

Notice that the $z$ component of the spin operators is related to density: using Eq.~\eqref{appeq:finite-temperature-order-parameter-derivation-intermediate},
\[
\langle
\hat\Psi_\mathbf{k}^\dagger \tau^z \hat\Psi_\mathbf{k}
\rangle = \left\langle 
[\hat \Psi_\mathbf{k}^\dagger]_1 [\hat \Psi_\mathbf{k}]_1 
\right\rangle - \left\langle 
[\hat \Psi_\mathbf{k}^\dagger]_2 [\hat \Psi_\mathbf{k}]_2
\right\rangle
\]
I proceed as previously, and from Eq.~\eqref{eq:number-operator-z-spin-relation},
\begin{align}
	\langle \hat n_{\mathbf{k}\uparrow} \rangle + \langle \hat 	n_{-\mathbf{k}\downarrow} \rangle = 1 &+ \langle \hat\Psi_\mathbf{k}^\dagger \tau^z \hat\Psi_\mathbf{k} \rangle \nonumber \\
	= 1 &+ \left(
	\abs{[W_\mathbf{k}]_{11}}^2 - \abs{[W_\mathbf{k}]_{12}}^2
	\right) f\left( -E_\mathbf{k}; \beta,0 \right) \nonumber \\
	&+ \left(
	\abs{[W_\mathbf{k}]_{21}}^2 - \abs{[W_\mathbf{k}]_{22}}^2
	\right) f\left( E_\mathbf{k}; \beta,0 \right) \label{eq:density-expectation-algorithmic} \\
	= 1 &- \cos \left(
	2 \theta_\mathbf{k}
	\right) \tanh(\frac{\beta E_\mathbf{k}}{2}) \label{eq:density-expectation-theoretical} 
\end{align}
The expectation value for the density is needed in order to extract the optimal chemical potential $\mu$ for the target density we aim to simulate at the given parametrization. This is numerically obtained by using Eq.~\eqref{eq:density-expectation-algorithmic} directly on the diagonalization matrix of $h_\mathbf{k}$.

%\subsection{A short comment on self-consistency}\label{subsec:mft-self-consitency-cooper-decomposition}
%
%The Bogoliubov fermions in spinor representation satisfy obviously $\hat \Psi_\mathbf{k} = W_\mathbf{k}^\dagger \hat \Gamma_\mathbf{k}$. Consider e.g.
%\[
%\langle
%\hat c_{\mathbf{k}\sigma}^\dagger
%\hat c_{-\mathbf{k}\sigma}^\dagger
%\rangle	
%\]
%which is a spin-symmetric anomalous Cooper pair. For simplicity, take $\sigma=\uparrow$. Expand:
%\[
%\begin{aligned}
%	\langle
%	\hat c_{\mathbf{k}\uparrow}^\dagger
%	\hat c_{-\mathbf{k}\uparrow}^\dagger
%	\rangle	&= \left\langle 
%	[\hat \Psi_\mathbf{k}^\dagger]_1 [\hat \Psi_{-\mathbf{k}}^\dagger]_1 
%	\right\rangle \\
%	&= \left\langle 
%	[W_\mathbf{k} \hat \Gamma_\mathbf{k}^\dagger]_1 [W_{-\mathbf{k}} \hat \Gamma_{-\mathbf{k}}^\dagger]_1 
%	\right\rangle
%\end{aligned}
%\]
%This expectation value is taken over the ground-state, the latter being the vacuum of $\Gamma$ fermions. Evidently the above expectation cannot assume non-zero values. Obviously the same holds for $\sigma=\downarrow$, and this argument explains why the Ferromagnetic terms of the hamiltonian decomposition do not contribute to Cooper instability. An identical argument, with the exchange
%\[
%(\sigma,\sigma) \to (\uparrow,\downarrow)
%\qq{and}
%(\mathbf{k},-\mathbf{k}) \to (\mathbf{K}+\mathbf{k},\mathbf{K}-\mathbf{k})
%\qq{with}
%\mathbf{K}\neq\mathbf{0}
%\]
%justifies why in Sec.~\ref{subsec:nambu-formalism-mean-field-extended-hubbard} the only relevant contribution was given by $\mathbf{K}=\mathbf{0}$. In the next sections, the results of the self-consistent HF algorithm are exposed.

\section{Results of the HF algorihtm}\label{sec:mft-analysis-hf-results}

\todo
